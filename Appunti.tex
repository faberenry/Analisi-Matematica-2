\documentclass{report}
\usepackage{mathtools}
\usepackage{mathbbol}
\usepackage{enumitem}
\usepackage{amssymb}
\usepackage{amsmath}
\usepackage{graphicx}
\usepackage{hyperref}
\usepackage{blindtext}
\usepackage{nicematrix}
\usepackage{booktabs}
\usepackage{amsfonts}
\usepackage{pgfplots}
\usepackage{dutchcal}
\usepackage{bbm}
\usepackage{listings}
\usepackage{tikz} 

\newtheorem{theorem}{Teorema}[section]
\newtheorem{proof}{\textit{Dim.}}[section]
\newtheorem{example}{Esempio}
\newtheorem{proposition}{Proposizione}[section]
\newtheorem{exercise}{Esercizio}
\newtheorem{eexercise}{Esercizio}[section]
\newtheorem{definition}{Definizione}[section]
\newtheorem{osservazione}[theorem]{Osservazione}

\newcommand{\sol}{\textbf{Soluzione:}}
\newcommand{\ac}[1]{\`#1}
\newcommand{\ace}{\`e }
\newcommand{\aci}{\`i }
\newcommand{\aca}{\`a }
\newcommand{\aco}{\`o }
\newcommand{\acu}{\`u }
\newcommand{\Ins}[1]{\mathbb{#1}}
\newcommand{\R}{\Ins{R}}
\newcommand{\f}{f: A \subseteq \R^2 \to \R}
\newcommand{\fn}{f: A \subseteq \R^n \to \R}
\newcommand{\frestr}[1]{f\lvert_{#1}}
\newcommand{\ppartx}{\frac{\partial f}{\partial x}(p_0)}
\newcommand{\pparty}{\frac{\partial f}{\partial y}(p_0)}

\title{Analisi Matematica 2}
\author{Enrico Favretto}
\date{28/02/2022}

\begin{document}
  \maketitle
  \tableofcontents
  \clearpage
  \chapter{Funzioni a pi\acu variabili}
\section{Lez - 01}
Studieremo funzioni a pi\acu variabili reali a valori scalari e vettoriali, cio\ace 
$f : A \subseteq \Ins{R}^n \to \Ins{R}^k$ con $n, k \in Ins{N}$ e $n \geq 1, k \geq 1$. \\
Se $k = 1, n \geq 2$, $f$ si dice \underline{funzione di pi\acu variabili a valori scalari}; \\
Se $k \geq 1, n \geq 1$, $f$ si dice \underline{funzione di pi\acu variabili a valori vettoriali}.\\\\
Incominciamo a trattare il caso in cui $n = 2,3$ e $k = 1$.\\\\
\underline{MOTIVAZIONE}: I fenomenti in Fisica/Ingegneria sono modelizzati da funzioni che dipendono da due/tre variabili. 
\begin{example}
  \begin{enumerate}
    \item La funzione temperatura di una piastra piana $A \subseteq \Ins{R}^2$. \\
          La funzione temperatura della piastra A pu\aco essere modelizzata da una funzione 
          $$T : A \subseteq \Ins{R}^2 \to [0, +\infty] \subseteq \Ins{R}$$
          $$\Ins{R}^2 := \Ins{R} \times \Ins{R} = \{(x,y)\mid x \in \R, y \in \R\}$$
    \item La funzione distanza dall'origine in $\R^3$, $$f: \R^3 \to [0,+\infty]$$
          $$f(p) := d(O, p) = \sqrt{x^2+y^2+z^2}$$
          $$\Ins{R}^3 := \Ins{R} \times \Ins{R} \times \R= \{(x,y,z)\mid x, y, z \in \R\}$$
  \end{enumerate}
\end{example} 
\subsection{Grafico di una funzione scalare di pi\acu variabili}
Ricordiamo che nel caso di una funzione scalare da una variabile $f: A \subseteq \R \to \R$ ($y = f(x)$, $x \in A$), 
$A$ intervallo di $\R$.
$$G_f := \{(x,f(x))\mid x \in A\} \subseteq \R^2$$
Se $f: A \subseteq \R^2 \to \R$ ($z = f(x,y)$, $(x,y) \in A$)
$$G_f := \{(x,y,f(x,y))\mid (x,y) \in A\} \subseteq \R^3$$
$f: A \subseteq \R^3 \to \R$ ($t = f(x,y,z)$, $(x,y,z) \in A$)
$$G_f := \{(x,y,z,f(x,y,z))\mid (x,y,z) \in A\} \subseteq \R^4$$
Disegnare $G_f$ in $\R^4$? Non pu\aco essere facilmente studiato, il grafico \ace una ipersuperficie di $\R^4$
\subsection{Curve di livello di una funzione di pi\acu variabili}
Sia $f: A \subseteq \R^2 \to \R$, fissato $t \in \R$, 
$$C_t := \{(x,y) \in A \mid f(x,y) = t\}$$
(\ace un insieme di tipo "curva" contenuto in A)
\begin{example}
  $f : \R^2 \to \R$, $f(x,y) := x-y$, (z = x-y) $x-y-z = 0$, $$((1,-1,-1),(x,y,z)) = 0$$
  $$C_t := \{(x,y) \in \R^2 \mid x-y = t\}$$ fascio di rette parallele al variare di t
  $$G_f := \{(x,y,x-y) \mid x,y \in \R\}$$ piano di $\R^3$ contenente la retta $r$ e ortogonale
  al vettore (1,-1,-1)
  $$r:= \{(x,y)\in \R^2\mid x-y = 0\}$$
  Pi\acu in generale se $f: A \subseteq \R^3 \to \R$, $C_t := \{(x,y,z) \in A \mid f(x,y,z) = t\}$ \ace un insieme di tipo 
  "superficie". 
\end{example} 
\begin{exercise}
  Studiare le curve di livello della funzione $f:\R^2 \to \R$, $f(x,y) = x^2 + y^2$.
  $$C_t := \{(x,y)\in \R^2 \mid x^2 + y^2 = t\}$$
  \begin{itemize}
    \item $C_t$ \ace la circonferenza di centro $(0,0)$ e raggio $\sqrt{t}$, se $t\geq 0$
    \item $C_t$ \ace vuoto ($\varnothing$), se $t < 0$
  \end{itemize}
\end{exercise}
\subsection{Limiti e continuit\aca per funzioni di pi\acu variabili}
\underline{Problema}: Data $f:A \subset \R^2 \to \R$, fissato $(x_0, y_0)\in \R^2$ introdurre la definizione 
$$\lim_{(x,y)\to (x_0,y_0)} f(x,y) = L$$
Ricordiamo la definizione di limite per funzioni reali di una variabile, $f:(a,b)\to \R$, $x_0 \in [a,b]$
$lim_{x\to x_0} f(x) = L \in \R \iff (def.)$, 
$$\forall \varepsilon > 0, \exists \delta = d(x_0, \varepsilon) > 0 \mid \lvert f(x)-L \rvert < \varepsilon$$ 
$\forall x \in (a,b) \cap (x_0-\delta, x_0+\delta), x \not = x_0$, $lim_{x\to a^{+}} f(x) = L, lim_{x\to b^{-}} f(x) = L$
$$B(x_0,\delta) := (x_0-\delta, x_0+\delta) = \{x\in \R \mid \lvert x-x_0\rvert < \delta\}$$
\textit{intorno sferico di centro $x_0$ e reaggio $\delta > 0$}
\subsubsection{Idea per l'introduzione di limite per funzioni di $n=2$ varaibili}
\underline{Generalizzazione}:
\begin{enumerate}
  \item La definizione di intorno di centro $x_0$ e raggio $r > 0$ a $\R^2$
  \item La nozione di intervallo apero e chiuso a $\R^2$, come pure la nozione di punto 
        estremo di un intervallo.
\end{enumerate}

  \section{Lez - 02}
\begin{definition}[Distanza Euclidea in $\R^2$]
  Si chiama \underline{distanza euclidea} di $\R^2$ (o nel piano) la funzione, 
  $d: \R^2 \times \R^2 \to [0,+\infty)$:
  $$d(p,q) := \sqrt{(x_1-x_2)^2 + (y_1-y_2)^2}$$
  $p=(x_1,y_1)$, $q=(x_2,y_2)$
\end{definition}
\begin{definition}
  Si chiama \underline{intorno} (sferico) di centro $p_0 = (x_0,y_0) \in \R^2$ e raggio $r>0$ (o anche
  palla aperta di centro $p_0$ e raggio $r>0$), l'insieme: 
  $$B_r(p_0) = B(p_0, r) :=  \{p \in \R^2 \mid d(p,p_0) < r\} = $$
    $$= \{(x,y)\in \R^2 \mid (x-x_0)^2 + (y-y_0)^2 < r^2\}$$ 
\end{definition}
\begin{definition}
  Sia $A \subseteq \R^2$
  \begin{enumerate}
    \item Un punto $p_0 \in \R^2$ si dice \underline{punto di frontiera} di A se 
    $$B(p_0,r)\cap A \not = \varnothing \text{ e } B(p_0,r) \cap (\R^2 \setminus A) \not = \varnothing, \forall r > 0$$
    L'insieme di tutti i punti di frontiera di A \ace detto \underline{frontiera di A} e di denota $\partial A$
    \item L'insieme A \ace detto \underline{chiuso} se ogni punto di frontiera di A appartiene ad A
    \item L'insieme A \ace detto \underline{aperto} se non contiene alcun punto della sua frontiera
    \item L'insieme di tutti i punti di A che non sono di frontiera si chiama \underline{parte interna di A} e si denota con 
          $\mathring{A}$
    \item L'insieme A \ace detto \underline{limitato} se $\exists R_0 > 0 $ t.c. $A \subseteq B(O, R_0)$
  \end{enumerate}
\end{definition}
\begin{example}
  \begin{enumerate}
    \item $A =\{(x,y)\in \R^2 \mid x^2 + y^2 \leq 1\}$, allora
          \begin{itemize}
            \item $\partial A = \{(x,y)\in \R^2 \mid x^2 + y^2 = 1\}$
            \item $\mathring{A} = \{(x,y)\in \R^2 \mid x^2 + y^2 < 1\}$
          \end{itemize}
    \item $A = \R^2$, $\partial A = \varnothing$, $\mathring{A} = A = \R^2$
  \end{enumerate}
\end{example}
\begin{definition}
  Dato $A \subseteq \R^2$
  \begin{enumerate}
    \item $p_0 \in \R^2$ si dice \underline{punto di accomulazione} per A se 
          $$B(p_0,r) \cap (A\setminus \{p_0\}) \not = \varnothing, \forall r > 0$$
    \item $p_0 \in A$ si dice \underline{punto isolato} di A se $p_0$ non \ace un punto di 
          accomulazione, cio\ace se:
          $$\exists r_0 > 0 \mid B(p_0,r_0) \cap A = \{p_0\}$$
  \end{enumerate}
\end{definition}
\begin{definition}[Limite di funzioni di due variabili]
  Sia $\f$ e sia $p_0 \in \R^2$ punto di accomulazione per A. Si dice che:
  $$\exists lim_{(x,y)\to (x_0,y_0)} f(x,y) = L \in \R$$
  oppure $\exists \lim_{p \to p_0} f(p) = L$ se 
  $$\forall \varepsilon > 0, \exists \delta = d(p_0,\varepsilon) > 0 \mid 
  \lvert f(x,y)-L\rvert < \varepsilon, \forall (x,y) \in B(p,\delta) \cap (A \setminus \{p_0\})$$
\end{definition}
\begin{osservazione}
  Tenendo presente il caso di funzioni di una variabile, si pu\aco enunciare anche la definizione nel caso in cui $L = \pm \infty$
\end{osservazione}
\subsection{Calcolo dei limiti}
\begin{proposition}[Unicit\aca del limite]
  Sia $\f$ e sia $p_0 \in \R^2$ punto di accomulazione per A. Supponiamo che 
  $\exists lim_{p \to p_0} f(p) = L \in \R$. Allora $L$ \ace \underline{unico}.
\end{proposition}
\begin{theorem}[Tecniche per il calcolo dei limiti]
  Siano $g,\f$, $p_0 \in \R^2$ punto di accomulazione per A. Supponiamo che 
  $\exists \lim_{p\to p_0} f(p) = L \in \R$ e $\exists \lim_{p\to p_0} g(p) = M \in \R$, allora:
  \begin{enumerate}
    \item $\exists \lim_{p\to p_0} f(p) + g(p)= L + M$
    \item $\exists \lim_{p\to p_0} f(p) \cdot g(p)= L \cdot M$
    \item Se $g(p) \not = 0, \forall p \in A\setminus \{p_0\}$ e $M \not = 0$, allora $\exists \lim_{p\to p_0} \frac{f(p)}{g(p)} = \frac{L}{M}$
    \item Sia $F:\R to \R$ continua e sia $h(p) = F(f(p))$, allora $\exists  \lim_{p\to p_0} h(p) = F(L)$
    \item \textbf{Teorema del confronto}: Sia $h,g,\f$, supponiamo che:
          \begin{itemize}
            \item[5.1] $f(p) \leq g(p) \leq h(p)$, $\forall p \in A \setminus \{p_0\}$
            \item[5.2] $\exists\lim_{p \to p_0} f(p) = \lim_{p \to p\to p_0} h(p) = L \in \R \cup \{\pm \infty\}$
          \end{itemize}
          allora $\exists \lim_{p \to p_0} g(p) = L$
  \end{enumerate}
\end{theorem}
\begin{proof}
  Le dimostrazioni di 1-4 sono lasciate al lettore :)
  \begin{itemize}
    \item[5] Supponiamo che $L \in \R$, dobbiamo provare che $\exists \lim_{p\to p_0} g(p) = L$, cio\ace per definizione:
    \begin{itemize}
      \item[1*] $\forall \varepsilon > 0 $, $\exists \delta \left(=\delta(p_0, \varepsilon)\right) > 0$ t.c. 
                  $\lvert g(p)-L\rvert < \varepsilon$ $\forall p \in B(p_0,\delta) \cap (A \setminus \{p_0\})$.
                  Per ipotesi sappiamo che 
                  $$\lim_{p\to p_0} f(p) = L, \lim_{p\to p_0} h(p) = L $$
                  cio\ace: 
      \item[2*] $\forall \varepsilon > 0 $, $\exists \delta_1 \left(=\delta_1(p_0, \varepsilon)\right) > 0$ t.c. 
                $\lvert f(p)-L\rvert < \varepsilon$ o equivalentemente 
                $L - \varepsilon < f(p) < L + \varepsilon$ $\forall p \in B(p_0,\delta_1) \cap (A \setminus \{p_0\})$, e:
      \item[3*] $\forall \varepsilon > 0 $, $\exists \delta_2 \left(=\delta_2(p_0, \varepsilon)\right) > 0$ t.c. 
                $\lvert h(p)-L\rvert < \varepsilon$ o equivalentemente
                $L - \varepsilon < h(p) < L + \varepsilon$ $\forall p \in B(p_0,\delta_2) \cap (A \setminus \{p_0\})$
    \end{itemize} 
    Da (5.1),(2*),(3*) segue che $\forall \varepsilon > 0$, scegliendo $\delta = \min\{\delta_1,\delta_2\}$ vale che 
    $$L - \varepsilon < f(p) \leq g(p) \leq h(p) < L+\varepsilon$$ $\forall p \in B(p_0,\delta) \cap (A \setminus \{p_0\})$ 
    e dunque vale la (1*).
  \end{itemize}
\end{proof}
Introduciamo un altro strumento importante per il calcolo dei limiti per funzioni di due variabili. \\
Ricordiamo che data $f: A \subseteq \R^n \to \R$ e $B \subseteq A$ si chiama \underline{funzione restrizione}
$f\lvert_{B} : B \to \R$, $\frestr{B}(x) := f(x)$ se $x\in B$.
\begin{theorem}[Limite lungo direzioni]
  Siano $\f$ e $p_0 \in \R^2$ punto di accomulazione, allora sono equivalenti
  \begin{enumerate}
    \item $\exists \lim_{p\to p_0} f(p) = L$
    \item Per ogni sottoinsieme $B \subseteq A$, per cui $p_0$ \ace un punto di accomulazione per $B$,
          $\exists \lim_{p\to p_0} \frestr{B}(p) = L$
  \end{enumerate}
\end{theorem}
Un insieme $B\subseteq A$ pu\aco essere visto come una direzione lungo cui $p \to p_0$.
\begin{osservazione}
  Il teorema precedente risulta efficace \underline{solo} per provare che il limite \underline{non} esiste.
\end{osservazione}
\subsection{Esempi calcolo limiti}
\begin{exercise}
  \begin{enumerate}
    \item Calcola, se esiste, $\lim_{(x,y)\to (0,0)} \frac{\sin(x^2+y^2)}{x^2+y^2} = 1$
    \begin{proof}
      Nel calcolo del limite bisogna valutare:
      \begin{itemize}
        \item Esistenza (il limite pu\aco non esistere)
        \item Tecninche appropriate per il calcolo
      \end{itemize}
      Utilizziamo il punto (4) del primo teorema. 
      \\Ricordiamo anche il limite notevole $\lim_{t\to 0} \frac{\sin{t}}{t} = 1$\\
      Denotiamo:
      \begin{itemize}
        \item $h(x,y) = \frac{\sin(x^2+y^2)}{x^2+y^2}$ se $(x,y) \in A = (\R^2 \setminus \{(0,0)\})$
        \item $t = x^2 + y^2$
        \item Sia $p_0 = (0,0)$ punto di accomulazione per A.
      \end{itemize}
      Osserviamo che $h(x,y) = F(f(x,y))$, dove $F:\R\to\R$
      $$F:= \left\{ \begin{array}{cl}
        \frac{\sin{t}}{t} & t\not = 0 \\
        1 & t = 0 \\
      \end{array}\right.$$
      \ace continua, e $f(x,y) = x^2 + y^2$ $(x,y) \in \R^2$. \\
      Poich\ace $\lim_{(x,y)\to(0,0)} f(x,y) = 0$, dal punto (4)
       $$\exists \lim_{p \to p_0} h(p) = \lim_{p\to p_0} F(f(p)) = F(0) = 1$$
    \end{proof}
    \item Calcola se esite $\lim_{(x,y)\to (0,0)} \frac{xy}{x^2+y^2}$
    \begin{proof}
      Sia $$f(x,y) = \frac{xy}{x^2+y^2}$$ $\forall (x,y)\in A = \R^2\setminus \{(0,0)\}$ e $p_0 = (0,0)$.\\
      Utilizziamo il teorema per provare che il limite non esiste.\\
      Infatti se $$\exists \lim_{(x,y)\to (0,0)} f(x,y) = L$$
      allora\\ (1*) $\exists \lim_{x\to 0} f(x,mx) = L$, $\forall m \in R$\\ dove 
      $y = mx$, $B = \{y=mx\}$(direzionale) e $m$ \ace finito.\\
      \underline{Osserviamo} che $f(x,mx) = \frac{mx^2}{(m^2+1)x^2} = \frac{m}{m^2+1}$ se $x\not = 0$, 
      quindi $$\lim_{x\to 0}f(x,mx) = \frac{m}{m^2+1}$$
      ma se $m = 0,1$ il limite prende valore $0, \frac{1}{2}$ ($0 \not = \frac{1}{2}$),\\
      dunque non pu\aco valere (1*), quindi il limite \underline{non esiste}
    \end{proof}
  \end{enumerate}
\end{exercise}
Dalla definizione di limite per funzioni di due variabili segue subito la nozione di continuit\aca.
\begin{exercise}
  Calcolare se esiste $$\lim_{(x,y)\to(0,0)}\frac{x^2y}{x^4+y^2}$$
  Sugg: Provare che $\not \exists$
\end{exercise}
  \section{Lez - 03}
\subsection{Definizioni limiti e continuit\aca per $\R^n$}
\begin{definition}
  Sia $\f$
  \begin{enumerate}
    \item f si dice continua in $p_0 \in A$ se 
    \begin{enumerate}
      \item $p_0$ \ace un punto \underline{isolato} di A, oppure
      \item $p_0$ \ace un punto di accomulazione ed $\exists \lim_{p \to p_0} f(p) = f(p_0)$
    \end{enumerate}
    \item f si dice \underline{continua} su A se f \ace continua in ogni punto $p_0 \in A$
  \end{enumerate}
\end{definition}
Le nozioni di limite e continuit\aca, introdotte per funzioni $\f$, si possono estendere
al caso di funzioni $\fn$ con $n\geq 3$.\\
Pi\acu precisamente su $\R^n$ possiamo definire la distanza Euclidea:
$$d(p,q) = \sqrt{(x_1-y_1)^2+...+(x_n-y_n)^2}$$
se $p = (x_1, ..., x_n)$ e $q = (y_1, ..., y_n)$. \\\\
\underline{Intorno} di centro $p_0 = (x_1^0, ..., x_n^0)$ e $r>0$ \ace l'insieme:
$$B(p_0,r) = \{p \in \R^n \mid d(p,p_0) < r\}$$
$$= \{(x_1,...,x_n) \in \R^n \mid (x_1-x_1^0)^2+...+(x_n-x_n^0)^2 < r^2\}$$
Tramite la nozione di intorni, si possono estendere a $\R^n$ la nozione di:
\begin{itemize}
  \item frontiera di un insieme $A \subseteq \R^n$
  \item insieme aperto/chiuso $A \subseteq \R^n$
  \item insieme limitato $A \subseteq \R^n$
  \item punto di accomulazione/isolato di $A \subseteq \R^n$
\end{itemize}
Pertanto:
\begin{definition}
  Sia $\fn$ e sia $p_0 \in \R^n$ punto di accomulazione di A. Allora si dice che:
  $$\exists \lim_{p \to p_0} f(p) = L \in \R$$
  se 
  $$\forall \varepsilon > 0, \exists \delta = \delta(p,\varepsilon) > 0 \text{ t.c. } 
  \lvert f(p) - L \rvert < \varepsilon, \forall p \in B(p_0,\delta) \cap (A \setminus \{p_0\})$$
\end{definition}
In modo simile si pu\aco introdurre la nozione di continuit\aca per funzioni $\fn$.
\subsection{Calcolo differenziale per funzioni a pi\acu variabili}
\subsubsection{Derivate parziali}
Sia $\f$, A \underline{aperto}, $p_0 = (x_0,y_0) \in A$, essendo A aperto, 
$\exists \delta_0 > 0$ t.c. 
$$[x_0-\delta, x_0+\delta]\times [y_0-\delta, y_0+\delta] \subset A$$
In particolare i segmenti:
\begin{itemize}
  \item $(x,y_0) \in A$ $\forall x \in [x_0-\delta, x_0+\delta]$
  \item $(x_0,y) \in A$ $\forall y \in [y_0-\delta, y_0+\delta]$
\end{itemize}
Pertanto son ben definiti i rapporti incrementali
\begin{itemize}
  \item $\left((x_0-\delta_0, x_0+\delta_0) \setminus \{x_0\}\right) \ni x \rightarrow \frac{f(x,y_0) - f(x_0,y_0)}{x-x_0}$
  \item $\left((y_d0-\delta_0, y_0+\delta_0) \setminus \{y_0\}\right) \ni y \rightarrow \frac{f(x_0,y) - f(x_0,y_0)}{y-y_0}$
\end{itemize}
\begin{definition}
  \begin{enumerate}
    \item Si dice che $f$ \ace \underline{derivabile}(parzialmente) rispetto alla variabile x nel punto $p_0 = (x_0,y_0)$ se 
          $$\exists \lim_{x \to x_0} \frac{f(x,y_0) - f(x_0,y_0)}{x-x_0} := \frac{\partial f}{\partial x}(x_0,y_0) = D_1 f(x_0,y_0) \in \R$$
    \item Si dice che $f$ \ace \underline{derivabile}(parzialmente) rispetto alla variabile y nel punto $p_0 = (x_0,y_0)$ se 
          $$\exists \lim_{y \to y_0} \frac{f(x_0,y) - f(x_0,y_0)}{y-y_0} := \frac{\partial f}{\partial y}(x_0,y_0) = D_2 f(x_0,y_0) \in \R$$
    \item Se $f$ \ace derivabile (parzialmente) sia rispetto ad x ed y nel punto $p_0 = (x_0,y_0)$, si chiama (vettore)\underline{gradiente} di $f$ in $p_0$
          il vettore:
          $$\nabla f(p_0) = \left(\frac{\partial f}{\partial x}(p_0), \frac{\partial f}{\partial y}(p_0)\right) \in \R^2$$
  \end{enumerate}
  Sia $\f$, A insieme aperto. Supponiamo che:
  $$\exists \frac{\partial f}{\partial x},\frac{\partial f}{\partial y} : A \to \R$$
  allora \ace ben definito il \underline{campo} dei vettori gradiente:
  $$\nabla f : \R^2 \supseteq A \ni p \to \nabla f(p) = \left(\frac{\partial f}{\partial x}(p), \frac{\partial f}{\partial y}(p)\right) \in \R^2$$
\end{definition}
\underline{Applicazione}: Sia $V:A\to \R$ il potenziale di una carica elettrica in un insieme A del piano. Allora 
vale la realzione $\nabla V = \underline{E}$, dove $\underline{E} := (E_1(x,y),E_2(x,y)) \rightarrow $ vettore campo elettrico.\\\\
\underline{Problema}: $\exists\nabla f(p_0)$ \ace la nozione corretta di derivabilit\aca per funzioni di due variabili? 
Per esempio se $\exists\nabla f(p_0) \Rightarrow $ f \ace continua in $p_0$?
\begin{example}
  Sia $f: \R^2 \to\R$, $p_0 = (0,0)$ e
  $$f(x,y):= \left\{\begin{array}{cl}
    0 & \text{se } (x,y) = (0,0) \\
    \frac{xy}{x^2+y^2} & \text{se } (x,y) \not = (0,0) \\
  \end{array}\right.$$
  Abbiamo visto che: $\not \exists \lim_{p \to p_0} f(p) \Rightarrow $ f non \ace continua in $p_0$.\\
  D'altra parte:
  $$\frac{f(x,0)-f(0,0)}{x} = 0$$
  se $x\not = 0 \Rightarrow \exists \frac{\partial f}{\partial x}(0,0) = 0$
  $$\frac{f(0,y)-f(0,0)}{y} = 0$$
  se $y\not = 0 \Rightarrow \exists \frac{\partial f}{\partial y}(0,0) = 0$. \\
  Pertanto $\exists \nabla f(0,0) = (0,0)$  ma f non \ace continua nel punto (0,0).
\end{example}
\subsection{Piano tangente al grafico}
\textbf{Approssimazione lineare e nozione di differenziabilit\aca per funzioni di pi\acu variabili}. \\\\
Sia $f:\R^2 \to \R$, $p_0 = (x_0,y_0) \in \R^2$, $z=f(x,y)$. \\\\
\underline{Problema}: Definire il "piano tangente" alla "superficie" $G_f$ nel punto $(x_0,y_0,f(x_0,y_0))$ se esiste.\\
Ricordiamo che l'equazione di un piano $\pi$ di $\R^3$, non parallelo all'asse z, passante per il punto 
$(x_0,y_0,f(x_0,y_0))$ \ace del tipo 
$$\pi : z = a(x-x_0) + b(y-y_0) + f(x_0,y_0)$$
dove $a,b \in \R$.\\
Ricordiamo inoltre che per funzioni di $n=1$ variabile, se $f:(a,b) \to \R$, $x_0\in (a,b)$, la retta tangente $r$ a 
$G_f$ nel punto $(x_0,f(x_0))$ ha equazione:
$$r: y = f'(x_0)(x-x_0) + f(x_0)$$
ed \ace caratterizzata dalla propriet\aca di essere \underline{l'unica retta} del fascio di rette $y=m(x-x_0)+f(x_0)$, 
$m\in\R$ t.c. 
$$\text{(D)} \exists \lim_{x\to x_0}\frac{f(x)-\left[m(x-x_0)+f(x_0)\right]}{\lvert x-x_0\rvert} = 0$$
(miglior approssimazione lineare al primo ordine)
Infatti: $n=1$, $L(x) = ax$, $a\in\R$ sono le applicazioni lineari di $\R$ in $\R$
\begin{exercise}
  $\exists f'(x_0) \in \R \iff \exists m \in \R$ t.c. vale $(D)$, inoltre $m = f'(x_0)$. \\
  \underline{Sugg:} Utilizzare (D) nel caso di funzioni di due variabili per definire il paino tangente. \\
  Pi\acu precisamente, data $\f$ con A aperto, sia $p_0=(x_0,y_0)\in A$. Suppponimao che esistono $a,b \in \R$ t.c.
  $$\text{(D)} \exists \lim_{(x,y)\to (0,0)}\frac{f(x)-\left[a(x-x_0)+b(y-y_0+f(x_0)\right]}{\sqrt{(x-x_0)^2+(y-y_0)^2}} = 0$$
  Allora se vale (D:)
\end{exercise}
\begin{definition}
  \begin{enumerate}
    \item il piano $\pi : z = a(x-x_0)+b(y-y_0)+f(x_0,y_0)$ si dice \underline{piano tangente} al grafico $G_f$ nel punto $\left(x_0,y_0,f(x_0,y_0)\right)$
    \item f si dice \underline{differenziabile} nel punto $p_0 = (x_0,y_0)$ proveremo che:
      \begin{enumerate}
        \item[(a)] Se f \ace differenziabile in $p_0 \in A \Rightarrow $ f \ace continua
        \item[(b)] Se f \ace differenziale in $p_0 \in A$, allora 
                   $$\exists \frac{\partial f}{\partial x}(p_0), \exists \frac{\partial f}{\partial y}(p_0)$$   
      \end{enumerate}
  \end{enumerate}
\end{definition}
\begin{exercise}
  $!\exists \lim_{(x,y) \to (0,0)} \frac{xy}{(x^2+y^2)\sqrt{x^2+y^2}} = 0$? NO.
\end{exercise}

  \chapter{Esercitazioni}
\section{Lezione 1 - 09/03/2022}
\begin{eexercise}
  Determinare e disegnare nel piano xy il dominio delle seguenti funzioni, $\f$, dove A: dominio che dobbiamo determinare.
  $$f(x,y) = \log(4(x^2+y^2)-1)$$
  \sol $$4(x^2+y^2)-1 > 0 \iff x^2+y^2 > \frac{1}{4}$$
  Studiamo quindi: $x^2+y^2 = \frac{1}{4}$ la circonferenza di centro $c=(0,0)$ e raggio $r = \frac{1}{2}$,
  $$A = \{(x,y)\in \R^2 \mid x^2+y^2 > \frac{1}{4}\} = \R^2 \setminus \overline{B((0,0), \frac{1}{2})}$$
  dove:
  \begin{itemize}
    \item $\overline{B((0,0), \frac{1}{2})} = \{(x,y)\in\R^2 \mid \sqrt{x^2+y^2} \leq \frac{1}{2}\}$
    \item $B((0,0), \frac{1}{2}) = \{(x,y)\in\R^2 \mid \sqrt{x^2+y^2} < \frac{1}{2}\}$
  \end{itemize}
  \hfill\break 
  \textbf{Insiemi aperti e chiusi}\\
  $A = \{(x,y)\in \R^2\mid xy\geq 0\}$, A \ace chiuso $\iff A^c$ \ace aperto.\\
  Definiamo $\bar{A} = A$, $xy \geq 0 \iff \left\{\begin{array}{c}
    x \geq 0 \\
    y \geq 0 \\
  \end{array}\right. \vee \left\{\begin{array}{c}
    x \leq 0 \\
    y \leq 0 \\
  \end{array}\right.$
  Disegnando gli assi: \\
  $A^c = \R^2 \setminus A$ \ace aperto. Fisso ora $(x_0,y_0) \in A^c$, $r = d(\partial A, (x_0,y_0)) = \min{\lvert x_0 \rvert, \lvert y_0 \rvert}$. 
  La palla $B((x_0,y_0), \frac{r}{2}) \subset A^c \Rightarrow A^c $ \ace aperto $\Rightarrow A $ \ace chiuso.
\end{eexercise}
\begin{eexercise}
  $f(x,y) = \sqrt{y^2-x^4}$, $y^2 \geq x^4$. $$A=\{(x,y)\in\R^2 \mid y^2\geq x^4\}$$
  Proviamo a scrivere $y^2-x^4$ come 
  $$y^2-x^4 = (y-x^2)(y+x^2) \geq 0$$
  Due casi:
  \begin{itemize}
    \item $y \geq x^2$
    \item $y \geq -x^2$
  \end{itemize} 
  (Dal grafico otteniamo)
  $$A = \{(x,y)\in\R^2 \mid y \geq x^2 \vee y \leq -x^2\} = \{(x,y)\in\R^2 \mid y \geq x^2\} \cup \{(x,y)\in\R^2 \mid y \leq -x^2\}$$
\end{eexercise}
\begin{eexercise}
  Disegnare l'insieme di livello delle seguenti funzioni
  $$C_t = \{(x,y\in\R^2 \mid f(x,y) = t)\}$$
  con $t \in \R$.\\
  $f(x,y) = x^2y$, fissiamo $t \in \R$, $t = x^2y$
  \begin{enumerate}
    \item $t = 0$, $x^2y = 0 \Rightarrow y = 0 \vee x = 0$
    \item $t > 0$, $t = x^2y \iff y = \frac{t}{x^2}$
    \begin{itemize}
      \item $t = 1$, $y = \frac{1}{x^2}$
      \item $t = 2$, $y = \frac{2}{x^2}$
    \end{itemize}
    \item $t < 0$, $t = x^2y \iff y = \frac{t}{x^2}$
    \begin{itemize}
      \item $t = -1$, $y = -\frac{1}{x^2}$
      \item $t = -2$, $y = -\frac{2}{x^2}$
    \end{itemize}
  \end{enumerate}
\end{eexercise}
\begin{eexercise}
  $f(x,y) = ye^{-x}$, $t\in\R$, $t=ye^{-x} \iff e^x t = y$
  \begin{itemize}
    \item $t=0 \Rightarrow y = 0$
    \item $t=1 \Rightarrow y = e^{-x}$
    \item $t=2 \Rightarrow y = 2e^{-x}$
    \item $t=-1 \Rightarrow y = -e^{-x}$
    \item $t=-2 \Rightarrow y = -2e^{-x}$
  \end{itemize}
\end{eexercise}
\begin{eexercise}
  $$\lim_{(x,y)\to (0,0)} \frac{x-y}{\sqrt[3]{x}-\sqrt[3]{y}} = ?$$
  eleviamo x e y al numeratore per $\frac{3}{3}$, otteniamo:
  $$\lim_{(x,y)\to (0,0)} \frac{(\sqrt[3]{x})^3-(\sqrt[3]{y})^3}{\sqrt[3]{x}-\sqrt[3]{y}}$$
  Ricordiamo ora la differenza tra cubi $A^3 - B^3 = (A-B)(A^2+AB+B^2)$, otteniamo:
  $$\lim_{(x,y)\to (0,0)} \frac{(\sqrt[3]{x}-\sqrt[3]{y})\left((\sqrt[3]{x})^2 + \sqrt[3]{x}\sqrt[3]{y} + (\sqrt[3]{y})^2\right)}{\sqrt[3]{x}-\sqrt[3]{y}} = $$
  $$= \lim_{(x,y \to (0,0))} (\sqrt[3]{x})^2 + \sqrt[3]{x}\sqrt[3]{y} + (\sqrt[3]{y})^2 = 0$$
\end{eexercise}
\begin{eexercise}
  $$\lim_{(x,y)\to(0,0)} \frac{x^2y}{x^4+y^2}=?$$
  $\lim_{(x,y)\to(x_0,y_0)} f(x,y) = l \iff$ per ogni restrizione a un sottoinsieme $B$, $\lim_{(x,y)\to(x_0,y_0)} \frestr{B}(x,y) = l$
  \begin{itemize}
    \item $B=\{(x,y)\in\R^2\mid y = mx\}$, $\lim \frac{x^2y}{x^4+y^2} \lvert_{B} = \lim \frac{x^2mx}{x^4 + m^2x^2} = $
    $$= \frac{x^3m}{x^2(x^2+m^2)} = x\left(\frac{m}{x^2+m^2}\right) = \lim_{x\to 0} x\left(\frac{m}{x^2+m^2}\right) = 0$$
    \item $B=\{(x,y)\in\R^2\mid y = mx^2\}$, $\lim \frac{x^2y}{x^4+y^2} \lvert_{B} = $
    $$\lim_{x\to 0} \frac{mx^4}{x^4+m^2x^4} = \lim_{x \to 0} \frac{m}{1+m^2}$$
    Proviamo due valori di m:
    \begin{itemize}
      \item $m = 1$, $\frac{1}{2}$
      \item $m=2$, $\frac{2}{5}$
    \end{itemize} 
    Ho trovato due restrizioni $\{y = x^2\}$ e $\{y = 2x^2\}$ dove il limite assume due valori distinti. 
    Allora per l'unicit\aca del limite, il limite non esiste.
  \end{itemize}
\end{eexercise}
\begin{eexercise}
  $$\lim_{(x,y) \to (0,0)} \frac{x^2y}{x^2+y^2}$$
  \textbf{Coordinate polari}\\
  $\rho = \sqrt{x^2+y^2}$, $\vartheta = arctan\left(\frac{y}{x}\right)$
  \begin{itemize}
    \item $x = \rho \cos \vartheta$
    \item $y = \rho \sin \vartheta$
  \end{itemize}
  $$\lim_{(x,y) \to (0,0)} \frac{x^2y}{x^2+y^2} = \lim_{(x,y)\to (0,0)} \frac{\rho^2 \cos^2 \vartheta \cdot \rho \sin\vartheta}
  {\rho^2 \cos^2 \vartheta + \rho^2 \sin^2 \vartheta} = $$
  $$= \lim_{(x,y)\to (0,0)} \frac{\rho^3 \cos^2 \vartheta \cdot \sin \vartheta}{\rho^2 \left(\cos^2 \vartheta + \sin^2 \vartheta\right)}$$
  Sappiamo che $\cos^2 \vartheta + \sin^2 \vartheta = 1$, quindi il limite rimane:
  $$\lim \rho \cos^2 \vartheta \cdot \sin \vartheta$$
  $$0 \leq \lvert \rho \cos^2 \vartheta \cdot \sin \vartheta \rvert < \rho$$
  Da cui se $(x,y) \to (0,0)$ allora anche $\rho \to 0$ e siccome $\left\{\begin{array}{c}
    \cos^2 \vartheta < 1\\
    \sin \vartheta < 1 \\
  \end{array}\right.$, grazie al 
  \textbf{teorema del confronto} il limite vale 0.
\end{eexercise}
\begin{eexercise}
  Dire quali insiemi sono aperti/chiusi e quali limitati, inoltre determinare la frontiera.
  $$H=\{(x,y)\in \R^2 \mid (xy)(y-1)\geq 0\}$$
  \begin{itemize}
    \item $x\geq 0$
    \item $y\geq 0$
    \item $y-1\geq 0$, $y\geq 1$
  \end{itemize}
  Frontiera: $\partial H = \{y=1\} \cup \{x=0\} \cup \{y=0\}$
\end{eexercise}
\end{document}