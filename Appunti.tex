\documentclass{report}
\usepackage{mathtools}
\usepackage{mathbbol}
\usepackage{enumitem}
\usepackage{amssymb}
\usepackage{amsmath}
\usepackage{graphicx}
\usepackage{hyperref}
\usepackage{blindtext}
\usepackage{nicematrix}
\usepackage{booktabs}
\usepackage{amsfonts}
\usepackage{pgfplots}
\usepackage{dutchcal}
\usepackage{bbm}
\usepackage{listings}
\usepackage{tikz} 

\newtheorem{theorem}{Teorema}[section]
\newtheorem{corollary}{Corollario}[section]
\newtheorem{proof}{\textit{Dim.}}
\newtheorem{example}{Esempio}
\newtheorem{proposition}{Proposizione}[section]
\newtheorem{exercise}{Esercizio}
\newtheorem{eexercise}{Esercizio}[section]
\newtheorem{definition}{Definizione}[section]
\newtheorem{osservazione}[theorem]{Osservazione}
\newtheorem{notazione}[theorem]{Notazione}
\newtheorem{NB}[theorem]{N.B.:}

\newcommand{\sol}{\textbf{Soluzione:}}
\newcommand{\ac}[1]{\`#1}
\newcommand{\ace}{\`e }
\newcommand{\aci}{\`i }
\newcommand{\aca}{\`a }
\newcommand{\aco}{\`o }
\newcommand{\acu}{\`u }
\newcommand{\Ins}[1]{\mathbb{#1}}
\newcommand{\R}{\Ins{R}}
\newcommand{\f}{f: A \subseteq \R^2 \to \R}
\newcommand{\fn}{f: A \subseteq \R^n \to \R}
\newcommand{\frestr}[1]{f\lvert_{#1}}
\newcommand{\ppartx}{\frac{\partial f}{\partial x}(p_0)}
\newcommand{\pparty}{\frac{\partial f}{\partial y}(p_0)}
\newcommand{\abs}[1]{\left\lvert #1 \right\rvert}
\newcommand{\polarBase}{\begin{array}{l}
                          x = \rho\cos\vartheta \\
                          y = \rho\sin\vartheta \\
                        \end{array}}
\newcommand{\tuple}[1]{\left\langle #1 \right\rangle}
\newcommand{\norma}[1]{\left\lVert#1\right\rVert}
\newcommand{\origine}{\underline{O}}
\newcommand{\p}{\partial}
\newcommand{\overcirc}[1]{\mathring{#1}}
\newcommand{\V}{\mathsf{V}}
\newcommand{\T}{\mathsf{T}}
\newcommand{\Qbase}{Q = [a,b]\times[c,d]}
\newcommand{\D}{\mathcal{D}}
\newcommand{\Rcal}{\mathcal{R}}
\newcommand{\ftilde}{\widetilde{f}}
\newcommand{\g}{\gamma}
\newcommand{\gtilde}{\widetilde{g}}
\newcommand{\om}{\omega}
\newcommand{\al}{\alpha}
\newcommand{\bb}{\beta}
\newcommand{\U}{\mathcal{U}}
\newcommand{\Dint}{D_{int}}
\newcommand{\Dest}{D_{est}}

\title{Analisi Matematica 2}
\author{Enrico Favretto}
\date{28/02/2022}

\begin{document}
  \maketitle
  \tableofcontents
  \clearpage
  \chapter{Funzioni a pi\acu variabili, [BDPG, 10]}
\section{Lez - 01}
Studieremo funzioni a pi\acu variabili reali a valori scalari e vettoriali, cio\ace 
$f : A \subseteq \Ins{R}^n \to \Ins{R}^k$ con $n, k \in Ins{N}$ e $n \geq 1, k \geq 1$. \\
Se $k = 1, n \geq 2$, $f$ si dice \underline{funzione di pi\acu variabili a valori scalari}; \\
Se $k \geq 1, n \geq 1$, $f$ si dice \underline{funzione di pi\acu variabili a valori vettoriali}.\\\\
Incominciamo a trattare il caso in cui $n = 2,3$ e $k = 1$.\\\\
\underline{MOTIVAZIONE}: I fenomenti in Fisica/Ingegneria sono modelizzati da funzioni che dipendono da due/tre variabili. 
\begin{example}
  \begin{enumerate}
    \item La funzione temperatura di una piastra piana $A \subseteq \Ins{R}^2$. \\
          La funzione temperatura della piastra A pu\aco essere modelizzata da una funzione 
          $$T : A \subseteq \Ins{R}^2 \to [0, +\infty] \subseteq \Ins{R}$$
          $$\Ins{R}^2 := \Ins{R} \times \Ins{R} = \{(x,y)\mid x \in \R, y \in \R\}$$
    \item La funzione distanza dall'origine in $\R^3$, $$f: \R^3 \to [0,+\infty]$$
          $$f(p) := d(O, p) = \sqrt{x^2+y^2+z^2}$$
          $$\Ins{R}^3 := \Ins{R} \times \Ins{R} \times \R= \{(x,y,z)\mid x, y, z \in \R\}$$
  \end{enumerate}
\end{example} 
\subsection{Grafico di una funzione scalare di pi\acu variabili}
Ricordiamo che nel caso di una funzione scalare da una variabile $f: A \subseteq \R \to \R$ ($y = f(x)$, $x \in A$), 
$A$ intervallo di $\R$.
$$G_f := \{(x,f(x))\mid x \in A\} \subseteq \R^2$$
Se $f: A \subseteq \R^2 \to \R$ ($z = f(x,y)$, $(x,y) \in A$)
$$G_f := \{(x,y,f(x,y))\mid (x,y) \in A\} \subseteq \R^3$$
$f: A \subseteq \R^3 \to \R$ ($t = f(x,y,z)$, $(x,y,z) \in A$)
$$G_f := \{(x,y,z,f(x,y,z))\mid (x,y,z) \in A\} \subseteq \R^4$$
Disegnare $G_f$ in $\R^4$? Non pu\aco essere facilmente studiato, il grafico \ace una ipersuperficie di $\R^4$
\subsection{Curve di livello di una funzione di pi\acu variabili}
Sia $f: A \subseteq \R^2 \to \R$, fissato $t \in \R$, 
$$C_t := \{(x,y) \in A \mid f(x,y) = t\}$$
(\ace un insieme di tipo "curva" contenuto in A)
\begin{example}
  $f : \R^2 \to \R$, $f(x,y) := x-y$, (z = x-y) $x-y-z = 0$, $$((1,-1,-1),(x,y,z)) = 0$$
  $$C_t := \{(x,y) \in \R^2 \mid x-y = t\}$$ fascio di rette parallele al variare di t
  $$G_f := \{(x,y,x-y) \mid x,y \in \R\}$$ piano di $\R^3$ contenente la retta $r$ e ortogonale
  al vettore (1,-1,-1)
  $$r:= \{(x,y)\in \R^2\mid x-y = 0\}$$
  Pi\acu in generale se $f: A \subseteq \R^3 \to \R$, $C_t := \{(x,y,z) \in A \mid f(x,y,z) = t\}$ \ace un insieme di tipo 
  "superficie". 
\end{example} 
\begin{exercise}
  Studiare le curve di livello della funzione $f:\R^2 \to \R$, $f(x,y) = x^2 + y^2$.
  $$C_t := \{(x,y)\in \R^2 \mid x^2 + y^2 = t\}$$
  \begin{itemize}
    \item $C_t$ \ace la circonferenza di centro $(0,0)$ e raggio $\sqrt{t}$, se $t\geq 0$
    \item $C_t$ \ace vuoto ($\varnothing$), se $t < 0$
  \end{itemize}
\end{exercise}
\subsection{Limiti e continuit\aca per funzioni di pi\acu variabili}
\underline{Problema}: Data $f:A \subset \R^2 \to \R$, fissato $(x_0, y_0)\in \R^2$ introdurre la definizione 
$$\lim_{(x,y)\to (x_0,y_0)} f(x,y) = L$$
Ricordiamo la definizione di limite per funzioni reali di una variabile, $f:(a,b)\to \R$, $x_0 \in [a,b]$
$lim_{x\to x_0} f(x) = L \in \R \iff (def.)$, 
$$\forall \varepsilon > 0, \exists \delta = d(x_0, \varepsilon) > 0 \mid \lvert f(x)-L \rvert < \varepsilon$$ 
$\forall x \in (a,b) \cap (x_0-\delta, x_0+\delta), x \not = x_0$, $lim_{x\to a^{+}} f(x) = L, lim_{x\to b^{-}} f(x) = L$
$$B(x_0,\delta) := (x_0-\delta, x_0+\delta) = \{x\in \R \mid \lvert x-x_0\rvert < \delta\}$$
\textit{intorno sferico di centro $x_0$ e reaggio $\delta > 0$}
\subsubsection{Idea per l'introduzione di limite per funzioni di $n=2$ varaibili}
\underline{Generalizzazione}:
\begin{enumerate}
  \item La definizione di intorno di centro $x_0$ e raggio $r > 0$ a $\R^2$
  \item La nozione di intervallo apero e chiuso a $\R^2$, come pure la nozione di punto 
        estremo di un intervallo.
\end{enumerate}
  \section{Lez - 02}
\begin{definition}[Distanza Euclidea in $\R^2$]
  Si chiama \underline{distanza euclidea} di $\R^2$ (o nel piano) la funzione, 
  $d: \R^2 \times \R^2 \to [0,+\infty)$:
  $$d(p,q) := \sqrt{(x_1-x_2)^2 + (y_1-y_2)^2}$$
  $p=(x_1,y_1)$, $q=(x_2,y_2)$
\end{definition}
\begin{definition}
  Si chiama \underline{intorno} (sferico) di centro $p_0 = (x_0,y_0) \in \R^2$ e raggio $r>0$ (o anche
  palla aperta di centro $p_0$ e raggio $r>0$), l'insieme: 
  $$B_r(p_0) = B(p_0, r) :=  \{p \in \R^2 \mid d(p,p_0) < r\} = $$
    $$= \{(x,y)\in \R^2 \mid (x-x_0)^2 + (y-y_0)^2 < r^2\}$$ 
\end{definition}
\begin{definition}
  Sia $A \subseteq \R^2$
  \begin{enumerate}
    \item Un punto $p_0 \in \R^2$ si dice \underline{punto di frontiera} di A se 
    $$B(p_0,r)\cap A \not = \varnothing \text{ e } B(p_0,r) \cap (\R^2 \setminus A) \not = \varnothing, \forall r > 0$$
    L'insieme di tutti i punti di frontiera di A \ace detto \underline{frontiera di A} e di denota $\partial A$
    \item L'insieme A \ace detto \underline{chiuso} se ogni punto di frontiera di A appartiene ad A
    \item L'insieme A \ace detto \underline{aperto} se non contiene alcun punto della sua frontiera
    \item L'insieme di tutti i punti di A che non sono di frontiera si chiama \underline{parte interna di A} e si denota con 
          $\mathring{A}$
    \item L'insieme A \ace detto \underline{limitato} se $\exists R_0 > 0 $ t.c. $A \subseteq B(O, R_0)$
  \end{enumerate}
\end{definition}
\begin{example}
  \begin{enumerate}
    \item $A =\{(x,y)\in \R^2 \mid x^2 + y^2 \leq 1\}$, allora
          \begin{itemize}
            \item $\partial A = \{(x,y)\in \R^2 \mid x^2 + y^2 = 1\}$
            \item $\mathring{A} = \{(x,y)\in \R^2 \mid x^2 + y^2 < 1\}$
          \end{itemize}
    \item $A = \R^2$, $\partial A = \varnothing$, $\mathring{A} = A = \R^2$
  \end{enumerate}
\end{example}
\begin{definition}
  Dato $A \subseteq \R^2$
  \begin{enumerate}
    \item $p_0 \in \R^2$ si dice \underline{punto di accomulazione} per A se 
          $$B(p_0,r) \cap (A\setminus \{p_0\}) \not = \varnothing, \forall r > 0$$
    \item $p_0 \in A$ si dice \underline{punto isolato} di A se $p_0$ non \ace un punto di 
          accomulazione, cio\ace se:
          $$\exists r_0 > 0 \mid B(p_0,r_0) \cap A = \{p_0\}$$
  \end{enumerate}
\end{definition}
\begin{definition}[Limite di funzioni di due variabili]
  Sia $\f$ e sia $p_0 \in \R^2$ punto di accomulazione per A. Si dice che:
  $$\exists lim_{(x,y)\to (x_0,y_0)} f(x,y) = L \in \R$$
  oppure $\exists \lim_{p \to p_0} f(p) = L$ se 
  $$\forall \varepsilon > 0, \exists \delta = d(p_0,\varepsilon) > 0 \mid 
  \lvert f(x,y)-L\rvert < \varepsilon, \forall (x,y) \in B(p,\delta) \cap (A \setminus \{p_0\})$$
\end{definition}
\begin{osservazione}
  Tenendo presente il caso di funzioni di una variabile, si pu\aco enunciare anche la definizione nel caso in cui $L = \pm \infty$
\end{osservazione}
\subsection{Calcolo dei limiti}
\begin{proposition}[Unicit\aca del limite]
  Sia $\f$ e sia $p_0 \in \R^2$ punto di accomulazione per A. Supponiamo che 
  $\exists lim_{p \to p_0} f(p) = L \in \R$. Allora $L$ \ace \underline{unico}.
\end{proposition}
\begin{theorem}[Tecniche per il calcolo dei limiti]
  Siano $g,\f$, $p_0 \in \R^2$ punto di accomulazione per A. Supponiamo che 
  $\exists \lim_{p\to p_0} f(p) = L \in \R$ e $\exists \lim_{p\to p_0} g(p) = M \in \R$, allora:
  \begin{enumerate}
    \item $\exists \lim_{p\to p_0} f(p) + g(p)= L + M$
    \item $\exists \lim_{p\to p_0} f(p) \cdot g(p)= L \cdot M$
    \item Se $g(p) \not = 0, \forall p \in A\setminus \{p_0\}$ e $M \not = 0$, allora $\exists \lim_{p\to p_0} \frac{f(p)}{g(p)} = \frac{L}{M}$
    \item Sia $F:\R \to \R$ continua e sia $h(p) = F(f(p))$, allora $\exists  \lim_{p\to p_0} h(p) = F(L)$
    \item \textbf{Teorema del confronto}: Sia $h,g,\f$, supponiamo che:
          \begin{itemize}
            \item[5.1] $f(p) \leq g(p) \leq h(p)$, $\forall p \in A \setminus \{p_0\}$
            \item[5.2] $\exists\lim_{p \to p_0} f(p) = \lim_{p \to p\to p_0} h(p) = L \in \R \cup \{\pm \infty\}$
          \end{itemize}
          allora $\exists \lim_{p \to p_0} g(p) = L$
  \end{enumerate}
\end{theorem}
\begin{proof}
  Le dimostrazioni di 1-4 sono lasciate al lettore :)
  \begin{itemize}
    \item[5] Supponiamo che $L \in \R$, dobbiamo provare che $\exists \lim_{p\to p_0} g(p) = L$, cio\ace per definizione:
    \begin{itemize}
      \item[1*] $\forall \varepsilon > 0 $, $\exists \delta \left(=\delta(p_0, \varepsilon)\right) > 0$ t.c. 
                  $\lvert g(p)-L\rvert < \varepsilon$ $\forall p \in B(p_0,\delta) \cap (A \setminus \{p_0\})$.
                  Per ipotesi sappiamo che 
                  $$\lim_{p\to p_0} f(p) = L, \lim_{p\to p_0} h(p) = L $$
                  cio\ace: 
      \item[2*] $\forall \varepsilon > 0 $, $\exists \delta_1 \left(=\delta_1(p_0, \varepsilon)\right) > 0$ t.c. 
                $\lvert f(p)-L\rvert < \varepsilon$ o equivalentemente 
                $L - \varepsilon < f(p) < L + \varepsilon$ $\forall p \in B(p_0,\delta_1) \cap (A \setminus \{p_0\})$, e:
      \item[3*] $\forall \varepsilon > 0 $, $\exists \delta_2 \left(=\delta_2(p_0, \varepsilon)\right) > 0$ t.c. 
                $\lvert h(p)-L\rvert < \varepsilon$ o equivalentemente
                $L - \varepsilon < h(p) < L + \varepsilon$ $\forall p \in B(p_0,\delta_2) \cap (A \setminus \{p_0\})$
    \end{itemize} 
    Da (5.1),(2*),(3*) segue che $\forall \varepsilon > 0$, scegliendo $\delta = \min\{\delta_1,\delta_2\}$ vale che 
    $$L - \varepsilon < f(p) \leq g(p) \leq h(p) < L+\varepsilon$$ $\forall p \in B(p_0,\delta) \cap (A \setminus \{p_0\})$ 
    e dunque vale la (1*).
  \end{itemize}
\end{proof}
Introduciamo un altro strumento importante per il calcolo dei limiti per funzioni di due variabili. \\
Ricordiamo che data $f: A \subseteq \R^n \to \R$ e $B \subseteq A$ si chiama \underline{funzione restrizione}
$f\lvert_{B} : B \to \R$, $\frestr{B}(x) := f(x)$ se $x\in B$.
\begin{theorem}[Limite lungo direzioni]
  Siano $\f$ e $p_0 \in \R^2$ punto di accomulazione, allora sono equivalenti
  \begin{enumerate}
    \item $\exists \lim_{p\to p_0} f(p) = L$
    \item Per ogni sottoinsieme $B \subseteq A$, per cui $p_0$ \ace un punto di accomulazione per $B$,
          $\exists \lim_{p\to p_0} \frestr{B}(p) = L$
  \end{enumerate}
\end{theorem}
Un insieme $B\subseteq A$ pu\aco essere visto come una direzione lungo cui $p \to p_0$.
\begin{osservazione}
  Il teorema precedente risulta efficace \underline{solo} per provare che il limite \underline{non} esiste.
\end{osservazione}
\subsection{Esempi calcolo limiti}
\begin{exercise}
  \begin{enumerate}
    \item Calcola, se esiste, $\lim_{(x,y)\to (0,0)} \frac{\sin(x^2+y^2)}{x^2+y^2} = 1$
    \begin{proof}
      Nel calcolo del limite bisogna valutare:
      \begin{itemize}
        \item Esistenza (il limite pu\aco non esistere)
        \item Tecninche appropriate per il calcolo
      \end{itemize}
      Utilizziamo il punto (4) del primo teorema. 
      \\Ricordiamo anche il limite notevole $\lim_{t\to 0} \frac{\sin{t}}{t} = 1$\\
      Denotiamo:
      \begin{itemize}
        \item $h(x,y) = \frac{\sin(x^2+y^2)}{x^2+y^2}$ se $(x,y) \in A = (\R^2 \setminus \{(0,0)\})$
        \item $t = x^2 + y^2$
        \item Sia $p_0 = (0,0)$ punto di accomulazione per A.
      \end{itemize}
      Osserviamo che $h(x,y) = F(f(x,y))$, dove $F:\R\to\R$
      $$F:= \left\{ \begin{array}{cl}
        \frac{\sin{t}}{t} & t\not = 0 \\
        1 & t = 0 \\
      \end{array}\right.$$
      \ace continua, e $f(x,y) = x^2 + y^2$ $(x,y) \in \R^2$. \\
      Poich\ace $\lim_{(x,y)\to(0,0)} f(x,y) = 0$, dal punto (4)
       $$\exists \lim_{p \to p_0} h(p) = \lim_{p\to p_0} F(f(p)) = F(0) = 1$$
    \end{proof}
    \item Calcola se esite $\lim_{(x,y)\to (0,0)} \frac{xy}{x^2+y^2}$
    \begin{proof}
      Sia $$f(x,y) = \frac{xy}{x^2+y^2}$$ $\forall (x,y)\in A = \R^2\setminus \{(0,0)\}$ e $p_0 = (0,0)$.\\
      Utilizziamo il teorema per provare che il limite non esiste.\\
      Infatti se $$\exists \lim_{(x,y)\to (0,0)} f(x,y) = L$$
      allora\\ (1*) $\exists \lim_{x\to 0} f(x,mx) = L$, $\forall m \in R$\\ dove 
      $y = mx$, $B = \{y=mx\}$(direzionale) e $m$ \ace finito.\\
      \underline{Osserviamo} che $f(x,mx) = \frac{mx^2}{(m^2+1)x^2} = \frac{m}{m^2+1}$ se $x\not = 0$, 
      quindi $$\lim_{x\to 0}f(x,mx) = \frac{m}{m^2+1}$$
      ma se $m = 0,1$ il limite prende valore $0, \frac{1}{2}$ ($0 \not = \frac{1}{2}$),\\
      dunque non pu\aco valere (1*), quindi il limite \underline{non esiste}
    \end{proof}
  \end{enumerate}
\end{exercise}
Dalla definizione di limite per funzioni di due variabili segue subito la nozione di continuit\aca.
\begin{exercise}
  Calcolare se esiste $$\lim_{(x,y)\to(0,0)}\frac{x^2y}{x^4+y^2}$$
  Sugg: Provare che $\not \exists$
\end{exercise}
  \section{Lez - 03}
\subsection{Definizioni limiti e continuit\aca per $\R^n$}
\begin{definition}
  Sia $\f$
  \begin{enumerate}
    \item f si dice continua in $p_0 \in A$ se 
    \begin{enumerate}
      \item $p_0$ \ace un punto \underline{isolato} di A, oppure
      \item $p_0$ \ace un punto di accomulazione ed $\exists \lim_{p \to p_0} f(p) = f(p_0)$
    \end{enumerate}
    \item f si dice \underline{continua} su A se f \ace continua in ogni punto $p_0 \in A$
  \end{enumerate}
\end{definition}
Le nozioni di limite e continuit\aca, introdotte per funzioni $\f$, si possono estendere
al caso di funzioni $\fn$ con $n\geq 3$.\\
Pi\acu precisamente su $\R^n$ possiamo definire la distanza Euclidea:
$$d(p,q) = \sqrt{(x_1-y_1)^2+...+(x_n-y_n)^2}$$
se $p = (x_1, ..., x_n)$ e $q = (y_1, ..., y_n)$. \\\\
\underline{Intorno} di centro $p_0 = (x_1^0, ..., x_n^0)$ e $r>0$ \ace l'insieme:
$$B(p_0,r) = \{p \in \R^n \mid d(p,p_0) < r\}$$
$$= \{(x_1,...,x_n) \in \R^n \mid (x_1-x_1^0)^2+...+(x_n-x_n^0)^2 < r^2\}$$
Tramite la nozione di intorni, si possono estendere a $\R^n$ la nozione di:
\begin{itemize}
  \item frontiera di un insieme $A \subseteq \R^n$
  \item insieme aperto/chiuso $A \subseteq \R^n$
  \item insieme limitato $A \subseteq \R^n$
  \item punto di accomulazione/isolato di $A \subseteq \R^n$
\end{itemize}
Pertanto:
\begin{definition}
  Sia $\fn$ e sia $p_0 \in \R^n$ punto di accomulazione di A. Allora si dice che:
  $$\exists \lim_{p \to p_0} f(p) = L \in \R$$
  se 
  $$\forall \varepsilon > 0, \exists \delta = \delta(p,\varepsilon) > 0 \text{ t.c. } 
  \lvert f(p) - L \rvert < \varepsilon, \forall p \in B(p_0,\delta) \cap (A \setminus \{p_0\})$$
\end{definition}
In modo simile si pu\aco introdurre la nozione di continuit\aca per funzioni $\fn$.
\subsection{Calcolo differenziale per funzioni a pi\acu variabili}
\subsubsection{Derivate parziali}
Sia $\f$, A \underline{aperto}, $p_0 = (x_0,y_0) \in A$, essendo A aperto, 
$\exists \delta_0 > 0$ t.c. 
$$[x_0-\delta, x_0+\delta]\times [y_0-\delta, y_0+\delta] \subset A$$
In particolare i segmenti:
\begin{itemize}
  \item $(x,y_0) \in A$ $\forall x \in [x_0-\delta, x_0+\delta]$
  \item $(x_0,y) \in A$ $\forall y \in [y_0-\delta, y_0+\delta]$
\end{itemize}
Pertanto son ben definiti i rapporti incrementali
\begin{itemize}
  \item $\left((x_0-\delta_0, x_0+\delta_0) \setminus \{x_0\}\right) \ni x \rightarrow \frac{f(x,y_0) - f(x_0,y_0)}{x-x_0}$
  \item $\left((y_d0-\delta_0, y_0+\delta_0) \setminus \{y_0\}\right) \ni y \rightarrow \frac{f(x_0,y) - f(x_0,y_0)}{y-y_0}$
\end{itemize}
\begin{definition}
  \begin{enumerate}
    \item Si dice che $f$ \ace \underline{derivabile}(parzialmente) rispetto alla variabile x nel punto $p_0 = (x_0,y_0)$ se 
          $$\exists \lim_{x \to x_0} \frac{f(x,y_0) - f(x_0,y_0)}{x-x_0} := \frac{\partial f}{\partial x}(x_0,y_0) = D_1 f(x_0,y_0) \in \R$$
    \item Si dice che $f$ \ace \underline{derivabile}(parzialmente) rispetto alla variabile y nel punto $p_0 = (x_0,y_0)$ se 
          $$\exists \lim_{y \to y_0} \frac{f(x_0,y) - f(x_0,y_0)}{y-y_0} := \frac{\partial f}{\partial y}(x_0,y_0) = D_2 f(x_0,y_0) \in \R$$
    \item Se $f$ \ace derivabile (parzialmente) sia rispetto ad x ed y nel punto $p_0 = (x_0,y_0)$, si chiama (vettore)\underline{gradiente} di $f$ in $p_0$
          il vettore:
          $$\nabla f(p_0) = \left(\frac{\partial f}{\partial x}(p_0), \frac{\partial f}{\partial y}(p_0)\right) \in \R^2$$
  \end{enumerate}
  Sia $\f$, A insieme aperto. Supponiamo che:
  $$\exists \frac{\partial f}{\partial x},\frac{\partial f}{\partial y} : A \to \R$$
  allora \ace ben definito il \underline{campo} dei vettori gradiente:
  $$\nabla f : \R^2 \supseteq A \ni p \to \nabla f(p) = \left(\frac{\partial f}{\partial x}(p), \frac{\partial f}{\partial y}(p)\right) \in \R^2$$
\end{definition}
\underline{Applicazione}: Sia $V:A\to \R$ il potenziale di una carica elettrica in un insieme A del piano. Allora 
vale la realzione $\nabla V = \underline{E}$, dove $\underline{E} := (E_1(x,y),E_2(x,y)) \rightarrow $ vettore campo elettrico.\\\\
\underline{Problema}: $\exists\nabla f(p_0)$ \ace la nozione corretta di derivabilit\aca per funzioni di due variabili? 
Per esempio se $\exists\nabla f(p_0) \Rightarrow $ f \ace continua in $p_0$?
\begin{example}
  Sia $f: \R^2 \to\R$, $p_0 = (0,0)$ e
  $$f(x,y):= \left\{\begin{array}{cl}
    0 & \text{se } (x,y) = (0,0) \\
    \frac{xy}{x^2+y^2} & \text{se } (x,y) \not = (0,0) \\
  \end{array}\right.$$
  Abbiamo visto che: $\not \exists \lim_{p \to p_0} f(p) \Rightarrow $ f non \ace continua in $p_0$.\\
  D'altra parte:
  $$\frac{f(x,0)-f(0,0)}{x} = 0$$
  se $x\not = 0 \Rightarrow \exists \frac{\partial f}{\partial x}(0,0) = 0$
  $$\frac{f(0,y)-f(0,0)}{y} = 0$$
  se $y\not = 0 \Rightarrow \exists \frac{\partial f}{\partial y}(0,0) = 0$. \\
  Pertanto $\exists \nabla f(0,0) = (0,0)$  ma f non \ace continua nel punto (0,0).
\end{example}
\subsection{Piano tangente al grafico}
\textbf{Approssimazione lineare e nozione di differenziabilit\aca per funzioni di pi\acu variabili}. \\\\
Sia $f:\R^2 \to \R$, $p_0 = (x_0,y_0) \in \R^2$, $z=f(x,y)$. \\\\
\underline{Problema}: Definire il "piano tangente" alla "superficie" $G_f$ nel punto $(x_0,y_0,f(x_0,y_0))$ se esiste.\\
Ricordiamo che l'equazione di un piano $\pi$ di $\R^3$, non parallelo all'asse z, passante per il punto 
$(x_0,y_0,f(x_0,y_0))$ \ace del tipo 
$$\pi : z = a(x-x_0) + b(y-y_0) + f(x_0,y_0)$$
dove $a,b \in \R$.\\
Ricordiamo inoltre che per funzioni di $n=1$ variabile, se $f:(a,b) \to \R$, $x_0\in (a,b)$, la retta tangente $r$ a 
$G_f$ nel punto $(x_0,f(x_0))$ ha equazione:
$$r: y = f'(x_0)(x-x_0) + f(x_0)$$
ed \ace caratterizzata dalla propriet\aca di essere \underline{l'unica retta} del fascio di rette $y=m(x-x_0)+f(x_0)$, 
$m\in\R$ t.c. 
$$\text{(D)} \exists \lim_{x\to x_0}\frac{f(x)-\left[m(x-x_0)+f(x_0)\right]}{\lvert x-x_0\rvert} = 0$$
(miglior approssimazione lineare al primo ordine)
Infatti: $n=1$, $L(x) = ax$, $a\in\R$ sono le applicazioni lineari di $\R$ in $\R$
\begin{exercise}
  $\exists f'(x_0) \in \R \iff \exists m \in \R$ t.c. vale $(D)$, inoltre $m = f'(x_0)$. \\
  \underline{Sugg:} Utilizzare (D) nel caso di funzioni di due variabili per definire il paino tangente. \\
  Pi\acu precisamente, data $\f$ con A aperto, sia $p_0=(x_0,y_0)\in A$. Suppponimao che esistono $a,b \in \R$ t.c.
  $$\text{(D)} \exists \lim_{(x,y)\to (0,0)}\frac{f(x)-\left[a(x-x_0)+b(y-y_0+f(x_0)\right]}{\sqrt{(x-x_0)^2+(y-y_0)^2}} = 0$$
  Allora se vale (D:)
\end{exercise}
\begin{definition}
  \begin{enumerate}
    \item il piano $\pi : z = a(x-x_0)+b(y-y_0)+f(x_0,y_0)$ si dice \underline{piano tangente} al grafico $G_f$ nel punto $\left(x_0,y_0,f(x_0,y_0)\right)$
    \item f si dice \underline{differenziabile} nel punto $p_0 = (x_0,y_0)$ proveremo che:
      \begin{enumerate}
        \item[(a)] Se f \ace differenziabile in $p_0 \in A \Rightarrow $ f \ace continua
        \item[(b)] Se f \ace differenziale in $p_0 \in A$, allora 
                   $$\exists \frac{\partial f}{\partial x}(p_0), \exists \frac{\partial f}{\partial y}(p_0)$$   
      \end{enumerate}
  \end{enumerate}
\end{definition}
\begin{exercise}
  $!\exists \lim_{(x,y) \to (0,0)} \frac{xy}{(x^2+y^2)\sqrt{x^2+y^2}} = 0$? NO.
\end{exercise}
  \section{Lez - 04}
Piano tangente al grafico $G_f$ in un punto $\left(x_0,y_o,f(x_0,y_0)\right)$, per una funzione $\f$ \ace un piano 
$\pi$ di equazione $z=a(x-x_0)+b(y-y_0)+f(x_0,y_0)$ dove $p_0 = (x_0,y_0)\in A$, verificante la seguente equazione:
$$\text{(D)} \exists \lim_{(x,y)\to (0,0)}\frac{f(x)-\left[a(x-x_0)+b(y-y_0)+f(x_0)\right]}{d(p,p_0)}$$
dove $d(p,p_0) = \sqrt{(x-x_0)^2+(y-y_0)^2}$
\begin{definition}
  Dato $A \subseteq \R^2$ aperto e dato $p_0=(x_0,y_0)\in A$, la funzione $\f$ si dice \underline{differenziabile}
  nel punto $p_0$ se vale $(D)$, per $a,b \in \R$ opportuni.
\end{definition}
\begin{proposition}
  Se f \ace differenziabile nel punto $p_0 =(x_0,y_0)$, allora $$\exists\nabla f(p_0) = \left(\ppartx, \pparty\right)$$
  e $$a = \ppartx , b =\pparty $$
  \begin{proof}
    Supponiamo che f sia differenziabile in $p_0$, cio\ace che valga (D). \\
    Ponendo nella (D), $y = y_0$ otteniamo che:
    $$\exists \lim_{x\to x_0} \frac{f(x,y_0) \left[a(x-x_0)+f(x_0,y_0)\right]}{\lvert x -x_0\rvert} = 0$$
    $$\Rightarrow \exists\ppartx = a$$
    procediamo allo stesso modo, ponendo $x=x_0$ nella (D) e otteniamo $\pparty = b$
  \end{proof}
\end{proposition}
\begin{definition}
  L'applcazione lineare $L:\R^2\to\R^2$, $$L(x,y) := \ppartx x + \pparty y$$ si chima \underline{differenziale} di f in $p_0$, si denota
  con: $$L = df(p_0) := \ppartx dx + \pparty dy$$
\end{definition}
\begin{definition}[Piano tangente]
  Sia $\f$, A aperto con f differenziabile in $p_0$. Si chiama \underline{piano tangente} al grafico $G_f$ nel 
  punto $(x_0,y_0,f(x_0,y_0))$ il piano $\pi$ di equazione:
  $$z = \ppartx (x-x_0) + \pparty (y-y_0) + f(x_0,y_0)$$
\end{definition}
\begin{theorem}
  Sia $\f$, A aperto, f differenziabile in $p_0 \in A$, allora f \ace continua in $p_0$
  \begin{proof}
    $$f(p)-f(p_0) = \frac{f(p)-f(p_0)-df(p_0)(p-p_0)}{d(p,p_0)} \cdot d(p,p_0) + df(p_0)(p-p_0) = $$
    $$= \ppartx (x-x_0) + \pparty (y-y_0)$$
    Il tutto tende a 0 per $p\to p_0$.\\
    $$\Rightarrow \exists \lim_{p\to p_0} \left(f(p)-f(p_0)\right) = 0$$
  \end{proof}
\end{theorem}
\subsection{Differenziabilit\aca in $n\geq 3$}
Sia $\fn$, A aperto, $p_0 \in A$, $p = (x_1,...,x_n)$, $p_0 = (x_1^0, ..., x_n^0)$ possiamo definire
$$\exists \frac{\partial f}{\partial x_i}(p_0) := \lim_{h\to 0} \frac{f(p_0+he_i)-f(p_0)}{h}$$
dove $i = 1,...,n$, $e_i,...,e_n$ denota la base canonica di $\R^n$, cio\ace $e_i = (0,0,...,0,1_{\text{i-esimo elemento}},0,0,...,0)$\\
Diremo che 
$$\exists\nabla f(p_0) := \left(\frac{\partial f}{\partial x_1}(p_0), ..., \frac{\partial f}{\partial x_n}(p_0)\right)$$
\underline{gradiente di f in $p_0$}, se $\exists\frac{\partial f}{\partial x_i}(p_0)$, $\forall i = 1,...,n$
\begin{definition}
  f si dice \underline{differenziabile} in un punto $p_0 \in A$ se esiste un'\underline{applicazione lineare} $L :\R^n \to \R$
  t.c. 
  $$(D) \exists \lim_{p \to p_0} \frac{f(p) -f(p_0)-L(p-p_0)}{d(p.p_0)} = 0$$
  L'applicazione lineare $L:\R^n \to \R$ per cui valga (D) si denota con $L = df(p_0)$
\end{definition}
\begin{proposition}[11.4]
  Se f \ace differenziabile nel punto $p_0$ allora 
  \begin{itemize}
    \item[i] $\exists\nabla df(p_0)$
    \item[ii] $$df(p_0)(v) = \sum_{i=1}^{n} \frac{\partial f}{\partial x_i}(p_0)v_i := \nabla f(p_0) \cdot v$$
        se $v = (v_1,...,v_n)$
  \end{itemize}
\end{proposition}
\begin{osservazione}
  Se $v = e_i$, $\nabla f(p_0)\cdot e_i = \frac{\partial f}{\partial x_i} (p_0)$
\end{osservazione}
\begin{notazione}
  $df(p_0) := \sum_{i=1}^{n} \frac{\partial f}{\partial x_i}(p_0) dx_i$
\end{notazione}
\begin{osservazione}
  Dalla definizione di differenziabilit\aca nel caso $n=1$, segue che, se $A = (a,b)$, $x_0 \in A$, allora 
  \textbf{Esercizio 1.5, foglio 2:} 
  $$\exists f'(x_0) \iff \text{f \ace differenziabile in }x_0$$
\end{osservazione}
\begin{exercise}[1b, foglio 2]
  Calcolare se esiste $$\lim_{(x,y) \to (0,0)} \frac{1-e^{xy^2}}{\sqrt{x^4+y^4}}$$
  \begin{proof}
    Ricordiamo che (1) $\exists \lim_{t\to 0} \frac{e^t-1}{t} = 1$. \\
    Utilizzando il precedente limite possiamo eseguire il seguente bilanciamento:
    $$\frac{1-e^{xy^2}}{xy^2}\cdot \frac{xy^2}{\sqrt{x^4+y^4}}$$
    $\forall (x,y) \in \R^2$, con $xy^2 \not = 0$. Osserviamo che:
    \begin{itemize}
      \item[(2)] $$\frac{1-e^{xy^2}}{\sqrt{x^4+y^4}} = 0$$
                Se $xy^2 \not = 0$ e $(x,y)\not = (0,0)$
      \item[(3)] $\lim_{(x,y)\to (0,0)}\frac{1-e^{xy^2}}{xy^2} = 1$. \\
      Rimane da calcolare, se esiste: 
      \item[(4)]  $\lim_{(x,y) \to (0,0)} \frac{xy^2}{\sqrt{x^4+y^4}}$
    \end{itemize}
    \ace molto utile, per studiare limite del tipo (4) fare un cambiamento di variabili ed utilizzare le coordinate polari:\\\\
    \textbf{Coordinate polari}\\
    Consideraimo il seguente cambiameto di variabili $\left\{\begin{array}{c}
      x = \rho\cos\vartheta \\
      y = \rho\sin\vartheta \\
    \end{array}\right.$
    con $\rho > 0$ e $0 \leq \vartheta \leq \pi$, quindi:
    $$\frac{xy^2}{\sqrt{x^4+y^4}} \rightarrow \frac{\rho\cdot\cos\vartheta \cdot \rho^2\sin^2\vartheta}
    {\sqrt{\rho^4\left(\cos^4\vartheta +\sin^4\vartheta\right)}} = \rho\cdot \frac{\cos\vartheta \cdot \sin^2\vartheta}
    {\sqrt{\left(\cos^4\vartheta +\sin^4\vartheta\right)}}$$
    Dalla (2) sappiamo che se $\exists \lim_{(x,y)\to (0,0)} \frac{1-e^{xy^2}}{\sqrt{x^4+y^4}} = L \Rightarrow L = 0$.\\
    \underline{Idea}: Utilizzare la funzione in coordinate polari, per cercare di provare tramite il 
    teorema del confronto che (5) $\exists \lim_{(x,y)\to (0,0)}\frac{xy^2}{\sqrt{x^4+y^4}}$. \\
    Le coordinate polari risulatano molto utili per trovare delle stime per applicare il 
    teorema del confronto:
    $$\text{(6) } 0\leq \lvert \frac{xy^2}{\sqrt{x^4+y^4}}\rvert = \lvert \rho\cdot \frac{\cos\vartheta \cdot \sin^2\vartheta}
    {\sqrt{\left(\cos^4\vartheta +\sin^4\vartheta\right)}} \rvert \leq$$
    $$\leq \rho\cdot \frac{\lvert \cos\vartheta \cdot \sin^2\vartheta \rvert}{\sqrt{\left(\cos^4\vartheta +\sin^4\vartheta\right)}}
     \leq \frac{\rho \cdot 1}{\sqrt{\cos^4\vartheta + \sin^4\vartheta}}$$
    \begin{exercise}
      $\cos^4\vartheta +\sin^4\vartheta \geq \frac{1}{2}$, $\forall \vartheta \in [0,2\pi]$ 
    \end{exercise}
    Pertanto da (6) segue che $$\left\{\begin{array}{cl}
      \vartheta > 0 & \forall \vartheta \in (0,2\pi) \\ 
      \\
      \frac{1}{\vartheta} > 0 & \vartheta \to 0^{+} \\
    \end{array}\right.$$
    $$0 \leq \lvert \frac{xy^2}{\sqrt{x^4+y^4}} \rvert \leq \sqrt{2}\cdot\rho = \sqrt{2}\cdot\sqrt{x^2+y^2}$$
    $\forall (x,y) \in \R^2 \setminus \{(0,0)\}$ se $(x,y) \to (0,0)$
    Dunque vale (5) e possiamo concludere che $$\exists \lim_{(x,y) \to (0,0)} \frac{1-e^{xy^2}}{\sqrt{x^4+y^4}} = 0$$
  \end{proof}
\end{exercise}
$$f(x,y) = \frac{1-e^{xy^2}}{\sqrt{x^4+y^4}}$$
se $(x,y)\not = (0,0)$, 
\begin{itemize}
  \item $!\exists \lim_{x\to 0} f(x,0) = 0$
  \item $!\exists \lim_{y\to 0} f(0,y) = 0$
\end{itemize}
$\Rightarrow \exists \lim_{(x,y)\to (0,0)}f(x,y) = L \Rightarrow L = 0$
\begin{proof}[1.5, foglio 2]
  $(\Rightarrow) \exists f'(p_0) \Rightarrow $ f \ace differenziabile in $x_0$. \\
  Ricordiamo che per definizione
  $$\exists f'(x_0) \in \R \iff (1) \exists \lim_{x\to x_0} \frac{f(x)-f(x_0)}{x-x_0} = f'(x_0) \in \R$$
  \textbf{N.B.}: $\lim_{x\to x_0} f(x) = 0 \iff \lim_{x\to x_0} \lvert f(x) \rvert= 0$
  \begin{exercise}
    $$(1) \iff (2) \exists \lim_{x\to x_0} \frac{f(x)-f(x_0)-f'(x_0)(x-x_0)}{\lvert x-x_0\rvert} = 0$$
    Osserviamo che per definizione f \ace differenziabile in $x_0 \iff $ vale (2). \\
    Mostriamo l'implicazione $(\Leftarrow)$, Supponiamo che valga (2). 
  \end{exercise}
  \begin{exercise}
    $$(2) \iff (3) \exists \lim_{x\to x_0} \frac{f(x)-f(x_0)-f'(x_0)(x-x_0)}{x-x_0} = 0$$
    \ac{E} chiaro che $$(3) \iff \exists \lim_{x\to x_0}\left(\frac{f(x)-f(x_0)}{x-x_0} - f'(x_0)\right) = 0 \iff$$
    $$\iff \exists \lim_{x\to x_0} \frac{f(x)-f(x_0)}{x-x_0} -f'(x_0) \xLeftrightarrow{def} \exists f'(x_0)$$
  \end{exercise}
\end{proof}
  \section{Lez - 05}
\subsection{Condizioni sulle derivate parziali che assicurino la diffrenenziablit\aca}
\begin{osservazione}
  La derivabilit\aca parziale non \ace sufficiente ad assicurare la diffrenenziablit\aca
\end{osservazione}
\textbf{Problema}: Data $\f$, A aperto e supponiamo che $\exists\nabla f(p_0)$ con $p_0\in A$.
Quale propriet\aca ulteriore bisogna aggiungere per ottenere la diffrenenziablit\aca di f in $p_0$?
\begin{theorem}[del differenziale totale]
  Sia $A\subseteq \R^2$ aperto, $p_0\in A$. Supponiamo che 
  \begin{enumerate}
    \item[(i)] $$\exists \frac{\partial f}{\partial x} , \frac{\partial f}{\partial y} : A \to \R$$
    \item[(ii)] $\frac{\partial f}{\partial x}, \frac{\partial f}{\partial y}$ siano continue nel 
              punto $p_0$, cio\ace $$\exists \lim_{p\to p_0} \frac{\partial f}{\partial x} (p) = \frac{\partial f}{\partial x} (p_0) \,
              e \lim_{p\to p_0} \frac{\partial f}{\partial y} (p) = \frac{\partial f}{\partial y} (p_0)$$
  \end{enumerate}
  Allora f \ace differenzibile nel punto $p_0$. \textit{[BDPG, 11.5]}
\end{theorem}
\begin{osservazione}
  \ac{E} sufficiente richiedre la (i) e (ii) in un intorno di $p_0$
\end{osservazione}
Il teorema del differenziale totale giustifica la seguente definizione:
\begin{definition}
  Sia $\f$
  \begin{enumerate}
    \item[(i)] f si dice \underline{differenziabile} su A se \ace diff su ogni punto di A.
    \item[(ii)] f si dice di \underline{classe $C^1$} su A se f \ace \underline{continua} e 
          $$\exists \frac{\partial f}{\partial x}, \frac{\partial f}{\partial y} : A \to \R \text{ continui}$$ 
          In questo caso scriveremo che $f \in C^1(A)$
  \end{enumerate}
\end{definition}
Dal teorema del differenziale totale segue anche:
\begin{corollary}
  Sia $f\in C^1(A)$ allora f \ace differenziabile su ogni punto di $p_0 \in A$
\end{corollary}
\subsection{Derivate direzionali}
\subsubsection{Norma di un vettore di $\R^n$}
Sia $v = (v_1,...,v_n)\in\R^n$, si chiama \underline{norma} di v, e si denota 
$$\norma{v} := \sqrt{v_1^2+...+v_n^2} = d(v,\origine) = \sqrt{v\cdot v}$$
\begin{example}
  \begin{enumerate}
    \item $n = 1$, $\norma{v} = \abs{v}$ se $v\in \R$
    \item $n = 2$. (immaginarsi il piano cartesiano)
  \end{enumerate}
\end{example}
\begin{osservazione}
  Se $p,q \in \R^n \Rightarrow d(p,q) = \norma{p-q}$
\end{osservazione}
\begin{exercise}[6,foglio 1]
  \begin{enumerate}
    \item $\norma{v} = 0 \iff v = \origine = (0,...,0)$
    \item Se $\lambda \in \R$ e $\lambda v = (\lambda v_1, ..., \lambda v_n)$ con $v = (v_1,...,v_n)$, 
          allora $\norma{\lambda v} = \abs{\lambda}\norma{v}$
    \item \underline{Disuguaglianza triangolare:} Se $v,w \in \R^n$, $\norma{v+w} \leq \norma{v} + \norma{w}$
  \end{enumerate}
\end{exercise}
\begin{definition}
  Un vettore $v \in \R^n$ si dice \underline{direzione} (\underline{vettore unitario}, \underline{versore}) se $\norma{v} = 1$
\end{definition}
\begin{example}
  $n=2$, i vettori $e_1 = (1,0)$ ed $e_2= (0,1)$ sono direzioni di $\R^2$
\end{example}
Sia $v = (v_1,v_2)\in \R^2$ una direzione, e $\f$, A aperto e $p_0 = (x_0,y_0)\in A$, allora 
$\exists \delta > 0$ t.c. $$p_0+hv = (x_0+hv_1,y_0+hv_2)\in A$$
se $\abs{h}\leq \delta$, pertanto \ace ben definita: 
$$ (-\delta, \delta) \setminus \{0\} \ni h \rightarrow \frac{f(p_0+hv) -f(p_0)}{h}$$
\begin{definition}
  Si dice che f \ace \underline{derivabile} (parzialmente) rispetto alla direzione v nel punto $p_0$
  se $$\exists \frac{\partial f}{\partial v} (p_0) := \lim_{h\to 0} \frac{f(p_0+hv) -f(p_0)}{h} \in \R$$
\end{definition}
\begin{notazione}
  Talvolta $\frac{\partial f}{\partial v} (p_0) = D_v f(p_0)$
\end{notazione}
\begin{osservazione}
  \begin{enumerate}
    \item[(i)] Sia $F : (-\delta,\delta) \to \R$, (funzione di $n=1$ variabile)
               $$F(t) := f(p_0+tv) \, \text{ se } t \in (-\delta,\delta)$$
               Allora \ace immediato verificare che 
               $$\exists \frac{\partial f}{\partial v} (p_0) \iff \exists F'(0) = \lim_{h\to 0} \frac{F'(h)-F(0)}{h}$$
               ed in questo case, $\frac{\partial f}{\partial v} (p_0) = F'(0)$
    \item[(ii)] \ac{E} immediato verificare che se $v = e_1$ o $v = e_2$, allora 
                $$\frac{\partial f}{\partial e_1} (p_0) = \frac{\partial f}{\partial x} (p_0) \text{ e }
                \frac{\partial f}{\partial e_2} (p_0) = \frac{\partial f}{\partial y} (p_0)$$ 
  \end{enumerate}
\end{osservazione}
\subsection{Teo: Diff. vs. Deriv. direz.}
\begin{theorem}[diffrenenziablit\aca vs derivabiliit\aca direzionale]
  Sia $\f$, A aperto e sia fissato $p_0 = (x_0,y_0) \in A$. Supponiamo che f sia differenziale in $p_0$, allora
  $$\exists \frac{\partial f}{\partial v} (p_0) = df(p_0)(v) = \nabla f(p_0) \cdot (v) = \frac{\partial f}{\partial x} (p_0)(v_1) + \frac{\partial f}{\partial y} (p_0) (v_2)$$
  per ogni direzione $v = (v_1,v_2) \in \R^2$
  \begin{proof}
    Consideriamo la funzione $F: (-\delta,\delta)\to \R$, 
    $$F(t) = f(p_0 +tv) = f(x_0+tv_1,y_0+tv_2)$$
    Per ipotesi, f \ace differenziabile in $p_0$, cio\ace vale:
    $$(D) \exists \lim_{p\to p_0} \frac{f(p)-f(p_0)-\nabla f(p_0)\cdot (p-p_0)}{d(p,p_0)} = 0$$
    la condizione (D) \ace equivalente a chiedere:
    $$(D*) f(p) = f(p_0) + \nabla f (p_0) \cdot (p-p_0) + o\left(d(p,p_0)\right) \, \forall p \in A$$
    dove con $o\left(d(p,p_0)\right) \xLeftrightarrow{def.} \exists \lim_{p\to p_0} \frac{o(d(p,p_0))}{d(p,p_0)}=0$
    Scegliendo $p = p_0+hv$ in (D*), otteniamo che:
    $$F(h) := f(p_0+hv) - f(p_0)+\nabla f(p_0)\cdot (hv) + o(d(p_0+hv,p_0)) = $$
    $$= F(0) + h\left(\nabla f(p_0)\cdot v\right) + o(\abs{h})$$
    Infatti ricordiamo che: 
    $$d(p_0+hv,p_0) =  \norma{p_0+hv-p_0} = \norma{hv} = \abs{h}\norma{v} = \abs{h}$$
    Dall'identit\aca precedente segue che:
    $$\exists F'(0) := \lim_{h\to 0}\frac{F(h) - F(0)}{h} = \nabla f(p_0) \cdot v = df(p_0)(v)$$
    Per l'osservazione precedente $F'(0) = \frac{\partial f}{\partial v} (p_0)$ da cui segue la tesi.
  \end{proof}
\end{theorem}
Dal teorema segue la generalizzazione del teorema del valore medio (G. Lagrange) a funzioni $n=2$ variabili.
\subsection{Teorema del valore medio}
\begin{theorem}[TdVM, n=1]
  Sia $f:[a,b]\to \R$ continua e derivabile in $(a,b)$. Allora $\exists c \in (a,b)$ t.c.
  $$\frac{f(b)-f(a)}{b-a} = f'(c)$$
\end{theorem}
\begin{theorem}[del valore medio, n=2]
  Sia $\f$, A aperto. Supponiamo che:
  \begin{enumerate}
    \item[(i)] $\exists p,q \in A$ t.c. $[p,q] := \{tq+(1-t)p \mid t\in [0,1]\} \subset A$
    \item[(ii)] f \ace continua sull'insieme $[p,q]$ e differenziabile su $(p,q) := \{tq+(1-t)p \mid t \in (0,1)\}$ 
  \end{enumerate}
  Allora esiste un punto $\bar{c} \in (p,q)$ t.c. $f(q)-f(p) = \nabla f(\bar{c})\cdot (q-p)$
  \begin{proof}
    Supponiamo $p \not = q$ altrimenti la tesi \ace banale e sia 
    $$v := \frac{q-p}{\norma{q-p}}$$ una direzione di $\R^2$. \\
    Definiamo la funzione (d n=1 variabile) $F(t) := f(p+tv)$ con $t\in \left[0,\norma{q-p}\right] (\subset \R)$ e fissiamo $p,q$, 
    osserviamo che F \ace ben definita per la (i) e $F(0) = f(p)$ e $F(\norma{q-p}) = f(q)$.
    Inoltre per la (ii):
    \begin{enumerate}
      \item $F:[0,\norma{q-p}] \to \R$ \ace continua;
      \item $\exists F'(t) = \frac{\partial f}{\partial v} (p+tv)$, $\forall t \in (0,\norma{q-p})$
    \end{enumerate}
    Possimao applicare il teorema del valore medio (n=1 variabile) a F e otteniamo che esiste 
    $\bar{t} \in (0,\norma{q-p})$ t.c. 
    $$f(q)-f(p) = F(\norma{q-p}) - F(0) = F(\bar{t})\norma{q-p} = $$
    $$=_{(2)} \frac{\partial f}{\partial v}(p+\bar{t}v)\norma{q-p} = \left(\nabla f(p+\bar{t}v)\cdot v\right)\norma{q-p} = $$
    $$= \left(\nabla f(p+\bar{t}v)\cdot \frac{q-p}{\norma{q-p}}\right)\cdot \frac{q-p}{\norma{q-p}} = \nabla f(p+\bar{t}v)(q-p)$$
    Scegliendo $\bar{c} = p+\bar{t}v \in (p,q)$ otteniamo la tesi  
  \end{proof}
\end{theorem}
\begin{corollary}
  Sia $f: B(p_0,r_0) \subset \R^2 \to \R$. Supponiamo che $\exists \nabla f(p_0) = (0,0) \, \forall p \in B(p_0,r_0)$.
  Allora $f(p) = f(p_0)$, $\forall p \in B(p_0,r_0)$
  \begin{proof}
    per il teorema del diff. tot. f \ace differenziale su $B(p_0,r_0)$. Possimao applicare il teorema del valore 
    medio e otteniamo che $\exists \bar{c} \in (p,p_0)$ t.c. $f(p)-f(p_0) = \nabla f(\bar{c}) (p-p_0) = 0$, $\forall p \in B(p_0,r_0)$
  \end{proof}
\end{corollary}
  \section{Lez - 06}
\subsection{Derivate parziali di una f composta di pi\acu variabili}
\textbf{Problema:} Vogliamo determinare una formula generale che ci consente di calcolare le derivate parziali di una (generica)
funzione composta di pi\acu variabili.
\begin{exercise}[7, foglio 3]
  Consideriamo la funzione composta $h: \R \to \R$ definita come $h := f \circ g$, dove $f : \R^2 \setminus \{(0,0)\} \to \R$,
  $g:\R\to \R^2\setminus \{(0,0)\}$:
  $$f(x,y)=\frac{xy}{x^2+y^2}$$
  se $(x,y)\neq (0,0)$ e\\
  $g(t) = (g_1(t),g_2(t)) = (\sin^2(t),\cos^2(t))$, $t\in \R$
  Calcolare $h'(t)$, $t\in \R$
\end{exercise}
\subsubsection{Richiami della RDC}
Siano $f: I \to \R$ e $g: J \to \R$ con $g(J)\subseteq I$, $I,J$ intervalli aperti di $\R$. \\
$h:= f\circ g$, $h(x):=f(g(x))$, $x\in I$
\begin{proposition}[Regola della catena, RDC]
  Se $f,g$ sono derivabli, rispettivamente, in $g(x_0)$ e in $x_0$, allora $\exists h'(x) = f'(g(x_0))\cdot g'(x_0)$
\end{proposition}
\begin{example}
  $f(y) = \sin{y}$, $g(x) = x^2$, $h = f\circ g$, $h(x) = \sin{x^2}$, 
  $\exists h'(x) = f'(g(x))\cdot g'(x) = \cos{x^2}\cdot 2x$
\end{example}
Prima di arrivare alla formula generale di derivazione di una funzione composta, introduciamo alcuni casi particolari
\subsection{I° caso particolare}
Consideriamo $g : I \subseteq \R \to \R^2$, I intervallo aperto di $\R$, $t_0 \in I$ fissato. 
$$I \ni t \rightarrow g(t) = (g_1(t),g_2(t)) = (x(t),y(t))$$
con $g_1,g_2 : I \to \R$. \\
Supponiamo che $\exists g_1'(t_0), g_2'(t_0)$ e $g(I) \subseteq A \subseteq \R^2$, A aperto.\\
Sia $\f$ e supponiamo che f sia differenziabile in $$p_0 = (x_0,y_0) = g(t_0) = (g_1(t_0),g_2(t_0))$$
Consideriamo la funzione composta $h: I \subseteq \R \to \R$, $h:= f\circ g$
$$I \ni t \rightarrow h(t) := (f\circ g)(t) = f(g(t))=f(g_1(t),g_2(t))$$
\begin{theorem}
  $$(1) \, \exists h'(t_0) = \ppartx \cdot g_1'(t_0) + \pparty \cdot g_2'(t_0)$$
  oppure tramite matrici
  $$(1bis) \, \exists h'(t_0) =  \begin{bmatrix} \ppartx & \pparty \end{bmatrix} \cdot \begin{bmatrix}{c}
    g_1'(t_0)\\
    g_2'(t_0)\\
  \end{bmatrix} = $$
  $= \nabla f(p_0)\cdot g'(t_0)$, dove $g'(t_0) = (g_1'(t_0), g_2(t_0))$.
\end{theorem}
\subsubsection{Espansione calssica di RDC, Leibniz}
Se scriviamo g e f, in termini di "variabili dipendenti", cio\ace 
$$g = \left\{\begin{array}{l}
  x = x(t) = g_1(t) \\
  y = y(t) = g_2(t) \\
\end{array}\right. \text{ (curva del piano)}$$
$z = z(x,y) = f(x,y)$, allora componendo f con g, la variabile dipendente $z$ dipender\aca dalla sola variabile
indipendente $t$ per cui, $z = z(t) = z(x(t),y(t))$, $t \in I$. \\
Quindi in termini di queste variabili $(z,x,y,t)$ si pu\aco scrivere la (1) come:
$$\frac{dz)}{dt} = \frac{\partial z}{\partial x} \cdot \frac{\partial x}{\partial t} + \frac{\partial z}{\partial y} \cdot \frac{\partial y}{\partial t}$$
oppure utilizzando (1bis)
$$\frac{dz}{dt} = \begin{bmatrix}
  \frac{\partial z}{\partial x} & \frac{\partial z}{\partial y}\\
\end{bmatrix} \cdot \begin{bmatrix}
  \frac{\partial x}{\partial t} \\ 
  \\
  \frac{\partial y}{\partial t}\\
\end{bmatrix}$$
\begin{proof}[Idea!]
  Proviamo la (1), cio\ace provare che 
  $$(2) \, \exists h'(t_0) = \lim_{t\to t_0} \frac{h(t)-h(t_0)}{t-t_0} = \nabla f(p_0)\cdot g'(t_0)$$
  Essendo f differenziabile in $p_0$ vale che:
  $$(3) \, f(p) = f(p_0) + df(p_0)\cdot(p-p_0) + o(\norma{p-p_0})$$
  $\forall p \in A$ se $p_0 = g(t_0)$. \\
  Da (3) segue che, se scegliamo $p = g(t)$ otteniamo:
  $$f(g(t)) = f(g(t_0)) + df(g(t_0))\cdot (g(t)-g(t_0)) + o (\norma{g(t)-g(t_0)})$$
  $\forall t \in I$, da cui:
  $$(4)\, \frac{f(g(t)) - f(g(t_0))}{t-t_0} = \frac{df(g(t_0))\cdot (g(t)-g(t_0))}{t-t_0} + \frac{o (\norma{g(t)-g(t_0)})}{t-t_0}$$
  $t \in I, t \neq t_0$. \\
  Osserviamo che essendo $df(p_0): \R^2 \to \R$ lineare allora:
  $$(5)\, \frac{df(p_0)(g(t)-g(t_0))}{t-t_0} = df(p_0)\cdot \left(\frac{g(t)-g(t_0)}{t-t_0}\right)$$
  Passando al limite nella (5) per $t \to t_0$, dalla continuit\aca della funzione $df(p_0)$, si ottiene che:
  $$\lim_{t\to t_0} df(p_0)\left(\frac{g(t)-g(t_0)}{t-t_0}\right) = df(p_0)\cdot g'(t_0)$$
  $$(6) \, \exists \lim_{t \to t_0} \frac{df(p_0)(g(t)-g(t_0))}{t-t_0} = df(p_0)\cdot g'(t_0) = \nabla f(p_0)\cdot g'(t_0)$$
  Si pu\aco provare anche (ed \ace il punto delicato che omettiamo)
  $$(7) \, \exists \lim_{t\to t_0} \frac{o(\norma{g(t)-g(t_0)})}{t-t_0} = 0$$
  Da (6) e (7), possiamo passare al limite per $t \to t_0$ nella (4) ed otteniamo la (2) e dunque la tesi.
\end{proof}
\subsection{II° caso particolare}
$g: A \subseteq \R^2 \to \R^2$, A aperto, e $p_0 = (s_0,t_0)\in A$
$$A \ni (s,t) \rightarrow g(s,t) = (g_1(s,t), g_2(s,t))$$
$g_1, g_2 : A \subseteq \R^2 \to \R$. \\
Supponiamo che $g_1$ e $g_2$ siano differenziabili in $p_0$ e $g(A) \subseteq B \subseteq \R$, B aperto. 
In particolare:
$$\exists \nabla g_i(p_0) = \left(\frac{\partial g_i}{\partial s}(p_0), \frac{\partial g_i}{\partial t}(p_0)\right) \, i = 1,2$$
Sia $f : B \subseteq \R^2 \to \R$, B aperto, f differenziabile in $q_0 = (x_0,y_0)=(g_1(p_0),g_2(p_0))$, 
$B \ni (x,y) \rightarrow f(x,y)\in \R$. \\
Supponiamo che f sia differenziabile in $q_0$. \\
Consideriamo $h := f \circ g : A \subseteq \R^2 \to \R$,
$$A \ni (s,t) \rightarrow (f\circ g)(s,t) = f(g(s,t)) = f(g_1(s,t),g_2(s,t))$$
\begin{theorem}
  $$(1) \, \begin{array}{l}
    \exists \frac{\partial h}{\partial s}(p_0) = \frac{\p f}{\p x}(g(p_0)) \cdot \frac{\p g_1}{\p s}(p_0) + \frac{\p f}{\p y}(g(p_0)) \cdot \frac{\p g_2}{\p s}(p_0) \\
    \\
    \exists \frac{\partial h}{\partial t}(p_0) = \frac{\p f}{\p x}(g(p_0)) \cdot \frac{\p g_1}{\p t}(p_0) + \frac{\p f}{\p y}(g(p_0)) \cdot \frac{\p g_2}{\p t}(p_0) \\
  \end{array}$$
\end{theorem}
in termini di matrici:
$$(1bis) \, \begin{bmatrix}
  \frac{\p h}{\p s}(p_0) & \frac{\p h}{\p t} \\
\end{bmatrix} = \begin{bmatrix}
  \frac{\p f}{\p x}(g(p_0)) & \frac{\p f}{\p y}(g(p_0)) \\
\end{bmatrix} \cdot \begin{bmatrix}
  \frac{\p g_1}{\p s}(p_0) & \frac{\p g_1}{\p t}(p_0)\\
  \\
  \frac{\p g_2}{\p s}(p_0) & \frac{\p g_2}{\p t}(p_0) \\
\end{bmatrix}$$
Dove $\frac{\p g_1}{\p s}(p_0), \frac{\p g_1}{\p t}(p_0) = \nabla g_1(p_0)$ e 
$\frac{\p g_2}{\p s}(p_0) ,\frac{\p g_2}{\p t}(p_0) = \nabla g_2 (p_0)$
\begin{exercise}
  Utilizzare (1bis) del teorma nel secondo caso e svolgere esercizio 7 foglio 3
\end{exercise}
\subsection{Caso generale di RDC}
Vogliamo ora trattare il caso generale della formuala di derivazione per funzioni composte di pi\acu variabili.
\subsubsection{Matrice Jacobiana}
\begin{definition}[Matrice Jacobiana]
  Sia $f : A \subseteq \R^n \to \R^m$, A aperto,
  $$A \ni x = (x_1, ..., x_n)\rightarrow f(x) = \left(f(x_1),..., f(x_n)\right)$$
  con $f_i : A \subseteq \R^n \to \R$, $i = 1,...,n$.\\ Supponiamo che dato $x_0 = (x_1^0, ..., x_n^0) \in A$, 
  $$\exists \nabla f_i(x_0) := \left(\frac{\p f_i}{\p x_1}(x_0), \dots, \frac{\p f_i}{\p x_n}\right)$$
  con $i = 1,..., m$. \\\\
  Si chiama \underline{Matrice Jacobiana} di f nel punto $x_0$ la matrice $m\times n$
  $$D f(x_0) = J f(x_0) = \begin{bmatrix}
    \frac{\p f_1}{\p x_1}(x_0) & \frac{\p f_1}{\p x_2}(x_0) & \cdots & \frac{\p f_1}{\p x_n}(x_0) \\
    \\
    \frac{\p f_2}{\p x_1}(x_0) & \frac{\p f_2}{\p x_2}(x_0) & \cdots & \frac{\p f_2}{\p x_n}(x_0) \\ 
    \vdots &  & & \vdots \\
    \frac{\p f_n}{\p x_1}(x_0) & \frac{\p f_n}{\p x_2}(x_0) & \cdots & \frac{\p f_n}{\p x_n}(x_0) \\
  \end{bmatrix}$$
\end{definition}
\hfill \break
\begin{osservazione}
  \begin{enumerate}
    \item[(i)] La nozione di matrice Jacobiana generalizza la nozione di vettore gradiente per una 
                funzione (scalare) $\fn$. \\
                Si noti che in questo caso la matrice Jacobiana $1 \times n$ \ace data da 
                $$D f_(x_0) := \left(\frac{\p f}{\p x_1}(x_0), \cdots, \frac{\p f}{\p x_n}(x_0)\right) \equiv \nabla f(x_0)$$ 
    \item[(ii)] La riga i-esima della matrice Jacobiana $D f(x_0)$ coincide con $\nabla f_i (x_0)$
    \item[(iii)] La (1bis) del precedente teorema, in termini di matrici Jacobiane pu\aco scriversi come 
                $$D h(p_0) = D f(g(p_0))\cdot D g(p_0) \text{ (RDC)}$$ 
  \end{enumerate}
\end{osservazione}
\subsection{Teorema RDC}
\begin{theorem}[Regola della catena, RDC]
  Siano $g: A \subseteq \R^n\to \R^m$ e $f : B \subseteq \R^m \to \R^k$, A e B aperti
  \begin{enumerate}
    \item[(i)] $g(A) \subseteq B$
    \item[(ii)] Se $g = (g_1, \dots, g_m)$, $f = (f_1, \dots, f_k)$ \\
              Supponiamo che  $\begin{array}{l}
                g_i : A \subseteq \R^n \to \R \, (i = 1,\dots,m) \text{ sia diff. in un dato } x_0 \in A \\
                f_i : B \subseteq \R^m \to \R \, (i = 1,\dots,k) \text{ sia diff. in un dato } y_0 = g(x_0) \\ 
              \end{array}$ \\
              Consideriamo ora la funzione $h:= f \circ g : A \subseteq \R^n \to \R^k$, $h = (h_1, \dots, h_k)$
              con $h_i : A \subseteq \R^n \to \R$, \\ allora le funzioni 
              $h_i : A \to \R (i = 1,\dots,k) \text{ sono diff. in } x_0$ e 
              $$D h(x_0) = D f(g(x_0)) \cdot D g(x_0)$$
  \end{enumerate}
\end{theorem}
  \section{Lez - 07}
\subsection{Derivate parziali di ordine superiore}
\subsection{Teo: Inversione dell'ordine di derivazione}
\subsection{Taylor per funzioni di pi\acu variabili}
\subsection{Taylor del II ordine + resto di Peano}
  \section{Lez - 08}
\subsection{Massimi e minimi per funzioni a pi\acu variabili}
\textbf{Problema:} Dato un insieme $A \subseteq \R^n$ e data $f:A\to\R$, determinare,
\underline{se esistono}, i punti di max e min di f.
\begin{definition}
  Data $\fn$: 
  \begin{enumerate}
    \item $p_0 \in A$ si dice, punto di \underline{massimo} (= max) \underline{relativo} di f su A se 
          $\exists r_0 > 0$ t.c. $f(p) \leq f(p_0) \, \forall p \in A \cap B(p_0,r_0)$ \\
          Rispettivamente $p_0 \in A$ si dice, punto di \underline{minimo} (= min) \underline{relativo} di f su A se 
          $\exists r_0 > 0$ t.c. $f(p) \geq f(p_0) \, \forall p \in A \cap B(p_0,r_0)$
    \item $p_0 \in A$ si dice punto di \underline{massimo} (= MAX) \underline{assoluto} se 
          $\forall p \in A$, $f(p) \leq f(p_0)$ \\
          Rispettivamente $p_0 \in A$ si dice punto di \underline{minimo} (= MIN) \underline{assoluto} se 
          $\forall p \in A$, $f(p) \geq f(p_0)$ 
  \end{enumerate}
\end{definition}
\begin{osservazione}
  Se $p_0$ \ace un punto di max ( o min) assoluto $\Rightarrow p_0$ \ace punto di max (o min) relativo. Il 
  viceversa non pu\aco valere.
\end{osservazione}
\begin{NB}
  Non confondere i punti di max e min di una funzione con il suo massimo e minimo.
  \begin{itemize}
    \item $Min_{A} f := Min{f(p) : p \in A} \in \R$, se esiste \ace unico
    \item $Max_{A} f := Max{f(p) : p \in A} \in \R$, se esiste \ace unico
  \end{itemize}
\end{NB}
Consideriamo il seguente esempio:
\begin{example}
  $n=1$, $f : \R \to \R \, f(x) =:= x(3-x^2)$ \\\\
  In particolare si pu\aco vedere che i punti $x = \pm 1$ sono rispettivamente max e min relativi, ma $x=-1$ non \ace 
  minimo assoluto e $x=+1$ non \ace massimo assoluto. \\
  Infatti essendo la funzione non limitata ($\sup_{\R} f = +\infty$ e $\inf_{\R} f = -\infty$) $\not \exists max_{\R} f$ e $min_{\R} f$ 
\end{example}
\subsection{Estremi liberi di una funzione (min/max relativi)}
\textbf{Problema:} Supponiamo che $A \subseteq \R^n$ sia aperto e $f: A \to \R$, vogliamo determinare se esistono i punti di max e min
relativo su A. Questi punti sono detti \underline{estremi liberi} di f. \\\\
Lo strumento principale per la ricerda di estremi liberi \ace: 
\begin{theorem}[Fermat]
  Sia $\fn$, A aperto. Supponiamo che esista $p_0 \in A$ t.c. 
  \begin{enumerate}
    \item[(i)] f differenziabile in $p_0$. In particolare $\exists \nabla f(p_0)$
    \item[(ii)] $p_0$ sia un estremo libero di f in A
  \end{enumerate}
  Allora $\nabla f(p_0) = \underline{O}_{\R^n} = (0,...,0) \text{ (n-volte)}$
\end{theorem}
Il precedente teorema giustifica la seguente definizione:
\begin{definition}
  Data $\fn$, A aperto, un punto $p_0\in A$ si chiama \underline{punto stazionario}(o \underline{critico}) di f se f \ace differenziabile
  in $p_0$ e $\nabla f(p_0) = \underline{O}_{\R^n}$
  \begin{proof}
    Per semplicit\aca, $n = 2$, $p_0 = (x_0,y_0)$. Essendo A aperto esiste $\delta > 0$ t.c. $p_0+te_1 \in A$ se $t \in (-\delta,\delta)$. \\
    Consideriamo $F:(-\delta,\delta)\to\R$, $F(t) := f(p_0+te_1)$, da (i) 
    $$\exists\frac{\p f}{\p x}(p_0) \iff (1) \left\{\begin{array}{l}
      \text{F \ace derivabile nel punto t = 0} \\
      \text{ e } F'(0) = \frac{\p f}{\p x}(p_0) \\
    \end{array}\right.$$
    Dall'altra parte, da (ii) $p_0$ estremo libero di f (2) $t=0$ \ace un estremo libero di F. \\
    Possiamo applicare il teorema di Fermat di una variabile ed otteniamo $F'(0) = 0 =_{(1)} \frac{\p f}{\p x}(p_0)$. \\\\
    Analogamente consideriamo $F(t) = f(p_0+te_2)$ e si prova che $\frac{\p f}{\p y}(p_0)=0$. \\
    Pertanto si prova che $\nabla f(p_0) = (0,0) = \underline{O}_{\R}^2$
  \end{proof}
\end{definition}
\begin{osservazione}
  Non ogni punto stazionario di f \ace un punto di estremo libero.
\end{osservazione}
\begin{example}
  $\f$, $f(x,y) = y^3$, $p_0 = (x_0,0)$. \\
  Poich\ace $\nabla f(x,y) = (0,3y^2)$, $\nabla f(p_0) = (0,0)$. Pertanto ogni punto $p_0 = (x_0,0)$ (per un fissato 
  $x_0\in \R$) \ace un punto stazionario di f, ma $p_0$ non \ace un estremo libero, infatti $\forall r > 0$ $f(x_0,0) = 0 \, 
  \forall x_0 \in \R$, quindi $p_0=(x_0,0)$ si dice punto di \underline{sella}.
\end{example}
\begin{definition}
  Sia $\fn$, A aperto. Un punto $p_0 \in A$ si dice \underline{punto di sella} se $p_0$ \ace un punto 
  stazionario di f e $f(p)-f(p_0)$ amette sia valori positivi che negativi in ogni intorno di $p_0$
\end{definition}
\subsection{Matrice Hessiana}
\textbf{Problema:} Sia $A \subseteq \R^n$, $f \in C^2(A)$. Supponiamo che $p_0 \in A$ sia un punto stazionario di f. \\
Come determinare se $p_0$ sua un estremo libero o un  punto di sella?
\begin{definition} 
  Sia $f \in C^2(A)$, si chiama, \underline{matrice hessiana} di f nel punto $p_0\in A$ la matrice simmetrica ($n\times n$)
  $$D^2f(p_0) = H f(p_0) = \begin{bmatrix}
    \frac{\p^2 f}{\p x_1^2}(p_0) & \dots & \frac{\p^2 f}{\p x_n \p x_1}(p_0) \\
    \vdots & & \vdots \\
    \frac{\p^2 f}{\p x_1 \p x_n}(p_0) & \dots & \frac{\p^2 f}{\p x_n^2}(p_0) \\
  \end{bmatrix} = \begin{bmatrix}
    \nabla \left(\frac{\p^2 f}{\p x_1}\right)(p_0) \\
    \vdots \\
    \nabla \left(\frac{\p^2 f}{\p x_n}\right)(p_0)
  \end{bmatrix}$$
\end{definition}
\subsection{Teorema: Criterio per il segno di una matrice}
\subsubsection{Richiami di algfebra lineare}
\begin{definition}
  Sia H una matrice $n\times n$
  \begin{enumerate}
    \item[(i)] H si dice \underline{definita positiva} se $\tuple{Hv,v} > 0$, $\forall v \in \R^n \times \{\origine\}$
    \item[(ii)] H si dice \underline{semi-definita positiva} se $\tuple{Hv,v} \geq 0$, $\forall v \in \R^n \times \{\origine\}$
    \item[(iii)] H si dice \underline{definita negativa} se $\tuple{Hv,v} < 0$, $\forall v \in \R^n \times \{\origine\}$
    \item[(iv)] H si dice \underline{semi-definita negativa} se $\tuple{Hv,v}  \leq 0$, $\forall v \in \R^n \times \{\origine\}$
  \end{enumerate}
\end{definition}
Un criterio semplice per verificare il segno di una matrice H $n\times n$:
\begin{theorem}[criterio per il segno di una matrice]
  Sia $$H = \begin{bmatrix}
    h_{11} & \dots & h_{1n} \\
    \vdots & & \vdots \\
    h_{n1} & \dots & h_{nn} \\
  \end{bmatrix} \text{ una matrice } n \times n$$
  Definiamo $$D_i = det\begin{matrix}
    h_{11} & \dots & h_{1i} \\
    \vdots & & \vdots \\
    h_{i1} & \dots & h_{ii} \\
  \end{matrix} \text{ con } 1 \leq i \leq n$$
  Allora 
  \begin{enumerate}
    \item[(a)] H \ace definita positiva $\iff D_i > 0 \, \forall i = 1, ..., n$
    \item[(b)] H \ace definita negativa $\iff \left\{\begin{array}{l}
      D_i > 0 \text{ per i valori pari di i} \\
      D_i < 0 \text{ per i valori dispari di i} \\
    \end{array}\right.$
    \item[(c)] Se $detH = Dn \neq 0$ e nessuna delle condizioni precedenti fosse soddisfatta, allora H non \ace semi-definita 
                positiva n\ace semi-definita negativa
  \end{enumerate}
\end{theorem}
\begin{corollary}
  Se H ($2\times 2$) matrice simmetrica $H = \begin{bmatrix}
    h_{11} & h_{12} \\
    h_{21} & h_{22} \\
  \end{bmatrix}$ con $h_{12} = h_{21}$
  \begin{enumerate}
    \item[(a)] H \ace definita positiva $\iff h_{11} > 0$ e $detH > 0$
    \item[(b)] H \ace definita negativa $\iff h_{11} < 0$ e $detH > 0$
    \item[(c)] Se $detH < 0$, allora H non \ace semi-def. pos. n\ace semi.def. neg.
  \end{enumerate}
\end{corollary}
\begin{theorem}[BDPG,11.25]
  Sia A aperto di $\R^n$, $f\in C^2(A)$ e sia $p_0\in A$ un punto stazionario di f
  \begin{enumerate}
    \item[(i)] Se $D^2f(p_0)$ fosse def. pos. $\Rightarrow p_0$ \ace un punto di \underline{minimo relativo} di f su A
    \item[(ii)] Se $D^2f(p_0)$ fosse def. neg. $\Rightarrow p_0$ \ace un punto di \underline{massimo relativo} di f su A
    \item[(iii)] Se $D^2f(p_0)$ non fosse semi-def. pos. n\ace semi-def. neg. $\Rightarrow p_0$ \ace un punto di \underline{sella} di f su A
    \item[(iv)] Se $D^2f(p_0)$ fosse semi-def. pos. o semi-def. neg. $\Rightarrow p_0$ pu\aco essere un punto di \underline{massimo o minimo relativo} o 
                un punto di \underline{sella} di f su A
  \end{enumerate}
\end{theorem}
\subsection{Esempi}
\begin{example}[1a,foglio 5]
  Data $\f$, $f(x,y) = x^2++2kxy+y^2$. \\
  Determinare al variare di $k \in \R$, i punti di max e min relativo di f.
  \begin{itemize}
    \item[Soluzione 1.] \textbf{Punti stazionari di f su $\R^2$}
          $$\nabla f(x,y) = (0,0) \iff \left\{\begin{array}{l}
            \frac{\p f}{\p x}(x,y) = 2x+2ky = 0 \\
            \\
            \frac{\p f}{\p y}(x,y) = 2kx+2y = 0 \\
          \end{array}\right.$$
          \begin{itemize}
            \item \begin{exercise}
                    Se $k\neq 1 \Rightarrow (0,0)$ \ace l'unico punto stazionario 
                  \end{exercise}
            \item Se $k = 1 \Rightarrow$ I punti della retta $x+y=0$ sono tutti e soli i punti stazionari
            \item Se $k = -1 \Rightarrow$ I punti della retta $x-y=0$ sono tutti e soli i punti stazionari 
          \end{itemize}
    \item[Soluzione 2.] \textbf{Studio del segno della matrice Hessiana}
          $$D^2f(x,y) = \begin{bmatrix}
            \frac{\p^2 f}{\p x^2} & \frac{\p^2 f}{\p y \p x}  \\
            \\
            \frac{\p^2 f}{\p x \p y} & \frac{\p^2 f}{\p y^2}  \\
          \end{bmatrix}(x,y) = \begin{bmatrix}
            2 & 2k \\
            2k & 2 \\
          \end{bmatrix}$$
          $\forall (x,y)\in \R^2$ e $detD^2f(x,y) = 4-4k^2 = 4(1-k^2)$
          \begin{itemize}
            \item \begin{exercise}
              Se $k^2 < 1 \Rightarrow D^2 f(0,0)$ \ace def. positiva $\Rightarrow$ (0,0) \ace un punto di minimo relativo
            \end{exercise}
            \item Se $k=1$, $detD^2f(x_0,-x_0) = 0 \Rightarrow D^2f(x_0,-x_0)$ non \ace def-neg. n\ace def-pos. $\Rightarrow$ nulla si pu\aco dire
              \begin{itemize}
                \item Se $k=1$, $f(x,y) = x^2+2xy+y^2 = (x+y)^2 \Rightarrow (x_0,-x_0)$ \ace un punto di minimo assoluto
                \item Se $k=-1$, $f(x,y) = x^2-2xy+y^2 = (x-y)^2 \Rightarrow (x_0,-x_0)$ \ace un punto di minimo assoluto
              \end{itemize}
          \end{itemize}
  \end{itemize}
\end{example}
\textbf{Appendice:}
\begin{enumerate}
  \item Se f \ace differenziabile in $p_0$ e $\nabla f(p_0) = (0,0) \Rightarrow \exists \frac{\p f}{\p v}(p_0)=0 \, 
        \forall v \in \R^2$, $\norma{v}=1$. \\
        Infatti poich\ace $\frac{\p f}{\p v}(p_0) = \nabla f (p_0 = \origine_{\R^2}) \cdot v = 0$
  \item $f : \R \to \R$, $f(x) = \abs{x}$, $x_0 = 0$, il punto di $x_0=0$ \ace un punto di minimo assoluto per f.
  \item Se $k=1$ i punti della retta di eq: $x+y=0$ sono tutti e soli i punti stazionari di f.
  \item $k=1$, $f(x,y) = (x+y)^2 \geq 0 \, \forall (x,y)\in \R^2$, $f(x_0,-x_0)=0$
\end{enumerate}
  \section{Lez - 09}
\subsection{Ricerca del max e min (assoluto) su insieme limitato e chiuso}
\subsection{Frontiera attraverso parametrizzazione}
\subsection{Metodo dei moltiplicatori di Lagrange, TML}
\subsection{Teorema della funzione implicita, U. Dini}

  \chapter{Integrale per funzioni a pi\acu variabili, [BDPG, 14]}
Vogilamo introdurre la nozione di \underline{integrale } per una funzione 
$\fn (n=2,3)$, detto anche \underline{integrale multiplo} 
\section{Lez - 10, Integrale doppio su un rettangolo}
\subsubsection{Caso $n=2$}
$A = Q = [a,b]\times [c,d]$ e sia $f:A\to \R$ limitata, 
cio\ace $\exists M > 0$ t.c. $\abs{f(p)} \leq M \, \forall p \in A$\\\\
\textbf{Idea:(Interpretazione geometrica dell'integrale)} \\\\
Supponiamo $f(p)\geq 0 \, \forall p \in A$, definiamo 
$$\T_f(A) =\{(x,y,z)\in\R^3 : 0 \leq f(x,y), (x,y)\in A\}$$
(trapezioidale indotta da $f : A \to \R$). \\
$\T_f(A) \equiv$ solido di $\R^3$ sotteso dal grafico di f, $G_f$.\\
Vogliamo definire un numero reale non negativo:
$$L = \iint_{A} f(x,y)\,dx\,dy \text{ (integrale doppio di f su A)}$$
t.c. $L = volume(\T_f(A))$ 
\begin{definition}
  \begin{itemize}
    \item[(i)] Si chiama suddivisione dell'intervallo $[a,b]$ un insieme \underline{finito} (detto retta reale)
              $\{x_0,x_1,..., x_{n-1}, x_n\}$ t.c. $a = x_0 < x_1 < ... < x_{n-1} < x_n = b$
    \item[(ii)] Si chiama sudddivisione dell'insieme $\Qbase$ l'insieme (del piano)
      $\mathcal{D} := \D_1 \times \D_2 = \{(x_i,y_j): i = 0,..., n , j = 0,..., m\}$, dove:
      $$\D_1 = \{x_0,...,x_n\} \text{ suddivisione di [a,b]}$$ 
      $$\D_2 = \{y_0,...,y_n\} \text{ suddivisione di [c,d]}$$
      Dato $\D$ una suddivisione di Q, Q resta suddiviso in $n\times m$ rettangoli
      $$Q_{ij} := [x_{i-1}, x_i]\times[y_{j-1},y_j]$$
      con $i = 1,..., n$ e $j = 1,..., m$\\
      Definiamo $area(Q_{ij}):=(x_i-x_{i-1})(y_j-y_{j-1})$
  \end{itemize}
\end{definition}
\begin{definition}
  Si chiama \underline{somme superiore} (risp. \underline{somme inferiore}) di f rispetto alla suddivisione 
  $\D$ di Q, fissata una funzione $f:Q \to \R$, il numero reale
  $$S(f,\D):= \sum_{j=1}^{m}\sum_{i=1}^{n} M_{ij} area(Q_{ij}) \, M_{ij} := sup_{Q_{ij}}f$$
  rispettivamente il numero reale
  $$s(f,\D):= \sum_{j=1}^{m}\sum_{i=1}^{n} m_{ij} area(Q_{ij}) \, m_{ij} := inf_{Q_{ij}}f$$
\end{definition}
\begin{osservazione}
  Essendo f limitata, $i = 1,..., n \, j = 1,..., m$
  $$M_{ij} :=  sup_{Q_{ij}}f := sup\{f(p):p \in Q_{ij}\}$$
  $$m_{ij} :=  inf_{Q_{ij}}f := inf\{f(p):p \in Q_{ij}\}$$
\end{osservazione}
\subsection{Propriet\aca importanti delle somme sup. ed inf. }
\begin{itemize}
  \item[(PS1)] Se $f\geq 0$ su Q, allora $M_{ij}area(Q_{ij})$ e $m_{ij}area(Q_{ij})$ rappresentano
          il volume di un parallelepipedo di base $Q_{ij}$ ed altezza $M_{ij}$ o $m_{ij}$
  \item[(PS2)] Per ogni suddivisione $\D$ di Q
            $$area(Q)\cdot inf_{Q} f \leq s(f,\D) \leq S(f,\D) \leq area(Q) \cdot \sup_{Q} f$$
  \item[(PS3)] Si potrebbe provare (ma lo omettiamo) che, prese $\D'$ e $\D''$ due suddivisioni di Q, allora 
            $s(f,\D') \leq S(f,\D'')$
\end{itemize}
\begin{definition}
  Siano $\Qbase$ e $f : Q \to \R$ limitata. La funzione si dice integrabile (secondo \underline{Reimann}) su Q, e scriveremo
  $f\in \mathcal{R}(Q)$, se $$L = sup{s(f,\D): \D \text{ sudd. di Q}} = inf\{S(f,\D), \D \text{ sudd. di Q}\}$$
  Il numero reale L si chiama \underline{integrale}(doppio) e si denota
  $$L = \iint_{Q} f(x,y) \, dx \, dy = \iint_{Q} f = \int_{Q} f$$
  Nel caso in cui $f\geq 0$ ed $f \in \Rcal(Q)$, definiamo il \underline{volume del solido $\T_{f}(Q)$}
  $$vol\left(\T_f(Q)\right):= \iint_{Q} f$$ 
\end{definition}

\subsection{Teoremi: Esistenza \& Propriet\aca integrale}
\textbf{Problema:} Condizioni che assicurano $f\in \Rcal(Q)$? \\\\
Richiami per funzioni di $n=1$ variabili $Q=[a,b]$
\begin{theorem}
  $f \in C^0\left([a,b]\right)$, allora $f\in \Rcal\left([a,b]\right)$, \\
  $f \geq 0, f \in C^0\left([a,b]\right)$, $area(\T_f([a,b])) := \int_{a}^{b} f(x)\, dx$
\end{theorem}
\begin{theorem}
  Se $f:[a,b]\to \R$ \ace non decrescete (cio\ace $x<y \Rightarrow f(x)\leq f(y)$), allora $f \in \Rcal\left([a,b]\right)$
\end{theorem}
\begin{theorem}[Esistenza dell'integrale][BDPG,14.4]
  Sia $f \in C^0\left(Q\right)$ allora $f \in \Rcal(Q)$
\end{theorem}
\begin{theorem}[Propriet\aca dell'integrale][BDPG, 14.5]
  Siano $f,g \in \Rcal (Q)$ con $\Qbase$
  \begin{itemize}
    \item[(i)] \textbf{Linearit\aca}: $\alpha f + \beta g \in \Rcal(Q)$, $\forall \alpha,\beta\in\R$ e 
              $$\iint_{Q}\left(\alpha f + \beta g\right) = \alpha \iint_{Q}f + \beta \iint_{Q} g$$
    \item[(ii)] \textbf{Monotonia}: Se $g\leq f$ su Q, allora $$\iint_{Q} g \leq \iint_{Q}f$$
    \item[(iii)] \textbf{Valore assoluto}: $\abs{f} \in \Rcal(Q)$ e $$\abs{\iint_{Q}f} \leq \iint_{Q}\abs{f}$$
    \item[(iv)] \textbf{Teorema della media interale} 
                $$inf_{Q} f \leq \frac{1}{area(Q)}\iint_{Q} f \leq sup_{Q}f$$
                e il valore $\frac{1}{area(Q)}\iint_{Q} f \equiv$ media integrale di f su Q. \\
                Se $f \in C^0(Q)$, allora esiste $p_0 = (x_0,y_0)\in Q$ t.c. 
                $$f(p_0) = \frac{1}{area(Q)}\iint_{Q} f$$
  \end{itemize}
\end{theorem}
\subsection{Formula di riduzione sui rettangoli}
\textbf{Problema}: Sia $f \in \Rcal(Q)$, come calcolare $\iint_{Q}f$?
\begin{theorem}[Formula di riduzione sui rettangoli][BDPG, 14.6]
  \label{ridRettangoli}
  Siano $\Qbase$ e $f \in \Rcal(Q)$
  \begin{itemize}
    \item[(i)] Supponiamo che, $\forall y \in [c,d]$, la funzione 
                $[a,b]\ni x \to f(x,y)$ sia integrabile (come funzione di una variabile), 
                allora la funzione 
                $[c,d]\ni y \to \int_{a}^{b} f(x,y) \, dx$ \ace integrabile su [c,d] e 
                $$(1) \iint_{Q} f = \int_{c}^{d} \left(\int_{a}^{b} f(x,y) \, dx\right) \, dy$$
    \item[(ii)] Supponiamo che, $\forall x \in [a,b]$, la funzione 
                $[c,d]\ni y \to f(x,y)$ sia integrabile (come funzione di una variabile), 
                allora 
                $$(2) \iint_{Q} f = \int_{a}^{b} \left(\int_{c}^{d} f(x,y) \, dy\right) \, dx$$
                In particolare se $f\in C^0(Q)$, valgono (i) e (ii) e 
                $$(3) \iint_{Q} f = \int_{a}^{b} \left(\int_{c}^{d} f(x,y) \, dy\right) \, dx = \int_{c}^{d} \left(\int_{a}^{b} f(x,y) \, dx\right) \, dy$$
  \end{itemize}
\end{theorem}
\begin{osservazione}[Principio di Cavalieri]
  La formula (2) pu\aco essere interpretata geometricamente nel modo seguente: \\
  sia $f \geq 0$, definimao la regione piana $\T_{f}^{x}(Q):=\{(x,y,z)\in \R^3 : 0 \leq z \leq f(x,y)\}$ per 
  $x\in[a,b]$ fissato. Allora $$A(x) = \int_{c}^{d} f(x,y) \, dy = area\left(\T_{f}^{x}(Q)\right)$$
  Pertanto la (2) si pu\aco interpretare come 
  $$volume(\T_{f}(Q)):= \iint_{Q}f =_{(2)} \int_{a}^{b} \left(\int_{c}^{d} f(x,y) \, dy\right) \, dx = $$
  $$= \int_{a}^{b} area(\T_{f}^{x}(Q)) \, dx \textit{ somma di volumi infinitesimi}$$
\end{osservazione}
\subsection{Esempio}
\begin{example}
  Calcolare $\iint_{Q} f$ dove $Q = [0,1]\times[0,\pi]$, $f(x,y):=x\cdot \sin(xy)$ \\\\
  \textbf{Soluzione:} \\
  \ac{E} facile verificare che $f \in C^0(Q)$, quindi possiamo utilizzare la formula (3), per\aco
  osserviamo che, ai fini del caolcolo, utilizzare (1) o (2) pu\aco essere differente.
  $$\iint_{Q}f = \int_{0}^{1}\left(\int_{0}^{\pi} x\cdot \sin(xy) \, dy\right) \, dx$$
  Fissiamo $0 \leq x \leq 1$
  $$\int_{0}^{\pi} x\cdot \sin(xy) = x \int_{0}^{\pi} \sin(xy) \, dy = x \left. \left(-\frac{\cos(xy)}{x}\right) \right|_{0}^{\pi}
  = -cos(\pi x)+1$$
  Quindi 
  $$\int_{0}^{1}\left(\int_{0}^{\pi}  x\cdot \sin(xy) \, dy\right)\, dx = 
  \int_{0}^{1} \left(-cos(\pi x)+1\right) \, dx =$$
  $$= \left. - \frac{\sin(\pi x)}{\pi} + x \right|_{0}^{1} = -\frac{\sin\pi}{\pi} + 1 + \frac{\sin0}{\pi} - 0 = 1$$
  \begin{exercise}
    Verificare che l'integrlae iterato 
    $$1 = \iint_{Q} f = \int_{0}^{\pi} \left(\int_{0}^{1} x\cdot \sin(xy) \, dx\right) \, dy = 
    \int_{0}^{\pi} \frac{\sin(y) - y\cos(y)}{y^2} \, dy$$
    L'ultimo integrale, esiste, ma la funzione integranda non ammettec ome primitivva rappresentabile come funzioni 
    elementari, come, per esempio $\int \frac{\sin{x}}{x}\,dx$, $\int e^{-x^2} \,dx$, 
    $[0,1]\ni x \to \frac{\sin(x)}{x}$ \ace continua, dunque $\exists \int_{0}^{1} \frac{\sin(x)}{x} \, dx \in \R$
  \end{exercise}
\end{example}
  \section{Lez 11 - Integrale doppio su insiemi generali}
\subsection{Insiemi numerabili e loro area}
\subsubsection{Teorema: Caraterizzazione degli insiemi misurabili}
\subsection{Integrali doppi su insiemi misurabili}
\subsubsection{Teorema: Esistenza integrale doppio su insiemi misurabili}
\subsection{Teo.: Integrale doppio su insieme di misura nulla}
\section{Integrali doppi su domini semplici e formule di riduzione}
\subsection{Teorema: Forumla di riduzione su domini semplici}
  \section{Lez - 12}
\subsection{Applicazione della formula di riduzione su domini semplici al calcolo di volumi di solidi}
\begin{definition}
  Sia $A \subseteq \R^2$ limitato e misurabile e $f\in\Rcal(A)$, con $f\geq 0$ su A. Denotiamo 
  $$\T_{f}(A) := \{(x,y,z)\in\R^3: 0 \leq z \leq f(x,y), (x,y)\in A\}$$
  Si chiama volume del solido $\T_{f}(A)$ il numero 
  $$volume\left(\T_{f}(A)\right):= \iint_{A} f$$
\end{definition}
Tramite la formula (1) e (2) del precedente teorema \ref{14.17} si possono calcolare i volumi di diversi solidi.
\begin{example}
  Sappiamo che $A \subseteq \R^2$ sia un dominio semplice rispetto a y, allora dalla (1) si ottiene:
  $$(*) \text{ } volume(\T_{f}(A))=\iint_{A}f = \int_{a}^{b}\left(\int_{g_1(x)}^{g_2(x)}f(x,y)\,dy \right) \, dx$$
\end{example}
\begin{exercise}
  Calcolare il volume del solido di $\R^3$, $$S = \{(x,y,z)\in \R^3: 0 \leq z \leq y^2, (x,y)\in [0,1]\times[0,1]\}$$
  \textbf{Soluzione:}\\\\
  \ac{E} facile verificare che $S = \T_{f}(A)$ con $A = [0,1]\times[0,1]$ dominio semplice sia rispetto y che x, ed 
  $f(x,y):= y^2$, $f \in C^0(A)$. Pertanto possiamo applicare (*) e otteniamo
  $$volume(S) = volume(\T_{f}(A)) = \iint_{A}f = \int_{0}^{1}\left(\int_{0}^{1}y^2 \,dy\right)\, dx$$
  se rappresentiamo 
  $$A = \{(x,y)\in \R^2 : 0 \leq x \leq 1, 0 \leq y \leq 1\}$$
  Fissato $x \in [0,1]$, 
  $$\int_{0}^{1} y^2 \, dy = \left.\frac{1}{3}y^3 \right|_{0}^{1} = \frac{1}{3}$$
  Pertanto $$volume(S) = \int_{0}^{1} \frac{1}{3}\,dx = \frac{1}{3}$$
\end{exercise}
Infine vale la seguente propriet\aca, molto utili nel calcolo di integrali doppi
\subsection{Teorema: Additivit\aca dell'integrale doppio}
\begin{theorem}[Additivit\aca dell'integrale doppio][BDPG,14.18] \\
  Siano $A_1,...,A_m \subseteq \R^2$ insiemi semplici t.c.
  $$A_i \cap A_j \subseteq \p A_i \cap \p A_j$$ se $i\neq j, i, j = 1,...,m$. \\
  Sia $f:A_1\cup...\cup A_m \to \R$ e supponiamo che $f \in \Rcal(A_i) \forall i = 1,..., m$. \\
  Allora f \ace integrabile su $A_1\cup...\cup A_m$, cio\ace $f \in \Rcal(A_1\cup...\cup A_m)$ e 
  $$\iint_{A_1\cup...\cup A_m} f = \sum_{i=1}^{m} \iint_{A_i}f$$ 
\end{theorem}
\section{Cambiamento di var. per gli integrali doppi}
\subsection{Caso particolare: coordinate polari}
\textbf{Problema}: Calcolare il volume della semisfera di centro $(0,0,0)$ e raggio $r>0$ in $\R^3$. \\
\ac{E} facile verificare che, se denotiamo S la semisfera di centro $(0,0,0)$ e raggio $r>0$, 
$$S:=\{(x,y,z)\in\R^3 : x^2+y^2+z^2 \leq r^2, z\geq 0\}$$
$z^2 \leq r^2-(x^2+y^2)$, $0 \leq z \leq \sqrt{r^2-(x^2+y^2)}$ \\\\
Inotre, se denotiamo $D:=\{(x,y)\in\R^2:x^2+y^2 \leq r^2\}$ allora S pu\aco essere anche rappresentato
nella forma 
$$S = \left\{(x,y,z)\in\R^3:(x,y)\in D, 0 \leq z \leq \sqrt{r^2-(x^2+y^2)}\right\} = \T_{f}(D)$$
dove $f(x,y) := \sqrt{r^2-(x^2+y^2)}$,$(x,y)\in D$. \\
Utilizzando la nostra definizione di volume $\T_{f}(D)$, otteniamo che 
$$volume(S) = volume(\T_{f}(D)) := \iint_{D}f(x,y)\,dx\,dy$$
\begin{exercise}
  Calcolare $\iint_{D}\sqrt{r^2-(x^2+y^2)}\,dx\,dy$
\end{exercise}
\textbf{Soluzione:}
\begin{itemize}
  \item \textbf{Primo modo} \\
        Osserviamo che l'insieme D pu\aco essere rappresentato come un dominio semplcei rispetto all'asse y.
        Infatti $$D = \{(x,y)\in\R^2 : -r \leq x \leq r, -\sqrt{r^2-x^2} \leq y \leq \sqrt{r^2-x^2}\}$$
        Utilizzando la formula di riduzione per integrali doppi su domini semplici, otteniamo
        $$\iint_{D} \sqrt{r^2-(x^2+y^2)}\,dx\,dy = \int_{-r}^{r}\left(\int_{-\sqrt{r^2-x^2}}^{\sqrt{r^2-x^2}} 
        \sqrt{r^2-(x^2+y^2)} \, dy \right)\,dx$$
        Notiamo che il calcolo dell'integrare iterato risulta abbastanza complicato.
  \item \textbf{Secondo modo} \\
        Utilizziamo le coordinate polari, cio\ace consideriamo l'applicazione 
        $\psi : (0,+\infty)\times(0,2\pi) \to \R^2$, $\rho, \vartheta \to (\rho\cos\vartheta, \rho\sin\vartheta)$
        $$\left\{\begin{array}{l}
          x = x(\rho,\vartheta) = \rho\cos\vartheta \\
          y = y(\rho,\vartheta) = \rho\sin\vartheta \\
        \end{array}\right.$$
        \ac{E} facile verificare che $\psi:(0,+\infty)\times(0,2\pi) \to \R^2 \setminus\{(x,0): x \geq 0\}$
        \ace bigettiva e se $D^{*}:= (0,r)\times(0,2\pi)$
        $$\psi(D^*) = \overcirc{D}\setminus\{(x,0):  0\leq x\leq r\}$$
        Poich\ace $$area(D) = area\left(\overcirc{D}\setminus\left\{(x,0):  0\leq x\leq r\right\}\right)$$ e 
        $$area(\p D) = area\left(\{(x,0):  0\leq x\leq r\}\right) = 0$$
        per la propriet\aca degli integrali doppi sugli insiemi di misura nulla, segue che
        $$\iint_{D} \sqrt{r^2-(x^2+y^2)}\,dx\,dy = \iint_{\overcirc{D}\setminus\{(x,0): 0 \leq x \leq r\}}\sqrt{r^2-(x^2+y^2)}\,dx\,dy$$
        $$= \iint_{\overcirc{D}\setminus\{(x,0): 0 \leq x \leq r\}}\sqrt{r^2-\rho^2}\,dx\,dy$$  
        \textbf{Idea:} Vogliamo cambiare le variabili di integrazione nell'integrale doppio 
        da $(x,y) \rightarrow \text{ a } (\rho,\vartheta)$. \\
        Il problema \ace capire come si trasforma l'elemento infinitesimo di area $dA = dxdy$ in funzione
        dell'elemento infinitesimo $dA^* = d\rho d\vartheta$\\\\
        Pi\acu precisamente capire quale sia il coefficiente di trasformazione $k = k(\rho,\vartheta)$ per cui 
        $dA = dxdy = k (\rho, \vartheta)\, d\rho \, d\vartheta = k(\rho,\vartheta)dA^{*}$\\
        Utilizziamo un ragionamento intuitivo: il rettangolo 
        $Q^* = [\rho, \rho + d\rho]\times[\vartheta, \vartheta+d\vartheta]$, sar\aca trasportato nella
        regione piana $Q = \psi(Q^*)$ delimitata da:
        \begin{itemize}
          \item $L_1$ = il segmento che congiunge i punti $\psi(\rho, \vartheta)$ e $\psi(\rho + d\rho, \vartheta)$
          \item $L_2$ = l'arco di cerchio che congiunge i punti $\psi(\rho+d\rho, \vartheta)$ e $\psi(\rho + d\rho, \vartheta+d\vartheta)$
          \item $L_3$ = il segmento che congiunge i punti $\psi(\rho+d\rho, \vartheta+d\vartheta)$ e $\psi(\rho, \vartheta+d\vartheta)$
          \item $L_4$ = l'arco di cerchio che congiunge i punti $\psi(\rho, \vartheta+d\vartheta)$ e $\psi(\rho, \vartheta)$
        \end{itemize} 
        Se $d\rho$ e $d\vartheta$ sono "molto piccoli", 
        $dA = dx dy \cong area(Q) \cong lunghezza(L_4) d\rho = \rho d\vartheta d\rho = \rho dA^*$
        con $A = [x,x+dx]\times[y,y+dy]$. \\
        Si pu\aco provare rigorosamente che $dA = \rho dA^*$. \\
        Ritornando al calcolo dell'integrale doppio
        $$\iint_{D}\sqrt{r^2-(x^2+y^2)} \,dx \,dy = \iint_{\overcirc{D}\setminus\{(x,0): 0 \leq x \leq r\}}
        \sqrt{r^2-\rho^2} dA =$$
        $$= \iint_{(0,r)\times(0,2\pi)} \sqrt{r^2-\rho^2} \rho dA^* = \iint_{(0,r)\times(0,2\pi)}
        \sqrt{r^2-\rho^2} \rho d\rho d\vartheta = $$
        $$= \int_{0}^{r}\left(\int_{0}^{2\pi} \sqrt{r^2-\rho^2} \rho \, d\vartheta\right)\, d\rho = 
        2\pi \int_{0}^{r} \sqrt{r^2-\rho^2} \rho \,d\rho$$
        \begin{exercise}
          $\int_{0}^{r} \sqrt{r^2-\rho^2} \rho \,d\rho = \frac{r^3}{3}$
        \end{exercise}
        In conclusione 
        $$volume(S) = \iint_{D} \sqrt{r^2-(x^2+y^2)} \,dx \, dy = \frac{2}{3}\pi r^3 $$
      \end{itemize}
\subsection{Caso generale}
Siano $D, D^* \subseteq \R^2$ aperti limitati e misurabili e sia 
$$\psi:D^*\to D, \psi(u,v) = \left(\psi_1(u,v), \psi_2(u,v)\right) = \left(x(u,v), y(u,v)\right)$$
$\psi_1, \psi_2 : D^* \to \R$
\begin{definition}
  \label{cambiamentovariabili}
  La mappa $\psi$ si dice un cambiamento di variabili se
  \begin{itemize}
    \item $\psi$ \ace bigettiva
    \item $\psi_i \in C^1(D^*)$, $\psi_i, \frac{\p \psi_i}{\p u}, \frac{\p \psi_i}{\p v} : D^* \to \R$ limitate (i=1,2)
    \item $\det D\psi(u,v)\not = 0$ , $\forall (u,v)\in D^*$, dove
          $$D \psi(u,v) := \begin{bmatrix}
            \frac{\p \psi_1}{\p u} (u,v) & \frac{\p \psi_1}{\p v} (u,v)\\
            \\
            \frac{\p \psi_2}{\p u} (u,v) & \frac{\p \psi_2}{\p v} (u,v) \\
          \end{bmatrix} \text{ (Matrice Jacobiana)}$$
  \end{itemize}
  Denotiamo $dA^* = du\,dv$ e $dA = dx\,dy$
\end{definition}
\textbf{Problema:} Legame tra $dA$ e $dA^*$? \\\\
Si pu\aco provare che $dA = \abs[det D\psi(u,v)] dA^*$.\\
Pi\acu precisamente vale:
\subsection{Teorema: Cambiamento di variabili negli integrali doppi}
\begin{theorem}[Cambiamento di variabili negli integrali doppi][BDPG,14.19]
  Siano $D, D^* \subseteq \R^2$ aperti limitati e misurabili, sia $\psi : D^* \to D$
  un cambiamento di variabili e sia $f:D \to \R$ continua e limitata. \\
  Allora vale la formula
  $$(FCV)_2 \, \iint_{D} f(x,y) \,dx\,dy = \iint_{D^*} f(\psi(u,v))\abs{\det D\psi(u,v)} \,du\,dv$$ 
\end{theorem}
\begin{exercise}
  \begin{itemize}
    \item[(i)] Calcolare l'area dell'insieme 
               $$D:=\left\{(x,y)\in\R^2:\frac{x^2}{a^2}+\frac{y^2}{b^2} \leq 1\right\}$$
               dove $a>0, b>0$ fissati. \\\\
               \textbf{Soluzione:} \\\\
               L'insieme D rappresenta un'elisse con semiassi di lunghezza a e b. L'insieme D \ace 
               limitato e misurabile. Infatti:
               \begin{exercise}
                 D \ace un dominio semplice rispetto all'asse y. Quindi D \ace misurabile.
               \end{exercise}
               Per definizione $$area(D) = \abs{D}_2 = \iint_{D} 1 \,dx\,dy$$
               Utilizzando il cambiamento di variabili rispetto a coordinate ellittiche, il calcolo
               dell'integrale doppio diventa abbastanza semplice. Infatti, consideriamo il cambiamento
               $$\left\{\begin{array}{l}
                 x(\rho,\vartheta) = \psi_1(\rho,\vartheta):= a\rho\cos\vartheta \\
                 y(\rho,\vartheta) = \psi_2(\rho,\vartheta):= b\rho\cos\vartheta \\
               \end{array}\right. \rho \geq 0, \vartheta \in [0,2\pi]$$
               e sia $D^* := (0,1)\times (0,2\pi)$, $\psi : D^* \to \R^2$, 
               $\psi(\rho,\vartheta):=\left(\psi_1(\rho,\vartheta), \psi_2(\rho,\vartheta)\right)$
               \begin{exercise}
                 Verificare che la mappa $\psi : D^* \to \overcirc{D}\setminus\{(x,0):0\leq x\leq a\}$
                 \ace un cambiamento di variabili, in accordo con la definizione \ref{cambiamentovariabili} prima introdotta. 
                 Inoltre $\det D\psi(\rho,\vartheta) = ab\rho$.
               \end{exercise}
               Possiamo applicare $(FCV)_2$ con $f\equiv 1$ su D, ed otteniamo
              $$area(D) = \iint_{D} 1 \,dx\,dy = \iint_{\overcirc{D}\setminus\{(x,0):0\leq x\leq a\}} 1 \,dx \,dy =$$
              $$= \iint_{D^*} 1 \cdot \abs{\det D\psi(\rho, \vartheta)} \,d\rho\,d\vartheta = 
              2\pi ab \int_{0}^{1} \rho \,d\rho\,d\vartheta = \pi ab$$
    \item[(ii)] Calcolare l'integrale doppio $$\iint_{D} \frac{y^2}{x} \,dx \,dy$$
                Dove $D:=\left\{(x,y)\in\R^2 : x^2 \leq y \leq 2x^2, y^2 \leq x \leq 3y^2\right\}$ \\\\
                \textbf{Soluzione:}\\\\
                L'insieme D pu\aco essere viso come $D = D_1 \cap D_2$, dove 
                \begin{itemize}
                  \item $D_1 = \{(x,y)\in\R^2 : x^2 \leq y \leq 2x^2\}$
                  \item $D_2 = \{(x,y)\in\R^2y^2 \leq x \leq 3y^2\}$
                \end{itemize}
                Incominciamo a studiare la geometria di D. \ac{E} chiaro che $(0,0)\in D$. Supponiamo che 
                $(x,y)\in D\setminus\{(0,0)\}$ allora per definizione di D, 
                $(x,y)\in \left(D_1\setminus\{(0,0)\}\right) \cap \left(D_2 \setminus \{(0,0)\}\right)$. \ac{E} chiaro
                che, per come sono definiti $D_1$ e $D_2$, necessariamente 
                \begin{enumerate}
                  \item $x>0$ e $y>0$
                  \item $x^2 \leq y\leq 2x^2$
                  \item $y^2\leq x\leq 3y^2$
                \end{enumerate}
                Dividendo la disuguaglianza (2) per $x^2$ e la (3) per $y^2$, grazie alla condizione (1), si intuisce
                che un possibile cambiamento di variabili $x=\psi_1(u,v)$, $y = \psi_2(u,v)$ potrebbe essere 
                quello per cui 
                $$\left\{\begin{array}{l}
                  u = \frac{y}{x^2} \\
                  \\
                  v = \frac{x}{y^2}
                \end{array}\right. \text{ con } 1 \leq u \leq 2 , 1 \leq v \leq 3$$
                \begin{exercise}
                  Risolvere il sistema precedente rispetto ad x e y.
                \end{exercise}
                Otteniamo 
                $$\left\{\begin{array}{l}
                  x = x(u,v) = \psi_1(u,v) = \frac{1}{u^{\frac{2}{3}}v^{\frac{1}{3}}} \\
                  \\
                  y = y(u,v) = \psi_2(u,v) = \frac{1}{u^{\frac{1}{3}}v^{\frac{2}{3}}} \\
                \end{array}\right.$$
                Sia $D^* :=\{(u,v) \in \R^2 : 1 < u < 2, 1 < v < 3\}$
                \ace chiaro per costruzione che:
                \begin{itemize}
                  \item $\psi : D^* \to \overcirc{D}$ \ace bigettiva
                  \item $D^*$ e $\overcirc{D}$ sono limitati e misurabili (da \ref{cormis})
                  \item $\psi_i \in C^1(D^*)$, i =1,2
                  \item \begin{exercise}
                    $$D\psi(u,v) = \begin{bmatrix}
                      -\frac{2}{3} u^{-\frac{5}{3}}v^{-\frac{1}{3}} & -\frac{1}{3} u^{-\frac{2}{3}}v^{-\frac{4}{3}}  \\
                      -\frac{1}{3} u^{-\frac{4}{3}}v^{-\frac{2}{3}} & -\frac{2}{3} u^{-\frac{1}{3}}v^{-\frac{5}{3}} 
                    \end{bmatrix}$$
                  \end{exercise}
                \end{itemize}
                $\det D\psi(u,v) = \frac{1}{3} u^{-2} v^{-2}$ se $(u,v) \in D^*$
                Possimao applicare l'osservazione (??) e $(FCV)_2$, ottenendo che 
                $$\iint_{D} \frac{y^2}{x} \,dx\,dy = \iint_{\overcirc{D}} \frac{y^2}{x} \,dx \,dy = $$
                $$=\iint_{D^*} \frac{1}{v} \frac{1}{3} \frac{1}{u^2} \frac{1}{v^2} \,du\,dv = 
                \frac{1}{3}\iint_{D^*}\frac{\,du\,dv}{u^2v^3}$$
                L'ultimo integrale doppio risulta essere un integrale doppo su un rettangolo, applicando la formula
                di riduzione sui rettangoli otteniamo:
                $$\iint_{D^*} \frac{\,du\,dv}{u^2v^3} = \int_{1}^{2} u^{-2} \,du \cdot \int_{1}^{3} v^{-3} \, dv = $$
                $$= \left.-\frac{1}{u}\right|_{1}^{2} \cdot \left.-2v^{-2}\right|_{1}^{3} = \frac{2}{9}$$
                Pertanto 
                $$\iint_{D} \frac{y^2}{x} = \frac{1}{3} \cdot \frac{2}{9} = \frac{2}{27}$$
  \end{itemize}
\end{exercise}
  \section{Lez - 13, Integrali tripli, [BDPG,14.5]}
\subsection{Integrale triplo su un parallelepipedo}
Sia $Q:= [a_1,b_1]\times [a_2,b_2]\times [a_3,b_3]  \subset \R^3$ un parallelepipedo.
Siano
\begin{itemize}
  \item $\D_1 := \{a_1 = x_0 \leq \dots \leq x_i \leq \dots x_{n_1} = b_1\}$ sudd. di $[a_1,b_1]$
  \item $\D_2 := \{a_2 = y_0 \leq \dots \leq y_j \leq \dots y_{n_2} = b_1\}$ sudd. di $[a_2,b_2]$
  \item $\D_3 := \{a_3 = z_0 \leq \dots \leq z_k \leq \dots z_{n_3} = b_3\}$ sudd. di $[a_3,b_3]$
\end{itemize}
\begin{definition}
  Si chiama \underline{suddivisione $\D$} del parallelepipedo $Q=[a_1,b_1]\times [a_2,b_2]\times [a_3,b_3]$
  un insieme del tipo 
  $$\D = \D_1 \times \D_2 \times \D_3 = \{(x_i,y_j,z_k): i = 0,...,n_1; j = 0,...,n_2; k = 0,..., n_3\}$$
  se $\D_1, \D_2, \D_3$ sono definiti come sopra. 
\end{definition}
Data una suddivisione $\D$ di Q, Q risulta suddiviso in $n_1\times n_2 \times n_3$ parallelepipedi
$$Q_{ijk} = [x_{i-1}, x_i]\times [y_{j-1},y_j] \times [z_{k-1},z_k]$$
con $i = 1,..., n_1$, $j = 1,...,n_2$, $k = 1,..., n_3$, il cui volume \ace 
$$vol(Q_{ijk}) = \abs{Q_{ijk}}_3 := (x_i-x_{i-1})(y_j-y_{j-1})(z_k-z_{k-1})$$
Sia $f : Q \to \R$ limitata e definiamo
$$m_{ijk} = inf_{Q_{ijk}} f \in \R \text{ e } M_{ijk} = sup_{Q_{ijk}} f \in \R$$
con $i = 1,..., n_1$, $j = 1,...,n_2$, $k = 1,..., n_3$
Definiamo 
\begin{itemize}
  \item $S(f,Q) = \sum_{ijk} M_{ijk} \cdot \abs{Q_{ijk}}_3$ (somma superiore di f su Q)
  \item $s(f,Q) = \sum_{ijk} m_{ijk} \cdot \abs{Q_{ijk}}_3$ (somma inferiore di f su Q)
\end{itemize}
\begin{definition}
  Si dice che f \ace integrabile su Q e si scrive $f \in \Rcal(Q)$ se 
  $$L = sup_{\D} s(f,\D) = inf_{\D} S(f,\D) \in \R$$
  Il valore L prende nome di \underline{integrale triplo} di f su Q e si denota con i simboli
  $$\iiint_{Q} f(x,y,z) \,dx\,dy\,dz, \iiint_{Q} f, \int_{Q} f, \int_{Q} f \,dx\,dy\,dz$$
\end{definition}
Continuano a valere i risultati degli interali doppi su rettangoli.
\subsubsection{Propriet\aca}
\begin{theorem}[Esistenza dell'integrale]
  Sia $f \in C^0\left(Q\right)$ allora $f \in \Rcal(Q)$
\end{theorem}
\begin{theorem}[Propriet\aca dell'integrale]
  Siano $f,g \in \Rcal (Q)$
  \begin{itemize}
    \item[(i)] \textbf{Linearit\aca}: $\alpha f + \beta g \in \Rcal(Q)$, $\forall \alpha,\beta\in\R$ e 
              $$\int_{Q}\left(\alpha f + \beta g\right) = \alpha \int_{Q}f + \beta \int_{Q} g$$
    \item[(ii)] \textbf{Monotonia}: Se $g\leq f$ su Q, allora $$\int_{Q} g \leq \int_{Q}f$$
    \item[(iii)] \textbf{Valore assoluto}: $\abs{f} \in \Rcal(Q)$ e $$\abs{\int_{Q}f} \leq \int_{Q}\abs{f}$$
    \item[(iv)] \textbf{Teorema della media interale} 
                $$inf_{Q} f \leq \frac{1}{\abs{Q}_3}\int_{Q} f \leq sup_{Q}f$$
                Se $f \in C^0(Q)$, allora esiste $p_0 = (x_0,y_0,z_0)\in Q$ t.c. 
                $$f(p_0) = \frac{1}{\abs{Q}_3}\int_{Q} f$$
  \end{itemize}
\end{theorem}
\subsection{Integrale triplo su insiemi generali}
\begin{definition}
  Sia $A\subseteq \R^3$ limitato, $f: A \to \R$ limitata. Allora f si dice \underline{integrabile}
  su A, e si scrive $f\in\Rcal(A)$ se la funzione $\ftilde:Q\to\R$ definita come
  $$ \ftilde(x,y,z) := \left\{\begin{array}{ccl}
    f(x,y,z) & se & (x,y,z)\in A \\
    0 & se & (x,y,z)\in Q\setminus A \\
  \end{array}\right.$$
  \ace integrabile su Q, dove Q \ace un (qualunque) parallelepipedo contenente A.\\
  Si pone: $$\int_A f := \int_{Q} \ftilde$$ 
\end{definition}
\subsection{Formule di riduzione per integrali tripli}
\subsubsection{Applicazione della formula di riduzione per strati: volume di un solido di rotazione}
\subsection{Cambiamento di variabili negli integrali tripli}
  \chapter{Curve ed integrali curvilinei, [BDPG, 12]}
\section{Lez - 14, Curve in $\R^n$}
\begin{definition}
  \begin{itemize}
    \item[(i)] Si chiama \underline{curva} una mappa $\gamma:I \to \R^n$ continua, 
                $\gamma(t) = (\gamma_1(t),...,\gamma_n(t))$
                con $I$ intervallo di $\R$
    \item[(ii)] Se $I = [a,b]$, i punti $\gamma(a)$, $\gamma(b)$ di $\R^n$ si chiamano
                \underline{estremi} della curva
    \item[(iii)] Si chiama \underline{sostegno} (o supporto) della curva $\gamma$, l'insieme 
                  $\gamma(I) \subseteq \R^n$. Si chiama \underline{equazione parametrica} di $\gamma$ 
                  l'equazione $x = (x_1,...,x_n) = \gamma(t) \, t \in I$
    \item[(iv)] La curva $\gamma$ si dice \underline{chiusa} se $I = [a,b]$ e $\gamma(a) = \gamma(b)$
    \item[(v)] La curva $\gamma:I\to\R^n$ si dice \underline{semplice} se $\gamma$ \ace iniettiva, o 
                se $\gamma$ \ace chiusa e $I = [a,b]$, allora $\gamma :[a,b) \to \R^n$ \ace iniettiva.
  \end{itemize}
\end{definition}
I casi pi\acu significativi di curve, interessanti per le applicazioni, sono in $n=2,3$.
\begin{example}
  \begin{itemize}
    \item[(i)] Sia $f\in C^0([a,b])$ e consideriamo le curve $\gamma, \gamma^* : [a,b]\to \R^2$,
              $\gamma(t) = (t,f(t))$ e $\gamma^*(t) = (f(t),t)$, $t \in [a,b]$.\\
              $\gamma, \gamma^*$ sono dette \underline{curve piane cartesiane}.
              \begin{itemize}
                \item Gli estremi della curva $\gamma$ sono i punti: $(a,f(a)), (b,f(b))$
                \item Gli estremi della curva $\gamma^*$ sono i punti: $(f(a),a), (f(b),b)$
              \end{itemize}
              Il supporto della curva $\gamma,\gamma^*$ coincide rispettivamente, con il grafico della funzione
              f, $G_f$, vista, nel primo caso, come funzione di y rspetto a x, cio\ace 
              $$G_f := \{(t,f(t)):t\in[a,b]\}$$
              e, nel secondo caso, come funzione di x rispetto a y, cio\ace $$G_f :=\{(f(t),t):t\in[a,b]\}$$
              Osserviamo che le due curve sono semplici e non chiuse. \\
              Le eq. parametriche sono, rispettivamente, 
              \begin{itemize}
                \item $(x,y) = \gamma(t) = (t,f(t)) \, t \in [a,b] \iff \left\{\begin{array}{l}
                  x = t \\
                  y = f(t) \\
                \end{array}\right.$, $t \in [a,b]$ e 
                \item $(x,y) = \gamma(t) = (f(t),t) \, t \in [a,b] \iff \left\{\begin{array}{l}
                  x = f(t)\\
                  y = t \\
                \end{array}\right.$, $t \in [a,b]$
              \end{itemize}
    \item[(ii)] Sia $\gamma:[0,2\pi] \to \R^2$ la curva definita da $\gamma(t) = (\cos(t),\sin(t))$, $t \in [0,2\pi]$
                \ac{E} faccile verificare che $\gamma$ \ace una curva piana chiusa $(\gamma(0) = (1,0) = \gamma(2\pi))$ e 
                semplice. \\
                Il sostegno di $\gamma$,$\gamma([0,2\pi])$ \ace dato da 
                $$C = \{(x,y)\in\R^2 : x^2+y^2 = 1\}$$
                L'equazione parametrica di $\gamma$ \ace data da 
                $$(x,y) = \gamma(t) = (\cos(t), \sin(t)), t \in [0,2\pi] \iff \left\{\begin{array}{l}
                  x = \cos(t) \\
                  y = \sin(t) \\
                \end{array}\right.t \in [0,2\pi]$$
    \item[(iii)] Sia $\gamma : [0,4\pi] \to \R^2$ la curva definita come nell'esempio (ii), cambaindo il dominio.\\
                  La curva \ace ancora una curva piana chiusa $(\gamma(0) = (1,0) = \gamma(4\pi))$ ma non \ace semplice.
                  Infatti la funzione $\gamma [0,4\pi) \to \R^2$ non \ace iniettiva.\\
                  La curva ha come sostego $C$ dell'esempio (ii). Da un punto di vista intuitivo, il sostegno di $\gamma$
                  \ace percorso due volte.
                  \begin{NB}
                    Due curve possono avere lo stesso sostegno ma essere differenti, come gli esempi (ii) e (iii)
                  \end{NB}
    \item[(iv)] Sia $\gamma:\R \to \R^3$ la curva definita da $\gamma(t) = (\cos(t),\sin(t),t)$, $t\in\R$
                  $\gamma$ \ace una curva semplice, on chiusa ed il suo sostegno rappresenta un'elica infinia
                  contenuta nel cilindro $\{(x,y,z)\in\R^3:x^2+y^2=1\}$. \\
                  L'eq. parametrica \ace data da 
                  $$(x,y,z) = \gamma(t) = (\cos(t),\sin(t),t) t \in \R \iff \left\{\begin{array}{l}
                    x = cos(t) \\
                    y = sin(t) \\
                    z = t \\
                  \end{array}\right. t \in \R$$
  \end{itemize}
\end{example}
\subsection{Orientazione di una curva semplice}
Sia data una curva semplice $\gamma:I\to\R^n$. Allora 
essa induce \underline{un'orientazione} sul suo sostegno $\gamma(I) \subseteq \R^n$. \\
Pi\acu precisamente
\begin{definition}
  Data $\gamma:I\to\R^n$ curva semplice, si dice che il punto $x_1 = \gamma(t_1)$ \underline{precede}
  il punto $x_2 = \gamma(t_2)$ se $t_1 < t_2$. L'orientazione della curva viene detta
  anche \underline{verso} della curva.
\end{definition}
\begin{example}
  Le curve degli esempi (i), (ii), (iv), essendo semplici, sono tutte orientabili, mentre 
  la curva (iii) non essendo semplice, non \ace orientabile.
\end{example}
\subsection{Vettore velocit\aca di una curva}
Sia $\gamma:I\to\R^n$ una curva. Se le componenti $\gamma_i :I \to \R$ ($i = 1,...,n$)
sono derivabili in un fissato punto $t_0 \in I$, il vettore
$$\gamma'(t_0) = (\gamma_1'(t_0),...,\gamma_n')$$
\ace detto \underline{vettore velocit\aca} di $\gamma$ in $t_0$.\\
Essendo la funzione $\gamma_i$ derivabile, sappiamo che 
$$(*) \, \gamma_i (t) = \gamma_i(t_0) + \gamma_i'(t_0)(t-t_0) + o(t-t_o) \, (t\to t_0)$$
per $i=1,...,n$. \\
Una forma pi\acu compatta per scrivere (*) \ace  
$$\gamma(t) = \gamma(t_0) + \gamma'(t_0)(t-t_0) + o(t-t_o) \, (t\to t_0)$$
\begin{osservazione}
  \ac{E} immediato verificare che $\gamma'(t_0) = D\gamma(t_0)^{T}$, dove 
  $$D\gamma(t_0) = \begin{bmatrix}
    \gamma_1'(t_0) \\
    \vdots \\
    \gamma_n'(t_0) \\
  \end{bmatrix} \text{ matrice Jacobiana } n\times 1$$
\end{osservazione}
\begin{definition}
  Se $\gamma'(t_0) \neq \origine_{\R^n}$, si chiama \underline{retta tangente} alla curva $\gamma$ nel 
  punto $x_0 = \gamma(t_0)$ la retta di equazione parametrica
  $$x = \gamma(t_0)+\gamma'(t_0)(t-t_0) = \gamma_1(t_0)+\gamma_1'(t_0)(t-t_0) + ... + \gamma_n(t_0)+\gamma_n'(t_0)(t-t_0) $$
  se $t \in I$
\end{definition}
\begin{osservazione}
  \begin{itemize}
    \item[(i)] Sia $n=2$, $\gamma(t) = (\gamma_1,(t)\gamma_2(t))$, $t\in I$, 
              $p_0 = \gamma(t_0) = (x_0,y_0)$, $\gamma'(t_0) = (v_1,v_2) \neq (0,0)$. 
              Supponiamo per esempio, $v_2\neq 0$. \\
              L'eq. parametrica della retta tangente diventa
              $$\left\{\begin{array}{l}
                x = v_1(t-t_0)+x_0 \\
                y = v_2(t-t_0)+y_0
              \end{array}\right. \iff \left\{\begin{array}{l}
                t-t_0 = \frac{y-y_0}{v_2} \\
                \\
                x = v_1(t-t_0) + x_0 \\
              \end{array}\right.$$
              $$\Rightarrow x = \frac{v_1}{v_2} (y-y_0) + x_0 \iff $$
              $$\iff r: v_2(x-x_0) - v_1(y-y_0) = 0$$
              (eq. di una retta nel piano x,y passante per $(x_0,y_0)$ di direzione v)
              \begin{NB}
                Si noti che il vettore $(v_2,-v_1)$ \ace \underline{ortogonale} al vettore $(v_1,v_2)$, in quanto
                $(v_1,v_2)\cdot(v_2,-v_1) = 0$, e la retta $r$ pu\aco essere riscritta come:
                $$r: (x-x_0,y-y_0)\cdot(v_2,-v_1)$$
              \end{NB}
    \item[(ii)] Se $\gamma'(t_0) = \origine_{\R^n}$, la retta tangente pu\aco non esistere
  \end{itemize}
\end{osservazione}
\begin{example}
  $n=2$, $\gamma : \R\to\R^2$, definita come $\gamma(t) = (x_0,y_0) \, \forall t \in \R$.\\
  Il sostegno di $\gamma$ \ace il punto $(x_0,y_0)$: non \ace una curva regolare.
\end{example}
\begin{definition}
  Una curva $\gamma:I\to\R^n$
  \begin{itemize}
    \item[(a)] si dice di classe $C^m$ se $\gamma_i : I \to \R$ sono di classe $C^m \, \forall i = 1,...,n$
    \item[(b)] si dice \underline{regolare} se $\gamma$ \ace di classe $C^1$ e $\gamma'(t) \neq \origine_{\R^n}$ $\forall t \in I$ 
  \end{itemize}
\end{definition}
\begin{definition}
  Data $\gamma : I \to \R^n$ curva regolare, si chiama \underline{versore} (o direzione) tangente a $\gamma$
  il campo vettore $$\T_{\gamma} (t) := \frac{\gamma'(t)}{\norma{\gamma'(t)}} \, t \in I$$
\end{definition}
\begin{definition}
  Una curva $\g : [a,b]\to\R^n$ si dice $C^1$ a tratti (o regolare a tratti) se esiste una 
  suddivisione $a = t_0 < ... < t_n = b$ di $[a,b]$ t.c.
  $$\g|_{[t_{i-1},t_i]}:[t_{i-1},t_i] \to \R^n$$
  \ace di classe $C^1$ (rispettivamente regolare).\\
  In tal caso $\g$ si dice anche uunionce delle $N$ curve $\g_i := \g|_{[t_{i-1},t_i]}$
  e si scrive $$\g:= \bigcup_{i=1}^{N} \g_i$$
\end{definition}
\begin{example}
  Sia $\g:[-1,1]\to \R^2$ la curva $\g(t) = (t,\abs{t})$, $t \in [-1,1]$. \\
  Allora \ace facile verificare che $\g$ una curva regolare a tratti. Infatti
  se $-1 = t_0 < 0 = t_1 < t_2 = 1$, \ace facile verificare che 
  $$\g_i \equiv \g|_{[t_{i-1},t_i]} : [t_{i-1},t_i] \to \R^2$$
  \ace \underline{regolare}. \\
  Poich\ace 
  \begin{itemize}
    \item $\g_1 := (t,-t)$, $t\in[t_0,t_1]$
    \item $\g_1 := (t,t)$, $t\in[t_1,t_2]$
  \end{itemize}
  Si noti che il sostegno di $\g$ \ace il grafico della funzione $f:[-1,1]\to\R$, $y = f(x) = \abs{x}$
\end{example}
\subsection{Cambiamento di parametro di una curva} 
\begin{definition}
  Due curve $\g : I \to \R^n$, $\gtilde:\widetilde{I}\to\R^n$ di classe $C^1$ si dicono \underline{equivalenti}
  se esiste una funzione bigettiva $\varphi:\widetilde{I}\to I$ t.c. 
  $$\varphi \in C^1(\widetilde{I}); \varphi'(\tau) \neq 0 \, \forall \tau \in \widetilde{I};
  \gtilde(\tau) = \g(\varphi(\tau)) \tau \in \widetilde{I};$$
  In tal caso $\tau \rightarrow t = \varphi(\tau) \in I$ si dice \underline{cambiamento di parametrizzazione}.\\
  Se $\varphi'(\tau)>0$, $\forall\tau \in\widetilde{I}$, allora si dice che $\g$, $\gtilde$ hanno
  \underline{lo stesso verso}; Se $\varphi'(\tau)<0$, $\forall\tau \in\widetilde{I}$, allora si dice che $\g$, $\gtilde$ hanno
  \underline{verso opposto}
\end{definition}
\begin{exercise}
  Siano
  \begin{itemize}
    \item $\g(t) := (cos(t), sin(t))$, $t \in [0,2\pi]$
    \item \item $\gtilde(\tau) := (cos(2\tau), sin(2\tau))$, $\tau \in [0,\pi]$
    \item $\g^*(s) := (cos(s), -sin(s))$, $s \in [0,2\pi]$
  \end{itemize}
  Provare che:
  \begin{enumerate}
    \item $\g,\gtilde, \g^*$ sono equivalenti
    \item $\g,\gtilde$ hanno lo stesso verso, mentre $\g,\g^*$ hanno verso opposto
  \end{enumerate}
  \textbf{Soluzione:}(suggerimento)\\\\
  1. Per provare che $\g, \gtilde$ sono eq. basta considerare il cambiamento
  di parametrizzazione $\varphi:[0,\pi]\to[0,2\pi]$, $\varphi(\tau):= 2\tau$
  per provare che $\g,\g^*$ sono eq. basta considerare il cambiamento di 
  parametrizzazione $\varphi:[0,2\pi]\to[0,2\pi]$, $\varphi(s) 2\pi-s$
\end{exercise}
\begin{osservazione}
  Si pu\aco che: date due curve equivalenti, allora
  \begin{enumerate}
    \item esse hanno lo stesso sostegno
    \item se una delle due fosse semplice, anche l'altra sarebbe semplice
  \end{enumerate}
\end{osservazione}
  \section{Lez - 15, Lunghezza di una curva}
Vogliamo ora definire la nozione di lunghezza di una curva.\\
Sia $\g:[a,b]\to\R^n$ una curva e sia $\D := {t_0 = a < t_1 < ... < t_N = b}$ 
una suddivisione di $[a,b]$: essa induce una suddivisione del sostegno di 
$\g$ in $N+1$ parti definite da 
$\g(t_0),\g(t_1)\dots \g(t_N)$. \\
Consideriamo i segmenti
$$[\g(t_{i-1}),\g(t_i)] := \{s\g(t_i)+(1-t)\g(t_{i-1}) : 0 \leq s \leq 1\}$$
$i = 1,...,N$. La lunghezza della spezzata definita dall'unione 
$\bigcup_{i=1}^{N} [\g(t_{i-1}),\g(t_i)]$ \ace data da 
$$L(\g,\D) := \sum_{i=1}^{N} \norma{\g(t_i)-\g(t_{i-1})} \in [0,+\infty)$$
Denotiamo 
$$L(\g) := sup_{\D} L(\g,\D) \in [0,+\infty] =_{def} [0,+\infty) \cup \{+\infty\}$$
\begin{definition}
  Sia $\g:[a,b]\to\R^n$ una curva. Se $L(\g) < +\infty$, allora la curva si dice
  \underline{rettificabile} e $L(\g)$ \ace detta \underline{lunghezza} di $\g$
\end{definition}
\begin{osservazione}
  Si pu\aco provare che esistono curve per cui $L(\g) = +\infty$, vedi esempio [BDPG,12.5].
\end{osservazione}
\begin{theorem}[Lunghezza di una curva][BDPG,12.10]
  \label{lunghezzacurva}
  Sia $\g:[a,b]\to\R^n$ una curva di classe $C^1$. Allora $\g$ \ace rettificabile e 
  $$L(\g)= \int_{a}^{b} \norma{\g'(t)} \,dt = \int_{a}^{b} \sqrt{\g_1'(t)^2 + ... + \g_n'(t)^2}\,dt$$
\end{theorem}
\begin{corollary}[Lunghezza curve piane cartesiane]
  Sia $\g:[a,b]\to\R^2$ una curva piana cartesiana di classe $C^1$, cio\ace
  $$\g(t)=\left\{\begin{array}{ll}
    (t,f(t)) & t \in [a,b] \\
    \text{oppure} \\
    (f(t),t) & t \in [a,b]
  \end{array}\right.$$
  con $f\in C^1([a,b])$. Allora $\g$ \ace rettificabile e 
  $$L(\g) = \int_{a}^{b} \sqrt{1+f'(t)^2} \,dt$$
  \begin{example}
    Sia $f(t)=t^2$, allora $L(\g) = \int_{0}^{1} \sqrt{1+4t^2} \,dt$
  \end{example}
\end{corollary}
\begin{theorem}[Indipendenza della lunghezza dalla parametrizzazione]
  Siano $\g:[a,b] \to \R^n$ e $\gtilde : [\alpha,\beta]\to \R^n$ due curve di classe $C^1$
  equivalenti. Allora $$L(\g) = L(\gtilde)$$
  \begin{proof}
    Sia $\varphi:[\alpha,\beta]\to[a,b]$ il cambiamento di parametrizzazione, cio\ace 
    $$\gtilde (\tau) = \g(\varphi(\tau)) \, \forall\tau\in[\alpha,\beta]$$
    $\varphi\in C^1$ e bigettiva. \\ 
    Supponiamo, per esempio, che $\varphi'(\tau)>0$ $\forall\tau\in[\alpha,\beta]$. Allora per 
    il Teorema \ref{lunghezzacurva} e (RDC) 
    $$L(\gtilde) =_{\ref{lunghezzacurva}} \int_{\alpha}^{\beta} \norma{\gtilde'(\tau)}\,d\tau =_{(RDC)}
    \int_{\alpha}^{\beta} \norma{\g'(\varphi(\tau))\cdot \varphi'(\tau)} \,d\tau = $$
    $$= \int_{\alpha}^{\beta} \norma{\g'(\varphi(\tau))} \varphi'(\tau) \,d\tau = $$
    Poniamo $t = \varphi(\tau)$ e otteniamo
    $$= \int_{\varphi(\alpha)}^{\varphi(\beta)} \norma{\g'(t)}\,dt = \int_{a}^{b} \norma{\g'(t)}\,dt = L(\g)$$
  \end{proof}
\end{theorem}
\begin{osservazione}
  \ac{E} facile verificare che una curva $C^1$ a tratti \ace rettificabile e, se 
  $\g = \bigcup_{i=1}^{N} \g_i$, con $\g_i: [t_{i-1},t_i] \to \R^n$ di classe 
  $C^1$, allora 
  $$L(\g) = \sum_{i=1}^{N} \int_{t_{i-1}}^{t_i} \norma{\g_i'(t)} \,dt$$
\end{osservazione}
\section{Integrali curvilinei di I specie}
\textbf{Motivazione fisica:} Sia $\g:[a,b]\to\R^3$ una curva di classe $C^1$ e supponimao che 
il sostegno di $\g$, $\Gamma:= \g([a,b])\subseteq\R^3$ modelizzi 
un filo rigiido dello spazio di densit\aca lineare f, ovvero f ha la dimensione di una 
massa x unit\aca di lunghezza.\\
Se f fosse costante, M:= massa totale filo, allora
$$M = f\cdot L(\g) = \int f\norma{\g'(t)}\,dt := \int_{\g} f \,ds$$
$ds \cong \norma{\g'(t)} \,dt$ elemento infinetesimale di lunghezza.\\
In generale, se la densit\aca f non fosse costante, $f:\Gamma \to \R$ e dunque 
$$M =\int_{a}^{b} f(\gamma(t)) \norma{\gamma'(t)}\,dt$$
\begin{definition}
  Sia $\g : [a,b]\to\R^n$ una curva di classe $C^1$ e sia $f:\Gamma \to \R$
  una funzione continua. Si definisce
  $$\int_{\gamma} f \,ds = \int_{a}^{b} f(\gamma(t))\norma{\g'(t)} \,dt$$
  e si chiama \underline{Integrale curvilineo} di I specie di f lungo $\g$.
\end{definition}
\begin{notazione}
  Se $\g$ fosse una crva chiusa e semplice su usa anche il simbolo $\oint_{\g}f\,ds$
\end{notazione}
\begin{osservazione}
  \begin{itemize}
    \item L'integrale curvilineo di I specie \ace lineare.
    \item L'integrale curvilineo di I psecie si estende a curve $C^1$ a tratti. Infatti
          $\g:[a,b]\to\R^n$ una curva $C^1$ a tratti e
          $\g = \bigcup_{i=1}^{N} \g|_{[t_{i-1},t_i]} : [t_{i-1},t_i] \to\R^n$, $i = 1,...,N$ di classe 
          $C^1$; sia $f:\Gamma \to \R$ continua. Allora possiamo definire
          $$\int_{\g}f\,ds:=\sum_{i=1}^{N}\int_{\g|_{t_{i-1},t_i}} f \,ds$$
  \end{itemize}
\end{osservazione}
\begin{proposition}
  Siano $\g:[a,b]\to\R^n$, $\gtilde:[\alpha,\beta]\to\R^n$ curve di classe $C^1$
  equivalenti e sia $f:\Gamma = \g([a,b]) = \gtilde([\alpha,\beta]) \to \R$ continua. 
  Allora $$\int_{\g} f\,ds = \int_{\gtilde} f \,ds$$
  \begin{exercise}
    Dimostrazione.
  \end{exercise}
\end{proposition}
\section{Integrali curvilinei di II specie: campi vettoriali e forme 
differenziali}
\subsection{Campi vettoriali e forme differenziali}
\begin{definition}
  Si chiama \underline{campo vettoriale} su un insieme $E\subseteq \R^n$ una mappa 
  $F:E\to \R^n$, $F(x) = (F_1(x),...,F_n(x))$ $x \in E$, 
  $F_i : E \to \R$
\end{definition}
\begin{osservazione}
  In fisica/ingegneria un campo vettoriale pu\aco rappresentare una forza applicata in 
  un punto $x\in E$, dove $E$ \ace una regione del piano o dello spazio,
  $E\subseteq \R^2$ o $E\subseteq \R^3$
\end{osservazione}
\begin{definition}
  Dato un campo vettoriale $F:E \to \R^n$, si chiama \underline{forma differenziale} (lineare) su E 
  l'espressione formale
  $$\omega = F_1\,dx_1+...+F_n\,dx_n = \sum_{i=1}^{n}F_i\,dx_i = \tuple{F,dx}$$
\end{definition}
\begin{osservazione}
  Dalla definizione, si evince che ad ogni 
  $$F:E\to\R^n \rightarrow \omega:= \tuple{F,dx}$$ 
  Viceversa, data 
  $$\omega = \tuple{F,dx} \text{ forma differenziale su E} \rightarrow F:E\subseteq\R^n\to\R^n$$
  Pertanto si pu\aco stabilire una \underline{corrispondenza biunivoca} tra:
  $$\text{campo vettoriale} \iff \text{forma differenziale}$$
\end{osservazione}
\begin{definition}
  Una forma differenziale $\omega = \tuple{F,dx}$ su un insieme $E \subseteq\R^n$ si dice 
  di classe $C^0$(risp $C^1$) se $F_i \in C^0(E)$ (risp $F_i \in C^1(E)$) $\forall i = 1,...,n$
\end{definition}
\textbf{Motivazione fisica:}[Lavoro compiuto da una forza lungo un percorso] \\
Sia $n=3$, $F:\R^3\to\R^3$ una forza assegnata, 
$$F(x,y,z) := \left(F_1(x,y,z),F_2(x,y,z),F_3(x,y,z)\right)$$
se $(x,y,z)\in\R^3$ con $F_i : \R^3\to\R$ ($i=1,2,3$) funzione continua. \\
Sia $\g:[a,b]\to \R^3$, $\g(t) = \left(\g_1(t), \g_2(t),\g_3(t)\right) = \left(x(t),y(t),z(t)\right)$
una curva di classe $C^1$.\\
La forma diff. $\omega$ rappresenta il \underline{lavoro} compiuto dalla forza 
$F$ su un punto materiale che si muove di uno spostamento infinitesimo 
$$(dx,dy,dz) =(x'(t)\,dt,y'(t)\,dt,z'(t)\,dt) $$ 
lungo la curva $\g$. \\
Pi\acu precisamente, se il punto si muovesse lungo la curva $\g$ e all'istante $t$ si trovasse
nella posizione $\g(t)$, allora il lavoro compiuto dalla forza nell'intervallo infinitesimo ddi tempo 
$dt$ sarebbe dato da $\tuple{F(\g(t)),\g'(t)}\,dt$
\begin{osservazione}
  Ricordare che $\tuple{F(\g(t)),\g'(t)} = \norma{F(\g(t))}\norma{\g'(t)}\cos(\vartheta)$
  dove $\vartheta = $ angolo formato dai vettori $F(\g(t))$ e $\g'(t)$
\end{osservazione}
La motivazione fisica suggerisce la seguente definizione:
\begin{definition}
  Sia $\g:[a,b]\to E\subseteq\R^n$ una curva di classe $C^1$ e sia
  $\omega$ una forma differenziale di classe $C^0$ su E. \\
  Si definisce \underline{integrale curvilineo} di II specie di $\omega$
  (o del campo F) lungo $\g$ il valore 
  $$\int_{\g} \omega := \int_{a}^{b} \tuple{f(\g(t)), \g'(t)}\,dt = 
    \int_{a}^{b} \sum_{i=1}^{n} F_i(\g(t))\g_i'(t) \,dt$$
  Se $\g$ fosse chiusa il precedente integrale si scrive anche $\oint_{\g} \omega$
\end{definition}
\begin{osservazione}
  \begin{enumerate}
    \item L'integrale curvilineo di II specie \ace lineare
    \item L'integrale curvilineo di II specie si estende a curve $C^1$ a tratti. Infatti
          data $\g = \bigcup_{i=1}^{N}\g_i :[a,b]\to E\subseteq\R^n$ una curva $C^1$ a tratti
          e $\omega$ una forma differenziale continua su E, allora si definisce
          $$\int_{\g} \omega := \sum_{i=1}^{N} \int_{\g_i}\omega$$
  \end{enumerate}
\end{osservazione}
  \subsection{Lez - 16, Teo: INtegrale curvilineo di II specie rispetto a curve eq.}
\subsection{Forme differenziali esatte (o campi vettoriali conservativi)}
\subsection{Forma differenziali chiuse}
\subsubsection{Teo: Chiusa = Esatta}
\subsection{Costruzione di un potenziale per una forma diff. chiusa su aperto conv.}

  \chapter{Superfici ed integrali di superfici, [BDPG,15]}
\section{Lez - 17, Superfici in $\R^3$}
Intuitivamente una superficie nello spazio \ace un oggetto bidimensionale, senza spessore. \\
Prima delle definizione intriduciamo due esemi di superfici note. 
\begin{example}
  \begin{enumerate}
    \item Sia $D=\{(x,y): x^2+y^2 < 1\}$ e sia $f : \overline{D} \to \R$ la funzione 
          definita da $f(x,y) = x^2+y^2$. \\
          Il grafico di f, 
          $$S_1 = G_f = \{(x,y,f(x,y)): (x,y)\in\overline{D}\}$$
          (porzione di paraboloide) \\\\
          $S_1$ pu\aco essere vista come l'immagine della mappa (detta \underline{parametrizzazione})
          $$\sigma : \overline{D} \subseteq \R^2\to\R^3 , \, \sigma(x,y) = (x,y,x^2+y^2)$$
    \item La sfera di raggio 1 e centro (0,0,0) in $\R^3$ 1ace il sottoinsieme definito da 
          $$S_2 = \{(x,y,z): x^2+y^2+z^2=1\}$$
          \ac{E} noto che $S_2$ non pu\aco essere visto come il grafico di una funzione di due 
          variabili, ma pu\aco essere visto come immagine di una parametrizzazione. \\
          Per esempio, una parametrizzazione di $S_2$ pu\aco essere ottenuta tramite le 
          coordinate sferiche:
          $$\left\{\begin{array}{l}
            x = \cos\vartheta\sin\varphi \\
            y = \sin\vartheta\sin\varphi \\
            z = \cos\varphi \\
          \end{array}\right.$$
          dove $\vartheta \in [0,2\pi], \varphi \in [0,\pi]$. \\
          Pi\acu precisamente $S_2$ \ace l'immagine della mappa 
          $\sigma : [0,2\pi]\times[0,\pi] \subseteq \R^2 \to \R^3$
          $$\sigma(\vartheta,\varphi) := (\cos\vartheta\sin\varphi, \sin\vartheta\sin\varphi, \cos\varphi)$$
  \end{enumerate} 
\end{example}
Prima della definizione di superficie, premettiamo la nozione di curva di Jordan nel piano.
\begin{definition}
  Una \underline{curva di Jordan} \ace una curva piana $\gamma : [a,b]\to\R^2$ semplice e chiusa.
\end{definition}
\begin{theorem}[BDPG,p. 359]
  Sia $\g$ una curva di Jordan. Allora valgono le seguenti prop.
  \begin{enumerate}
    \item Il \underline{sostegno} $\Gamma = \g([a,b])$ divide il piano in due insiemi aperti
          di cui uno \ace limitato, e si chiama \underline{interno della curva} ($D_{int}$), e l'altro
          \ace illimitato, e si chiama \underline{esterno della curva} ($D_{est}$)
    \item $\Dint,\Dest$ hanno la stessa frontiera e coincide con $\Gamma$
  \end{enumerate}
\end{theorem}
\begin{definition}
  Un sottoinsieme $S\subset\R^3$ si dice \underline{superficie} (elementare) se esiste una mappa
  $\sigma:\overline{D}\subseteq\R^2\to\R^3$, 
  $$\sigma(u,v) = (x(u,v),y(u,v),z(u,v))$$
  verificante
  \begin{enumerate}
    \item D \ace un aperto di $\R^2$, interno di una curva di Jordan
    \item $\sigma$ \ace continua e $\sigma:D\to\R^3$ \ace iniettiva
    \item $\sigma(\overline{D})=S$
  \end{enumerate}
  Una funzione verificante (1.-3.) \ace detta parametrizzazione di S. \\\\
  S si dice \underline{superficie cartesiana} se esiste una parametrizzazione
  $\sigma : \overline{D}\subseteq\R^2\to\R^3$ del tipo
  $$\sigma(u,v) = \begin{array}{lr}
    (u,v,f(u,v)) & z = f(x,y) \\
    \text{oppure} \\
    (f(u,v),u,v) & x = f(y,z) \\
    \text{oppure} \\
    (u,f(u,v),v) & y = f(x,z) \\
  \end{array} (u,v)\in\overline{D}$$
  dove $f:\overline{D}\to\R$ continua.
\end{definition}
\begin{osservazione}
  \begin{enumerate}
    \item Si noti che, a differenza della nozione di curva, nella nozione di superficie \ace all'immagine
          della parametrizzazione che si assegna il nome "superficie" e non alla parametrizzazione
    \item Le superifici considerate sono \underline{limitate}. \\
          Per includere superfici illimitate si necessiterebbe nella definizione di un cambiamento 
          che non \ace tratto nel corso.
    \item Data una superficie $S\subseteq\R^3$, la parametrizzazione di S non \ace unica.
  \end{enumerate}
\end{osservazione}
\begin{exercise}
  Si provvi che la mappa 
  $\sigma^*(u,v) = (u\cos{v},u\sin{v}, u^2)$, $(u,v)\in[0,1]\times[0,2\pi]$
  \ace un'altra parametrizzazione della superficie $S_1$, con $D:= (0,1)\times(2\pi)$
\end{exercise}
\subsection{Punti interni e bordo di una superficie}
Vogliamo ora precisare la nozione di bordo e punti interni per una superficie 
$S\subseteq\R^3$ da un punto di vista intrinseco. \\
Osserivamo che la precisazione "da un punto di vista intrinseco" evidenzia la differenza
con le nozioni di frontiera e parte interna di S, visto come sottoinsieme di $\R^3$. \\
Infatti, si pu\aco provare che, data $S\subseteq\R^3$ superficie, allora 
$$\p S = S \text{ e } \overcirc{S} = \varnothing$$
D'altra parte \ace abbastanza intuitivo ritenere che, per esempio, il bordo (intrinseco) della 
porzione di paraboloide ($S_1$), sia la circonferenza 
$$\{(x,y,1):x^2+y^2=1\}$$
mentre il bordo (intrinseco) della sfera sia $\varnothing$. \\\\
Vogliamo introdurre due nozioni che formalizzino questa intuizione.
\begin{definition}
  Sia $S$ una superficie
  \begin{enumerate}
    \item Un puno $p\in S$ si dice \underline{interno a S} se esiste un 
          $B(p,r)$ ed una parametrizzazione $\sigma^* : \overline{D^*}\subseteq\R^2\to\R^3$
          di $\overline{B(p,r)\cap S}$ tale che $p\in \sigma^*(D^*)$. \\
          L'insime dei punti interni \ace denotato da: S'
    \item Si chiama \underline{bordo di S} e si denota $bor(S)$ l'insieme dei punti 
          che non sono interni ad S, cio\ace $bor(S) = S \setminus S'$
  \end{enumerate}
\end{definition}
\begin{example}
  \begin{enumerate}
    \item Si potrebbe provare che $S_1$ ha come punti interni l'insieme
          $$S_1' = \{(x,y,x^2+y^2):x^2+y^2<1\}$$
          mentre 
          $$bor(S_1) = \{(x,y,x^2+y^2):x^2+y^2=1\}$$
    \item La superficie $S_2$ ha come insieme dei punti interni tutti i punti, $S_2' = S_2$ 
          mentre $bor(S_2) = \varnothing$
  \end{enumerate}
\end{example}
\section{Regolarit\aca della parametrizzazione e piano tangente ad una superficie}
Possiamo intuire, tenendo presente il caso delle curve, che la regolarit\aca delle parametrizzazioni
di una superficie potrebbe non bastare per l'esistenza del piano tangente ad una superficie. \\
In effett, come vedremo, si possono costruire superifci che ammettono 
parametrizzazione di classe $C^1$ e non ammettono piano tangente in qualche punto.\\
Quindi necessiter\aca individuare, come nel caso delle curve, una condizione aggiuntiva alla regolarit\aca 
$C^1$ della parametrizzazione di una superficie, per l'esistenza del piano tangente. \\\\
Per capire quale sia questa condizione aggiuntiva, ci aiuteremo con un argomento geometrico
che utilizza la nozione di tangente di una curva. \\\\
Sia $S$ superficie, sia $\sigma:\overline{D}\to\R^3$ una sua parametrizzazione di classe $C^1$\\
L'esistenza di un piano tangente $\pi$ a S in un punto interno 
$p_0=\sigma(u_0,v_0)$ con $(u_0,v_0)\in D$ dovrebbe implicare la seguente propriet\aca:
se $\g :[a,b]\to D$ curva di classe $C^1$ con $\g'(t_0) \neq (0,0) \Rightarrow
\gtilde = \sigma\circ\g:[a,b]\to S$ \ace ancora di classe $C^1$ con 
$\gtilde'(t_0)\neq (0,0,0)$ e la retta tangente alla curva $\gtilde$, passante 
per $\gtilde(t_0)$, deve appartenere a $\pi$. \\
Siano $\g:[a,b]\to D$ di classe $C^1$ con $\g'(t_0)\neq (0,0)$ 
e $\g(t) = (u(t),v(t))$, $t\in[a,b]$, $\g(t_0)=(u_0,v_0)$, 
$$\sigma(u,v):= (x(u,v),y(u,v),z(u,v))$$ con $(u,v)\in\overline{D}$, 
$$\gtilde(t) := \sigma(\g(t)) = (x(\g(t)), y(\g(t)), z(\g(t)))$$ per RDC.
\begin{exercise}
  $(1) \, \g'(t_0) = u'(t_0)\sigma_u(u_0,v_0) + v'(t_0)\sigma_v(u_0,v_0)$
  dove 
  $$\sigma_u(u,v) := \left(\frac{\p x}{\p u}, \frac{\p y}{\p u}, \frac{\p z}{\p u}\right)(u,v)$$
  $$\sigma_v(u,v) := \left(\frac{\p x}{\p v}, \frac{\p y}{\p v}, \frac{\p z}{\p v}\right)(u,v)$$
  Vogliamo imporre che $\gtilde'(t_0)\neq(0,0,0)$. \\
  Essendo $\g'(t_0) \neq (0,0)$, allora da (1) segue 
  $$(2) \, \sigma_u(u_0,v_0) \text{ e } \sigma_v(u_0,v_0) \text{ sono L.I.}$$
  Ricordiamo ora che:
  $$(2) \iff \sigma_u(u_0,v_0)\wedge\sigma_v(u_0,v_0)\neq(0,0,0)$$
  dove dati $w=(w_1,w_2,w_3), z = (z_1,z_2,z_3) \in \R^3$
  $$w\wedge z = \det\begin{bmatrix}
    e_1 & e_2 & e_3 \\
    w_1 & w_2 & w_3 \\
    z_1 & z_2 & z_3 \\
  \end{bmatrix} = \left(w_2z_3 - z_2w_3, w_1z_3+z_1w_3, w_1z_2-z_1w_2\right) \in \R^3$$ 
  (prodotto vettore di w e z) \\
  Ricordiamo inoltre che valgono le seguenti propriet\aca
  \begin{itemize}
    \item $w\wedge z$ \ace ortogonale sia a w che a z;
    \item $\norma{w\wedge z} = \norma{w}\norma{z}\sin{\alpha}$;
    \item $w\wedge z = (0,0,0) \iff$ w e z sono paralleli;
  \end{itemize}
  Consideriamo ora il piano $\pi \subset \R^3$ definito da 
  $$\pi := \{\sigma(u_0,v_0) + \lambda \sigma_u(u_0,v_0) + \mu \sigma_v(u_0,v_0) : \lambda, \mu \in \R\}$$
  (eq. parametrica di un piano) \\\\
  Osserivamo che $\pi$ \ace il piano di eq.:
  $$a(x-x_0)+b(y-y_0)+c(z-z_0) = 0$$
  dove, $(a,b,c) := \sigma_u(u_0,v_0) \wedge \sigma_v(u_0,v_0) \neq (0,0,0)$, 
  $(x_0,y_0,z_0 ):= \sigma(u_0,v_0)$ \\
  Infatti, basta osservare che, se $(x,y,z)\in \pi$, allora $(x,y,z)$
  verifica (*), allora, per propriet\aca del prodotto vettore, esistono
  $\lambda,\mu \in \R$ t.c. 
  $$(x,y,z) = \sigma(u_0,v_0) + \lambda \sigma_u(u_0,v_0) + \mu \sigma_v(u_0,v_0) : \lambda, \mu \in \R$$
  Infine si osservi che la retta tangente alla curva $\gtilde$, passante pr 
  $\gtilde(t_0)$, di eq. parametrica 
  $$(x,y,z) = \sigma(u_0,v_0) + \gtilde'(t_0)(t-t_0) = 
      \sigma(u_0,v_0) + \left(u'(t_0)\sigma_u(u_0,v_0) + v'(t_0)\sigma_v(u_0,v_0)\right)(t-t_0)$$
  $$= \sigma(u_0,v_0) + u'(t_0)\sigma_u(u_0,v_0)(t-t_0) + v'(t_0)\sigma_v(u_0,v_0)(t-t_0) \in \pi$$
  $\forall t \in \R$. Dunquq essa \ace contenuta in $\pi$
\end{exercise}
\begin{definition}
  Sia S superfice e sia $p_0\in S'$
  \begin{enumerate}
    \item Il punto $p_0$ si dice \underline{regolare} se esistono 
          $B(p_0,r_0)$ ed una parametrizzazione $\sigma:\overline{D}\to\R^3$ di 
          $\overline{B(p_0,r_0)\cap S}$ t.c. 
          \begin{enumerate}
            \item[1.1] $\sigma$ \ace di classe $C^1$
            \item[1.2] vale (2) per il punto $(u_0,v_0)\in D$ t.c. $\sigma(u_0,v_0)=p_0$.\\
                        In tal caso il piano $\pi$ si chiama \underline{piano tangente}
                        a S nel punto $p_0$.\\
                        Le due direzioni (o versori)
                        $$\pm \frac{\sigma_u(u_0,v_0)\wedge\sigma_v(u_0,v_0)}
                                {\norma{\sigma_u(u_0,v_0)\wedge \sigma_v(u_0,v_0)}}$$
                        si chiamano \underline{direzioni} (o versori) normali a S in $p_0$
          \end{enumerate}
    \item S si dice \underline{regolare}, se tutti i punti interni di S sono regolari.
  \end{enumerate}
\end{definition}
\begin{osservazione}
  Sia $D\subseteq\R^2$ intorno di una curva di Jordan e sia $f\in C^0(\overline{D})\cap C^1(D)$. 
  Consideriamo la superficie cartesiana
  $$S=G_f = \{(x,y,f(x,y)) : (x,y) \in \overline{D}\}$$
  Allora 
  \begin{exercise}
    \begin{enumerate}
      \item ogni punto $p_0 = (x_0,y_0,f(x_0,y_0))$ con $(x_0,y_0)\in D$ \ace interno di S regolare.
      \item L'eq. del piano tangente a S nel punto $p_0$ \ace data da 
            $$z = f(x_0,y_0) + \frac{\p f}{\p x}(x_0,y_0) (x-x_0) + \frac{\p f}{\p y}(x_0,y_0)(y-y_0)$$
    \end{enumerate}
    \textbf{Soluzione:} Si consideri la parametrizzazione di S data da 
    $$\sigma(u,v) = (u,v,f(u,v))$$
    $(u,v)\in\overline{D}$
    \begin{exercise}
      Provare che
      $$\sigma_u(u_0,v_0) \wedge \sigma_v(u_0,v_0) = \left(-\p_u f(u,v), -\p_v f(u,v), 1\right) \, (u,v)\in D$$
    \end{exercise}
    Da ci\aco segue subito il punto uno. \\ 
    Inoltre ricordando l'eq cartesiana del piano tangente a S nel punto $p_0$ (vedi (*)), segue 
    subito il punto due. 
  \end{exercise}
\end{osservazione}
  \section{Lez - 18, Superfici orientabili}
Per introdurre il concetto di orientabilit\aca di una superficie
partiamo dal caso pi\acu semplice di una superficie cartesiana. \\
Per esempio, supponiamo che una superficie sia parametrizzata dalla mappa 
$$\sigma:\overline{D}\to\R^3 \, \sigma(u,v)=(u,v,f(u,v))$$
$D\subseteq \R^2$ interno di una curva di Jordan $f \in C^0(\overline{D})\cap C^1(D)$. \\
I due versori normali a S nel punto $p = \sigma(u,v)$, $(u,v)\in D$ sono date da 
$$\pm \frac{\sigma_u(u_0,v_0)\wedge\sigma_v(u_0,v_0)}
{\norma{\sigma_u(u_0,v_0)\wedge \sigma_v(u_0,v_0)}} = \pm \frac{\left(-\p_u f(u,v), -\p_v f(u,v), 1\right)}
    {\sqrt{1+\abs{\nabla f(u,v)}^2}}$$
Si noti che la terza componente del vettore 
$$\pm \frac{1}{\sqrt{1+\abs{\nabla f(u,v)}^2}} \neq 0$$
e $S' = \sigma(D)$. \\
Pertanto \ace sempre possibile selezionare, in ciascun punto, il versore normale che 
"punti verso l'alto" e lo denotiamo con 
$$N_S^{+} (x,y,z) = \frac{\sigma_u\wedge \sigma_v}{\norma{\sigma_u\wedge\sigma_v}}\left(\sigma^{-1}(x,y,z)\right)$$
$(x,y,z)\in S'$, dove $\sigma^{-1}:S'\to D$, $\sigma'(x,y,z) = (x,y)$, mentre 
denotiamo 
$$N_S^{-} (x,y,z) = - N_S^{+} (x,y,z)$$
Si noti che $N_S^{+}, N_S^{-}: S' \to \R^3$ sono continue. \\
Si dice in questo caso che sono possibili due \underline{orientazioni} della superficie S, indotte
dalla parametrizzazione. \\
Pi\acu in generale vale la seguente
\begin{definition}
  Una superficie $S\subseteq\R^3$ si dice \underline{orientabile} se esiste una mappa 
  $N_S^{+} : S'\to\R^3$ t.c. 
  \begin{enumerate}
    \item $N_S^{+}(p)$ coincide con uno dei due versori normali definiti tramite parametrizzazione;
    \item $N_S^{+}$ \ace continua.  
  \end{enumerate}
  $N_S^{+}$ \ace detto \underline{versore normale positivo} a S. \\
  Definiamo $N_S^{-} = - N_S^{+}$
\end{definition} 
\begin{example}
  \begin{itemize}
    \item (Porzione di paraboloide) \\
          Sia $S=\{(x,y,x^2+y^2): x^2+y^2\leq r^2\}$ \\
          Sappiamo che l'insieme dei punti interni \ace dato da 
          $$S':=\{(x,y,x^2+y^2):x^2+y^2< r^2\}$$
          $$\overline{D} := \{(u,v)\in\R^2 : u^2+v^2\leq r^2\}$$
          $\sigma:\overline{D}\to\R^3$, $\sigma(u,v):= (u,v,u^2+v^2)$\\
          $\sigma^{-1}:S' \to D$, $\sigma^{-1}(x,y,z)=(x,y)$. \\
          Definiamo la mappa $N_S^{+}$,
          $$N_S^{+}(x,y,z) = \frac{\sigma_u\wedge\sigma_v}{\norma{\sigma_u\wedge\sigma_v}}\left(\sigma^{-1}(x,y,z)\right)
          = \frac{(-2x,-2y,1)}{\sqrt{4x^2+4y^2+1}}$$
          se $(x,y,z)\in S'$. \\
          \ac{E} facile verificare che $N_S^{+}$ \ace un'orientazione di S. \\
          L'altra orientazione \ace data da $N_S^{-} = -N_S^{+}$. 
          In questo caso si dice cche l'orientazione positiva \ace indotta dalla parametrizzazione $\sigma$
    \item (Sfera di $\R^3$) \\
          $$S = \{(x,y,z): x^2+y^2+z^2=r^2\}$$
  \end{itemize}
\end{example}
\section{Integrali di superficie}
Incominciamo a definire la nozione di area di una superficie.
\subsection{Idea per definire l'area di una superficie}
Sia $S\subseteq\R^3$ una superficie regolare e sia 
$\sigma:\overline{D}\subseteq \R^2\to\R^3$ una sua parametrizzazione. \\
Sia $Q = [u_0+u_0d_u]\times[v_0,v_0+dv]$ cpn $(u_0,v_0)\in D$, l'elemento 
infinitesimo di area $dS$ \ace dato da 
$$(1)\, dS = area(\sigma(Q))$$
se $du$ e $dv$ sono quantit\aca positive "molto piccole". \\
Ricordiamo che essendo S regolare, ammette piano tangente nel punto $p_0 = \sigma(u_0,v_0)$ 
definito da 
$$\pi = \{\sigma(u_0,v_0) + \lambda\sigma_u(u_0,v_0) + \mu\sigma_v(u_0,v_0):\lambda,\mu\in\R\}$$
Essendo $\sigma$ di classe $C^1$ \ace differenziabile in $p_0$, cio\ace 
$$(2) \, \sigma(u,v) = \sigma(u_0,v_0)+d\sigma(u_0,,v_0)(u-u_0,v-v_0)+ o\left(\norma{(u-u_0,v-v_0)}\right) $$
$\forall (u,v)\in D$, dove per definizione 
$d\sigma(u_0,v_0) : \R^2\to\R^3$ e 
$$d\sigma(u_0,v_0)(\lambda,\mu) := D\sigma(u_0,v_0)\begin{bmatrix}
  \lambda \\
  \mu \\
\end{bmatrix} \, (\lambda,\mu)\in\R^2$$
$$D\sigma(u_0,v_0):= \begin{bmatrix}
  \frac{\p x}{\p u} & \frac{\p x}{\p v} \\
  \\
  \frac{\p y}{\p u} & \frac{\p y}{\p v} \\
  \\
  \frac{\p z}{\p u} & \frac{\p z}{\p v} \\
\end{bmatrix}(u_0,v_0) \text{ matrice Jacobiana di } \sigma \text{ in } (u_0,v_0)$$
$$D\sigma(u_0,v_0) = \begin{bmatrix}
  \sigma_u(u_0,v_0) & | & \sigma_v(u_0,v_0) \\
\end{bmatrix}$$
Se du e dv sono molto piccoli, per l'approssimazione (2), possiamo considerare nullo 
$o\left(\norma{(u-u_0,v-v_0)}\right)$. \\
Perci\aco $$(3) \, area(\sigma(Q)) \simeq area(\widetilde{Q})$$
dove 
$$\widetilde{Q} = \T(Q) = \{\sigma(u_0,v_0) + \lambda\sigma_u(u_0,v_0) + \mu\sigma_v(u_0,v_0): \lambda\in[0,du],
  \mu \in [0,dv]\} \subseteq \pi$$
(parallelogramma determinato dal vertice $\sigma(u_0,v_0)$ e dai vettori 
 $du\cdot \sigma_u(u_0,v_0)$ e $dv\cdot \sigma_v(u_0,v_0)$) e
$$\T (\lambda,\mu):= \sigma(u_0,v_0)+d\sigma(u_0,v_0)(\lambda,\mu) \, (\lambda,\mu)\in\R^2$$
Denotiamo 
$$w = du\cdot\sigma_u(u_0,v_0) \text{ e } z = dv\cdot\sigma_v(u_0,v_0)$$
Allora 
$$(4)\, area(\widetilde{Q}) = \norma{w}\norma{z} \sin\alpha := \norma{w\wedge z}
= \norma{\sigma_u(u_0,v_0)\wedge \sigma_v(u_0,v_0)}\,du\,dv$$
dove $\alpha =$ angolo tra w e z. \\
Pertanto da (1), (3), (4) otteniamo 
$$dS \simeq \norma{\sigma_u(u_0,v_0) \wedge \sigma_v(u_0,v_0)}\,du\,dv$$
\begin{definition}
  Sia S una superficie regolare di parametrizzazione $\sigma : D\subseteq \R^2 \to \R^3$
  insieme misurabile e supponiamo che la funzione 
  $$(*) D \ni (u,v) \to \norma{\sigma_u\wedge\sigma_v}(u,v)$$
  sia limitata. \\
  Si chiama \underline{area di S} il valore 
  $$A(S) := \iint_D \norma{\sigma_u(u,v)\wedge\sigma_v(u,v)}\,du\,dv$$
  Una superficie S regolare per cui valga $(*)$ si dice \underline{di area ben definita}
\end{definition}
\begin{example}
  \begin{enumerate}
    \item Calcolare l'area della sfera di centro (0,0,0) e raggio $r>0$ \\
          \textbf{Soluzione:} \\\\
          Possiamo rappresentare 
          $$S = \{(x,y,z)\in\R^3:x^2+y^2+z^2 = r^2\}$$
          e consideriammo la sua parametrizzazione in coordinate sferiche, cio\ace 
          la mappa $\sigma:\overline{D}\to\R^3$, $\overline{D}=[0,2\pi]\times[0,\pi]$, 
          $$\sigma(u,v) = r\left(\cos{u}\sin{v}, \sin{u}\sin{v}, \\cos{v}\right)$$
          \begin{exercise}
            Verificare $$\norma{\sigma_u\wedge\sigma_v} = r^2\abs{\sin{v}}$$
          \end{exercise}
          Pertanto 
          $$A(S) = \iint_{D} \norma{\sigma_u\wedge\sigma_v} (u,v) \,du\,dv = \iint_{D} r^2\abs{\sin{v}} \,du\,dv = $$
          $$= r^2\left(\int_{0}^{2\pi} \,du\right)\cdot\left(\int_{0}^{\pi}\sin{v} \,dv\right) = 
              r^2\cdot 2\pi \left(\left.-cos{v}\right|_{0}^{\pi}\right) = 4\pi r^2$$
    \item Sia $D\subseteq\R^2$ interno di una curva di Jordan, e sia 
          $f\in C^0(\overline{D})\cap C^1(D)$ e supponiamo che 
          $\p_u f, \p_v f : D \to \R$ siano limitate. \\
          Allora se $S = G_f := \{(u,v,f(u,v)):(u,v)\in D\}$, 
          $$A(S) = \iint_{D} \sqrt{1+\abs{\nabla f(u,v)}^2} \,du \,dv$$
          \textbf{Soluzione:} \\\\
          Consideriamo la parametrizzazione cartesiana $\sigma : \overline{D}\to\R^3$ di S definita come
          $$\sigma(u,v)=(u,v,f(u,v)) \, (u,v)\in\overline{D}$$
          Sappiamo che 
          $$\sigma_u \wedge \sigma_v = \left(-\p_u f, -\p_v f, 1\right)$$
          se $(u,v)\in D$. \\
          Pertanto 
          $$A(S) = \iint_{D}\norma{\sigma_u\wedge\sigma_v}(u,v)\,du\,dv = 
              \iint_{D} \sqrt{1+\p_u f^2+\p_v f^2} \,du\,dv = $$
          $$= \iint_{D} \sqrt{1+\abs{\nabla f(u,v)}^2} \,du\,dv$$
  \end{enumerate}
\end{example}
\begin{exercise}
  Siano $\g = (\g_1,\g_2):[a,b]\to\R^2$ una curva regolare, 
  $f:\Gamma = \g([a,b])\to [0,+\infty)$ continua, e sia
  $$S=\{\left(\g_1(u),\g_2(u),v\right): a \leq u\leq b, 0 \leq v \leq f(\g(u))\}$$
  (sottografico di f lungo $\Gamma$). \\
  Provare che S \ace una superficie regolare e $A(S) = \int_{\g} f \,ds$. \\
  \textbf{Soluzione:} \\\\
  Consideriamo la parametrizzazione $\sigma:\overline{D}\to\R^3$, $\sigma(u,v)=\left(\g_1(u),\g_2(u),v\right)$
  dove 
  $$\overline{D} = \{(u,v)\in\R^2: u \in [a,b], v \in [0,f(\g(u))]\}$$
  Allora \begin{itemize}
    \item $\sigma_u = \left(\g_1'(u),\g_2'(u), 0\right)$
    \item $\sigma_v = (0,0,1)$
  \end{itemize}
  se $(u,v)\in D$ e 
  \begin{exercise}
    Verificare:
    \begin{enumerate}
      \item D \ace interno di una curva di Jordan
      \item $\sigma_u\wedge\sigma_v= \left(\g_2'(u),-\g_1'(u),0\right)$ se $(u,v)\in D$
      \item $\sigma(\overline{D}) = S$
    \end{enumerate}
  \end{exercise}
  In particolare, essendo $\g$ una curva regolare, dal punto (2) segue che S \ace regolare.
  Inoltre 
  $$A(S) = \iint_{D} \norma{\sigma_u\wedge\sigma_v}(u,v) \,du\,dv = 
          \iint_{D} \sqrt{\g_1'(u)^2 + \g_2'(u)^2} \,du\,dv$$
  Essendo D un insieme semplice rispetto a v, per la formula di riduzione su domini semplici,
  otteniamo:
  $$A(S) = \iint_{D}\norma{\g'(u)}\,du\,dv = \int_{a}^{b}\norma{\g'(u)}\left(\int_{0}^{f(\g(u))} \,dv\right)\,du = $$
  $$= \int_{a}^{b} f(\g(u))\norma{\g'(u)}\,du =: \int_{\g} f \, ds$$
\end{exercise}
\subsection{Generalizzazione di nozione di integrale di I sp. per curve a superfici}
Si pu\aco generalizzare la nozione di integrale di I specie per curve alle superfici.
\begin{definition}
  Sia $S$ una superficie regolare di parametrizzazione $\sigma:\overline{D}\to\R^3$ t.c. 
  \begin{enumerate}
    \item $D\subseteq \R^2$ misurabile
    \item $D\ni (u,v) \to \norma{\sigma_u\wedge\sigma_v}(u,v)$ sia limitata
  \end{enumerate}
  Sia $f:S' \to \R$ continua e limitata. Il valore
  $$\iint_{S} f \,dS := \iint_{D} f(\sigma(u,v))\norma{\sigma_u\wedge\sigma_v}(u,v) \,du\,dv$$
  si chiama \underline{integrale di superficie} di f.
\end{definition}

  \chapter{Il teorema della divergenza nel piano e nello spazio, [BDPG,16]}
\section{Lez - 19}
Ricordiamo che se $v\in E\subseteq\R^n\to\R^n$ \ace un campo vettoriale di classe $C^1(E)$
su un aperto E, l'operatore
$$v(x) = \left(v_1(x),..,v_n(x)\right) \to div(v)(x) = \sum_{i=1}^{n} \frac{\p v_i}{\p x_i} (x) \in \R$$
se $x\in E$, \underline{operatore di divergenza}
\begin{definition}
  Sia $E\subseteq\R^n$ chiuso, una funzione $f:E\to\R$ si dice di classe $C^1(E)$ se esiste 
  un aperto $E^*\supset E$ ed una funzione $f^* : E^* \to \R$ di classe $C^1(E^*)$ t.c.
  $$f^*(x) = f(x) \, \forall x\in E$$
  Pi\acu in generale, se $v = (v_1,...,v_n) : E \subseteq \R^n \to \R^n$ campo vettoriale
  si dice di classe $C^1(E)$ se ogni componente
  $v_i : E \to \R$ \ace di classe $C^1(E)$
\end{definition}
\begin{osservazione}
  Tipicamente utilizzeremo questa nozione nel caso in cui $E = \overline{A} \supset A$ con 
  $A\subseteq \R^n$ aperto.
\end{osservazione}
\begin{definition}
  \begin{enumerate}
    \item Un dominio E semplice rispetto a y si dice \underline{regolare a tratti} se 
          $$E = \{(x,y)\in\R^2 : x\in[\alpha,\beta], g_1(x)\leq y \leq g_2(x)\}$$
          con $g_1(x) < g_2(x)$ $\forall x \in [\alpha,\beta]$, $g_1,g_2\in C^1([\alpha,\beta])$
    \item Un dominio $E\subseteq \R^2$ semplice rispetto a x si dice regolare a tratti se 
          $$E = \{(x,y)\in\R^2 : y\in[\alpha,\beta], g_1(y)\leq x \leq g_2(y)\}$$
          con $g_1(y) < g_2(y)$ $\forall y \in [\alpha,\beta]$, $g_1,g_2\in C^1([\alpha,\beta])$
  \end{enumerate}
\end{definition}
Assumiamo, per esempio, che $E\subseteq\R^2$ sia un dominio y-semplice come in (1). Allora si pu\aco provare che 
$\p E$ \ace il sostegno di una curva semplice chiusa e ragolare a tratti 
$\g : [\alpha,\beta+3] \to \R^2$ definita come 
$$\g = \bigcup_{i=1}^{4} \g_i$$
dove 
\begin{itemize}
  \item $\g_1(t)=(t,g_1(t))$, $t\in[\alpha,\beta]$
  \item $\g_2(t)=(\beta,g_1(\beta)+(g_2(\beta)-g_1(\beta))(t-\beta))$, $t\in[\beta, \beta+1]$
  \item $\g_3(t)=(\beta+(\beta-\al)(\beta+1-t),g_2(\beta+(\beta-\al)(\beta+1-t)))$, $t\in[\beta+1,\beta+2]$
  \item $\g_4(t)=(\al, g_2(\alpha)+ (g_2(\alpha)-g_1(\alpha))(\beta+2-t))$, $t\in[\beta+2,\beta+3]$
\end{itemize}
Si noti che $\g$ \ace una curva di Jordan, che induce un orientamento positivo su $\p E$. \\
Infatti una curva di Jordan si dice che induca un orientamento positivo su $\p E$, frontiera
del suo interno, quando $\p E$ \ace prcorso in senso anti-orario. In questo caso 
l'insieme E \ace tenuto a sinistra quanto $\p E$ \ace percorso. \\
Con questa convenzione, se $\om$ \ace una forma differenziale continua su $\p E$ e 
$f\in C^0(\p E)$ si pone, per definizione, 
$$\int_{\p^+ E} \om = \int_{\g} \om , \int_{\p E} f \, ds = \int_{\g} f \, ds$$
Vale allora il seguente fondamentale risultato
\section{Teorema Gauss-Green}
\begin{theorem}[formule di Gauss-Green per domini semplici] 
  \label{gaussgreen}
  Sia $E\subseteq \R^2$ un dominio semplice, regolare a tratti e sia $f\in C^1(E)$. 
  Allora 
  $$(GG1) \, \iint_{E} \frac{\p f}{\p x} \,dx\,dy = \int_{\p^+ E} f \,dy$$
  $$(GG2) \, \iint_{E} \frac{\p f}{\p y} \,dx\,dy = -\int_{\p^+ E} f \,dx$$
\end{theorem}
\begin{osservazione}
  Le formule (GG1) e (GG2) collegano un integrale doppio su un insieme $E\subseteq\R^2$ 
  ad un integrale curvilineo sulla frontiera $\p E$. Quindi si "abbassa" la
  dimensione sull'insieme su cui si integra da 2 (insieme E) a 1 (insieme $\p E$). \\
  In questo senso questo risultato pu\aco essere considerato l'analogo in $\R^2$ del 
  teorema fondamentale del calcolo dell'integrale dove si afferma che se 
  $v \in C^1([a,b])$, $$\int_{a}^{b} \frac{dv}{dx}(x)\,dx = v(b)-v(a)$$
  Quindi l'integrale di una funzione su un intervallo (dim = 1) al valore della sua primitiva
  in due punti (dim = 0).
\end{osservazione}
\begin{exercise}
  \label{es41}
  Siano $f\in C^1(\R^2)$, $a,b\in C^1(\R)$ e $F:\R\to\R$ definita da 
  $$F(x) = \int_{a(x)}^{b(x)}f(x,y)\,dy$$
  Supponendo noto che $\forall \al, \bb \in \R$
  $$\frac{d}{dx}\int_{\al}^{\bb} f(x,y)\,dy = \int_{\al}^{\bb} \frac{\p f}{\p x} (x,y) \, dy$$
  provare che 
  $$\exists F'(x) = f(x,b(x))b'(x) - f(x,a(x))a'(x) + \int_{a(x)}^{b(x)} \frac{\p f}{\p x}(x,y)\,dy$$
  \textbf{Soluzione:} \\\\
  Si osservi che $F(x) = G(H(x))$, $x\in\R$, dove $G:\R^3\to\R$, $H:\R\to\R^3$ definite come
  $$G(x,s,t):=\int_{s}^{t} f(x,y)\,dy , H(x) = (x,a(x),b(x))$$
  Per RDC, 
  $$\exists F'(x) = \nabla G(H(x))\cdot H'(x) \, \forall x \in \R$$
  D'altra parte poich\ace
  $$\nabla G(x,s,t) = \left(\frac{\p}{\p x} \int_{s}^{t} f(x,y)\,dy, -f(x,s), f(x,t)\right) = $$
  $$= \left(\int_{s}^{t} \frac{\p}{\p x}f(x,y)\,dy, -f(x,s), f(x,t)\right)$$
  e $$H'(x) = (1,a'(x),b'(x))$$
  segue la tesi.
\end{exercise}
  \section{Lez - 20, Dimostrazione Gauss-Green}
Dimostriamo ora il teorema di Gauss-Green, \ref{gaussgreen},
\begin{proof}
  Supponimao che E sia semplice rispetto a y e sia rappresentato con le 
  notazioni precedenti. \\
  \begin{itemize}
    \item Incominciamo a provare (GG2). \\
          Dall'esercizio 5 foglio 10, segue:
          $$(1)\, \int_{\p^+ E} f \, dx = \int_{\al}^{\bb}\left(f(x,g_1(x))-f(x,g_2(x))\right)\,dx$$
          Per la formula di riduzione degli integrali doppi su domini semplici, 
          $$(2)\, \iint_{E}\frac{\p f}{\p y} \,dx\,dy = \int_{\al}^{\bb}\left(\int_{g_1(x)}^{g_2(x)} \frac{\p f}{\p y} \,dy\right)\,dx = $$
          $$= \int_{\al}^{\bb} \left(f(x,g_2(x))-f(x,g_1(x))\right)\,dx$$
          Da (1) e (2) segue (GG2)
    \item Proviamo ora (GG1). \\
          Definiamo $F:[\al,\bb]\to\R$ come $$F(x) = \int_{g_1(x)}^{g_2(x)} f(x,y)\,dy$$
          Per l'esercizio precedente (\ref{es41}) 
          $$F'(x) = f(x,g_2(x))g_2'(x) - f(x,g_1(x))g_1'(x) + \int_{g_1(x)}^{g_2(x)} \frac{\p f}{\p x}(x,y)\,dy$$
          $\forall x \in [\al,\bb]$. \\
          Integrando rispetto a x la precedente identit\aca, otteniamo
          $$F(\bb)-F(\al) = \int_{\al}^{\bb} F'(x) \,dx = \int_{\al}^{\bb} = $$
          $$= \int_{\al}^{\bb} f(x,g_2(x))g_2'(x) - 
                \int_{\al}^{\bb} f(x,g_1(x))g_1'(x) + \int_{\al}^{\bb} \left(\int_{g_1(x)}^{g_2(x)} \frac{\p f}{\p x}(x,y)\,dy\right)\,dx = $$
          $$ = \iint_{E} \frac{\p f}{\p x}\,dx\,dy$$
          Possiamo riscrivere la precedente identit\aca come 
          $$(3)\, \iint_{E} \frac{\p f}{\p x} \,dx\,dy = \int_{\al}^{\bb} f(x,g_1(x))g_1'(x) \,dx + 
            \int_{g_1(\bb)}^{g_2(\bb)} f(\bb,y)\,dy - $$
            $$ - \int_{\al}^{\bb} f(x,g_2(x))\,dx - \int_{g_1(\al)}^{g_2(\al)} f(\al,y)\,dy$$
          D'altra parte, utilizzando la definizione di integrale curvilineo di 
          II specie di $\om = f\,dy$ lungo $\g$ si ottiene:
          \begin{exercise}
            $$(4)\, \int_{\p^+ E} \om = \int_{\g} \om = \sum_{i=1}^{4} \int_{\g_i}\om =  $$
            $$= \int_{\al}^{\bb} f(x,g_1(x))g_1'(x) \,dx + 
            \int_{g_1(\bb)}^{g_2(\bb)} f(\bb,y)\,dy - $$
            $$ - \int_{\al}^{\bb} f(x,g_2(x))\,dx - \int_{g_1(\al)}^{g_2(\al)} f(\al,y)\,dy$$
          \end{exercise}
          Da (3) e (4) segue (GG1)
  \end{itemize}
\end{proof}
Una conseguenza importante delle formule di \ref{gaussgreen} \ace il teorema della divergenza. 
Prima per\aco enunciamo la nozione di versore normale esterno ad un dominio semplice 
regolare a tratti. 
\subsection{Versore normale esterno ad un insieme semplice regoalre a tratti nel piano}
Assumiamo che $E\subseteq\R^2$ si un insieme y-semplice regolare a tratti ed utilizziamo 
ancora le nozioni precedenti.\\
Ricordiamo che $\p E = $ sostegno di $\g$, dove $\g$ \ace chiusa, semplice, regolare 
a tratti, $\g = \cup_{1}^{4} \g_i$ e $\g$ \ace percorsa in senso anti-orario. 
Pi\acu precisamente 
$$\g:[\al,\bb+3] \to \p E$$
e, se $\al = t_0 < t_1 = \bb < t_2 = \bb + 1 < t_3 = \bb+2 < t_4 = \bb +3$,
$\g_i : [t_{i-1},t_i]\to \p E$ regolare e 
$$\g_i(t) = (x_i(t),y_(t)) \, i = 1,2,3,4$$
Definiamo, per ogni punto $p\in \p E$, con $p \neq \g(t_{i-1})$, $\g(t_i)$
(i = 1,2,3,4) i versori 
$$\T^+(p) \frac{(x_i'(t),y_i'(t))}{\norma{\g'(t)}}$$
se $p = \g(t)$, $t \in (t_{i-1},t_i)$ e 
$$Ne(p) := \frac{(y_i'(t), -x_i'(t))}{\norma{\g'(t)}}$$
se $p = \g(t)$, $t\in(t_{i-1},t_i)$. \\
I versori $\T^+(p)$ e $Ne(p)$ sono detti, risp., \underline{versore tangente positivo} e  \\
\underline{versore normale esterno a $\p E$}. \\
\ac{E} facile verificare che $\T^+(p)$ e $Ne(p)$ sono ortogonali e si potrebbe 
provare che $Ne(p)$ punta vero l'esterno di E. 
\begin{theorem}[divergenza per domini semplici]
  Sia $E\subseteq\R^n$ un dominio semplice, regolare a tratti e sia $v=(v_1,v_2):E\subseteq\R^2\to\R^2$
  di classe $C^1(E)$. Allora 
  $$(GG)\, \iint_{E} div(u)\,dx\,dy = \int_{\p E} \tuple{v,Ne}\,ds = \int_{\p^+ E} (v_1\,dy-v_2\,dx)$$
  \begin{proof}
    Proviamo (GG) nel caso in cui E sia y-semplice, regolare a tratti ed utilizziamo le notazioni precedenti, con cui abbiamo rappresentato E.
    $$\int_{\p E} \tuple{v,Ne} \,ds := \int_{\g} \tuple{v,Ne} \,ds = \sum_{i=1}^4 \int_{\g_i} \tuple{v,Ne}\,ds = $$
    $$= \sum_{i=1}^4 \int_{t_{i-1}}^{t_i} \tuple{v(\g_i(t)),Ne(\g_i(t))}\norma{\g'(t)}\,dt = $$
    $$= \sum_{i=1}^4 \int_{t_{i-1}}^{t_i} \left(v_1(\g_i(t))y_i'(t) - v_2(\g_i(t))x_i'(t)\right)\,dt = $$
    $$= \sum_{i=1}^4 \int_{t_{i-1}}^{t_i} v_1(\g_i(t))y_i'(t) - \sum_{i=1}^4 \int_{t_{i-1}}^{t_i} v_2(\g_i(t))x_i'(t)\,dt = $$
    $$= \int_{\p^+ E} v_1\,dy - \int_{\p^+ E} v_2\,dx =_{(GG1+GG2)} = 
        \iint_{E} \frac{\p v_1}{\p x}\,dx\,dy + \iint_{E} \frac{\p v_2}{\p y}\,dx\,dy = $$
    $$= \iint_{E} \left(\frac{\p v_1}{\p x}+\frac{\p v_2}{\p y}\right)\,dx1\,dy = \iint_{E} div(v) \,dx\,dy$$
  \end{proof}
\end{theorem}
\subsection{Teorema della divergenza per insiemi generali del piano}
Il teorema della divergenza vale per insiemi del piano molto generali
\begin{definition}
  Dato $A\subseteq\R^n$ si dice \underline{convesso} se, $\forall p,q \in A$, essite una curva 
  $C^1$ a tratti $\g:[a,b]\to A$ t.c. $\g(a)=p$ e $\g(b) = q$
\end{definition}
\begin{osservazione}
  Un insieme convesso \ace connesso, mentre il viceversa pu\aco non valere. Infatti, se A \ace convesso
  presi $p,q\in A$ se definiamo la curva $C^1$ $\g:[0,1]\to A$, definita come 
  $\g(t) = tq+(1-t)p$ \ace la curva cercata.\\
  Invece se, n=2, e 
  $$A=\{(x,y):0<x^2+y^2<1\}$$
  abbiamo vissto che A \ace{non \ace convesso}. D'altra parte \ace facile convincersi che A \ace connesso.
\end{osservazione}
\begin{definition}
  Un insieme E si dice \underline{dominio reolare a tratti} se 
  \begin{enumerate}
    \item $E=\overline{A}$, con A aperto, connesso e limitato
    \item E \ace misurabile
    \item $\p E$ \ace l'unione disgiunta del sostegno di k curve di Jordan, $C^1$ a tratti, orientate in 
          modo tale da percorrere $\p E$ tenendo a sinistra E.
  \end{enumerate}
\end{definition}
\begin{example}
  Sia $E=\{(x,y): 1\leq x^2+y^2\leq 4\}$. \\
  Allora E \ace n dominio regolare a tratti. \\ 
  Infatti $E=\overline{A}$ dove $A = \{(x,y): 1 \leq x^2+y^2 \leq 4\}$, con A aperto limitato e connesso. 
  Inoltre E \ace misurabile e $\p E = \Gamma_1 \cup \Gamma_2$, dove 
  \begin{itemize}
    \item $\Gamma_1 = \{(x,y):x^2+y^2=1\} = \g_1([0,2\pi])$
    \item $\Gamma_2 = \{(x,y):x^2+y^2=4\} = \g_2([0,2\pi])$
  \end{itemize}
  dove 
  \begin{itemize}
    \item $\g_1 : [0,2\pi]\to \p E$, $\g_1(t) = (cost, -sint)$
    \item $\g_2 : [0,2\pi]\to \p E$, $\g_2(t) = (2cost, 2sint)$
  \end{itemize}
  Dato E dominio regolare a tratti, data $\om$ forma diff. di classe $C^0$ su $\p E$, 
  data $f:\p E \to \R$ di classe $C^0$ su $\p E$, definiamo 
  $$\int_{\p^+ E} \om = \sum_{i=1}^k \int_{\g_i}\om$$
  e 
  $$\int_{\p E} f\,ds = \sum_{i=1}^k \int_{\g_i}f \,ds $$
\end{example}
\subsection{Versore normale esterno ad un dominio regolare a tratti del piano e 
teo. della divergenza}
Sia E un dominio regolare a tratti. \\
Ricordiamo che 
$\p E$ = unione disgiunta dei sostegni di k curve $\g_1,...,\g_k$ di Jordan regolari 
a tratti, orientate in modo tale da percorrere $\p E$ tenendo a sinistra E.\\
Definiamo in ogni punto di $\p E$, eccetto al pi\acu un numero finito di punti, il 
\underline{versore tangente} positivo in un punto $p\in\p E$ nel modo seguente:\\
se $\g_i(t) = (x_i(t),y_i(t)) = p$, 
$$\T^+(p)=\frac{(x_i'(t), y_i'(t))}{\norma{\g_i'(t)}}$$
Si definisce \underline{versore normale esterno} in un punto $p\in\p E$ il versore
$$Ne(p) = \frac{(y_i'(t), -x_i'(t))}{\norma{\g_i'(t)}}$$
se $p=\g_i(t)$
\begin{theorem}[della divergenza nel piano][BDPG,16.5]
  Sia $E\subseteq\R^2$ un dominio regolare a tratti e sia $v:E\to\R^2$ 
  di classe $C^1(E)$. Allora
  $$(GG)\, \iint_{E}div(v)\,dx\,dy = \int_{\p E}\tuple{v,Ne}\,ds = \int_{\p^+ E} v_1\,dy-v_2\,dx$$
  La quantit\aca $$\int_{\p E}\tuple{v,Ne}\,dS$$
  rappresenta il flusso del campo v ... dell'insieme E.
\end{theorem}
Il teorema della divergenza ha fondamentali applicazioni fisiche/ingegneristiche.

  %\chapter{Esercitazioni}
\section{Lezione 1 - 09/03/2022}
\begin{eexercise}
  Determinare e disegnare nel piano xy il dominio delle seguenti funzioni, $\f$, dove A: dominio che dobbiamo determinare.
  $$f(x,y) = \log(4(x^2+y^2)-1)$$
  \sol $$4(x^2+y^2)-1 > 0 \iff x^2+y^2 > \frac{1}{4}$$
  Studiamo quindi: $x^2+y^2 = \frac{1}{4}$ la circonferenza di centro $c=(0,0)$ e raggio $r = \frac{1}{2}$,
  $$A = \{(x,y)\in \R^2 \mid x^2+y^2 > \frac{1}{4}\} = \R^2 \setminus \overline{B((0,0), \frac{1}{2})}$$
  dove:
  \begin{itemize}
    \item $\overline{B((0,0), \frac{1}{2})} = \{(x,y)\in\R^2 \mid \sqrt{x^2+y^2} \leq \frac{1}{2}\}$
    \item $B((0,0), \frac{1}{2}) = \{(x,y)\in\R^2 \mid \sqrt{x^2+y^2} < \frac{1}{2}\}$
  \end{itemize}
  \hfill\break 
  \textbf{Insiemi aperti e chiusi}\\
  $A = \{(x,y)\in \R^2\mid xy\geq 0\}$, A \ace chiuso $\iff A^c$ \ace aperto.\\
  Definiamo $\bar{A} = A$, $xy \geq 0 \iff \left\{\begin{array}{c}
    x \geq 0 \\
    y \geq 0 \\
  \end{array}\right. \vee \left\{\begin{array}{c}
    x \leq 0 \\
    y \leq 0 \\
  \end{array}\right.$
  Disegnando gli assi: \\
  $A^c = \R^2 \setminus A$ \ace aperto. Fisso ora $(x_0,y_0) \in A^c$, $r = d(\partial A, (x_0,y_0)) = \min{\lvert x_0 \rvert, \lvert y_0 \rvert}$. 
  La palla $B((x_0,y_0), \frac{r}{2}) \subset A^c \Rightarrow A^c $ \ace aperto $\Rightarrow A $ \ace chiuso.
\end{eexercise}
\begin{eexercise}
  $f(x,y) = \sqrt{y^2-x^4}$, $y^2 \geq x^4$. $$A=\{(x,y)\in\R^2 \mid y^2\geq x^4\}$$
  Proviamo a scrivere $y^2-x^4$ come 
  $$y^2-x^4 = (y-x^2)(y+x^2) \geq 0$$
  Due casi:
  \begin{itemize}
    \item $y \geq x^2$
    \item $y \geq -x^2$
  \end{itemize} 
  (Dal grafico otteniamo)
  $$A = \{(x,y)\in\R^2 \mid y \geq x^2 \vee y \leq -x^2\} = \{(x,y)\in\R^2 \mid y \geq x^2\} \cup \{(x,y)\in\R^2 \mid y \leq -x^2\}$$
\end{eexercise}
\begin{eexercise}
  Disegnare l'insieme di livello delle seguenti funzioni
  $$C_t = \{(x,y\in\R^2 \mid f(x,y) = t)\}$$
  con $t \in \R$.\\
  $f(x,y) = x^2y$, fissiamo $t \in \R$, $t = x^2y$
  \begin{enumerate}
    \item $t = 0$, $x^2y = 0 \Rightarrow y = 0 \vee x = 0$
    \item $t > 0$, $t = x^2y \iff y = \frac{t}{x^2}$
    \begin{itemize}
      \item $t = 1$, $y = \frac{1}{x^2}$
      \item $t = 2$, $y = \frac{2}{x^2}$
    \end{itemize}
    \item $t < 0$, $t = x^2y \iff y = \frac{t}{x^2}$
    \begin{itemize}
      \item $t = -1$, $y = -\frac{1}{x^2}$
      \item $t = -2$, $y = -\frac{2}{x^2}$
    \end{itemize}
  \end{enumerate}
\end{eexercise}
\begin{eexercise}
  $f(x,y) = ye^{-x}$, $t\in\R$, $t=ye^{-x} \iff e^x t = y$
  \begin{itemize}
    \item $t=0 \Rightarrow y = 0$
    \item $t=1 \Rightarrow y = e^{-x}$
    \item $t=2 \Rightarrow y = 2e^{-x}$
    \item $t=-1 \Rightarrow y = -e^{-x}$
    \item $t=-2 \Rightarrow y = -2e^{-x}$
  \end{itemize}
\end{eexercise}
\begin{eexercise}
  $$\lim_{(x,y)\to (0,0)} \frac{x-y}{\sqrt[3]{x}-\sqrt[3]{y}} = ?$$
  eleviamo x e y al numeratore per $\frac{3}{3}$, otteniamo:
  $$\lim_{(x,y)\to (0,0)} \frac{(\sqrt[3]{x})^3-(\sqrt[3]{y})^3}{\sqrt[3]{x}-\sqrt[3]{y}}$$
  Ricordiamo ora la differenza tra cubi $A^3 - B^3 = (A-B)(A^2+AB+B^2)$, otteniamo:
  $$\lim_{(x,y)\to (0,0)} \frac{(\sqrt[3]{x}-\sqrt[3]{y})\left((\sqrt[3]{x})^2 + \sqrt[3]{x}\sqrt[3]{y} + (\sqrt[3]{y})^2\right)}{\sqrt[3]{x}-\sqrt[3]{y}} = $$
  $$= \lim_{(x,y \to (0,0))} (\sqrt[3]{x})^2 + \sqrt[3]{x}\sqrt[3]{y} + (\sqrt[3]{y})^2 = 0$$
\end{eexercise}
\begin{eexercise}
  $$\lim_{(x,y)\to(0,0)} \frac{x^2y}{x^4+y^2}=?$$
  $\lim_{(x,y)\to(x_0,y_0)} f(x,y) = l \iff$ per ogni restrizione a un sottoinsieme $B$, $\lim_{(x,y)\to(x_0,y_0)} \frestr{B}(x,y) = l$
  \begin{itemize}
    \item $B=\{(x,y)\in\R^2\mid y = mx\}$, $\lim \frac{x^2y}{x^4+y^2} \lvert_{B} = \lim \frac{x^2mx}{x^4 + m^2x^2} = $
    $$= \frac{x^3m}{x^2(x^2+m^2)} = x\left(\frac{m}{x^2+m^2}\right) = \lim_{x\to 0} x\left(\frac{m}{x^2+m^2}\right) = 0$$
    \item $B=\{(x,y)\in\R^2\mid y = mx^2\}$, $\lim \frac{x^2y}{x^4+y^2} \lvert_{B} = $
    $$\lim_{x\to 0} \frac{mx^4}{x^4+m^2x^4} = \lim_{x \to 0} \frac{m}{1+m^2}$$
    Proviamo due valori di m:
    \begin{itemize}
      \item $m = 1$, $\frac{1}{2}$
      \item $m=2$, $\frac{2}{5}$
    \end{itemize} 
    Ho trovato due restrizioni $\{y = x^2\}$ e $\{y = 2x^2\}$ dove il limite assume due valori distinti. 
    Allora per l'unicit\aca del limite, il limite non esiste.
  \end{itemize}
\end{eexercise}
\begin{eexercise}
  $$\lim_{(x,y) \to (0,0)} \frac{x^2y}{x^2+y^2}$$
  \textbf{Cordinate polari}\\
  $\rho = \sqrt{x^2+y^2}$, $\vartheta = arctan\left(\frac{y}{x}\right)$
  \begin{itemize}
    \item $x = \rho \cos \vartheta$
    \item $y = \rho \sin \vartheta$
  \end{itemize}
  $$\lim_{(x,y) \to (0,0)} \frac{x^2y}{x^2+y^2} = \lim_{(x,y)\to (0,0)} \frac{\rho^2 \cos^2 \vartheta \cdot \rho \sin\vartheta}
  {\rho^2 \cos^2 \vartheta + \rho^2 \sin^2 \vartheta} = $$
  $$= \lim_{(x,y)\to (0,0)} \frac{\rho^3 \cos^2 \vartheta \cdot \sin \vartheta}{\rho^2 \left(\cos^2 \vartheta + \sin^2 \vartheta\right)}$$
  Sappiamo che $\cos^2 \vartheta + \sin^2 \vartheta = 1$, quindi il limite rimane:
  $$\lim \rho \cos^2 \vartheta \cdot \sin \vartheta$$
  $$0 \leq \lvert \rho \cos^2 \vartheta \cdot \sin \vartheta \rvert < \rho$$
  Da cui se $(x,y) \to (0,0)$ allora anche $\rho \to 0$ e siccome $\left\{\begin{array}{c}
    \cos^2 \vartheta < 1\\
    \sin \vartheta < 1 \\
  \end{array}\right.$, grazie al 
  \textbf{teorema del confronto} il limite vale 0.
\end{eexercise}
\begin{eexercise}
  Dire quali insiemi sono aperti/chiusi e quali limitati, inoltre determinare la frontiera.
  $$H=\{(x,y)\in \R^2 \mid (xy)(y-1)\geq 0\}$$
  \begin{itemize}
    \item $x\geq 0$
    \item $y\geq 0$
    \item $y-1\geq 0$, $y\geq 1$
  \end{itemize}
  Frontiera: $\partial H = \{y=1\} \cup \{x=0\} \cup \{y=0\}$
\end{eexercise}
  %\section{Esercitazione 2 - 23/03/2022}
\begin{eexercise}
  \begin{enumerate}
    \item[(a)] $$\lim_{(x,y)\to (0,0)} \frac{(e^{xy^2}-1)\log(1+x^2+y^2)}{(x^2+y^2)\sin(xy)}$$
          Ricordiamo che: 
          \begin{itemize}
            \item $\frac{\log(1+t)}{t} \xrightarrow[t\to 0]{} 1$
            \item $\frac{e^t-1}{t} \xrightarrow[t\to 0]{} 1$
            \item $\frac{\sin(t)}{t} \xrightarrow[t\to 0]{} 1$
          \end{itemize}
          Grazie a ci\aco il nostro limite diventa:
          $$\lim_{(x,y)\to (0,0)} \frac{e^{xy^2}-1}{xy^2} \cdot \frac{\log(1+x^2+y^2)}{x^2+y^2}\cdot \frac{xy}{\sin(xy)} \cdot y$$
          \begin{enumerate}
            \item[(i)] Definiamo $t = x^2+y^2 \rightarrow 0$ per $(x,y) \to (0,0)$, 
                      $$\frac{\log(1+x^2+y^2)}{x^2+y^2} = \frac{\log(1+t)}{t} \xrightarrow[t\to 0]{} 1$$
            \item[(ii)]  Definiamo $t = xy \rightarrow 0$ per $(x,y) \to (0,0)$, 
                      $$ \frac{xy}{\sin(xy)} = \frac{t}{\sin(t)} \xrightarrow[t\to 0]{} 1$$
            \item[(iii)] Definiamo $t = xy^2 \rightarrow 0$ per $(x,y) \to (0,0)$,
                      $$ \frac{e^{xy^2}-1}{xy^2} = \frac{e^t-1}{t} \xrightarrow[t\to 0]{} 1$$ 
          \end{enumerate}
          $$= 1\cdot \lim_{(x,y)\to (0,0)} y = 0$$
    \item[(c)] $$\lim_{(x,y) \to (0,0)} \frac{1-\cos(xy)}{\log(1+x^2+y^2)}$$
            Ricordiamo che: $$\lim_{t\to 0} \frac{1-\cos(t)}{t^2} = \frac{1}{2}$$
            Allora il limite diventa: 
            $$\lim_{(x,y)\to (0,0)}\frac{1-\cos(xy)}{(xy)^2}\cdot \frac{x^2+y^2}{\log(1+x^2+y^2)} \cdot \frac{(xy)^2}{x^2+y^2}$$
            \begin{itemize}
              \item[(i)] $t = xy \rightarrow 0$ per $(x,y)\to (0,0)$
                          $$\frac{1-\cos(xy)}{(xy)^2} = \frac{1-\cos(t)}{t^2} \xrightarrow[t\to 0]{} \frac{1}{2}$$
              \item[(ii)] Per (i) dell'esercizio (a) si ha: $$\lim_{(x,y)\to (0,0)}\frac{\log(1+x^2+y^2)}{x^2+y^2} = 1$$
            \end{itemize}
            $$= 1 \cdot \frac{1}{2} \cdot \lim_{(x,y)\to (0,0)} = ?$$
            Passiamo alle coordinate polari: $\left\{\begin{array}{c}
              x = \rho \cos\vartheta \\
              y = \rho \sin\vartheta \\
            \end{array}\right.$
            $$0 \leq \frac{x^2\cdot y^2}{x^2+y^2} = \frac{\rho^4 \cdot \cos^2\vartheta \cdot \sin^2 \vartheta}{\rho^2 \left(\cos^2\vartheta + \sin^2 \vartheta\right)} \leq \rho^2$$
            Per $\rho \to 0$ tutto $0 \to 0$ e $\rho^2 \to 0$, quindi anche il limite tende a zero per il teorema del confronto. \\
            Consideriamo il caso in cui $x = 0$ o $y = 0$
            \begin{itemize}
              \item Vediamo $x = 0$, 
                    $$\lim_{y\to 0} \frac{1-\cos(0)}{log(1+y^2)} = \left[\frac{0}{0}\right]_{F.IND.} = \lim_{y\to 0} 1 - \cos(0) \cdot \frac{y^2}{\log(1+y^2)}  \cdot \frac{1}{y^2} = 0$$
              \item Vediamo $y = 0$, 
                    $$\lim_{x\to 0} \frac{1-\cos(0)}{log(1+x^2)} = \left[\frac{0}{0}\right]_{F.IND.} = \lim_{y\to 0} 1 - \cos(0) \cdot \frac{x^2}{\log(1+x^2)}  \cdot \frac{1}{x^2} = 0$$
            \end{itemize}
    \item[(e)] $$\lim_{(x,y,z)\to (0,0,1)} \frac{xy(z-1)}{x^2+y^2+(z-1)^2}$$
              \begin{itemize}
                \item \textbf{Primo metodo}
                      $$\left\{\begin{array}{l}
                        x = \rho \cos\vartheta \\
                        y = \rho \sin\vartheta \\
                        t = z-1 \xrightarrow[z\to 1]{} t \to 0 \\
                      \end{array}\right.$$
                      $$0\leq \abs{\frac{\rho\cos\vartheta\cdot\rho\sin\vartheta \cdot t}{\rho^2\left(\cos^2\vartheta + \sin^2 \vartheta\right) + t^2}} \leq \abs{\frac{\rho^2\cdot t}{\rho^2+t^2}} \leq 1 \cdot t$$
                      $$\left(\frac{\rho^2}{\rho^2+t^2} \leq 1 \iff \rho^2 \leq \rho^2+t^2 \iff t^2 \geq 0 \Rightarrow \text{ sempre }\right)$$
                      Quindi per $t \to 0$, $0 \to 0$ e $t \to 0$, quindi per il teorema del confronto il limite 
                      $$\lim_{(x,y,z)\to (0,0,1)} \frac{xy(z-1)}{x^2+y^2+(z-1)^2} = 0$$
                \item \textbf{Secondo metodo}: $t = z-1 \xrightarrow[]{z\to 1} 0$ \\
                      $\lim_{(x,y,t)\to (0,0,0)} \frac{xyt}{x^2+y^2+t^2}$
                      $$0 \leq \abs{\frac{xyt}{x^2+y^2+t^2}} \leq^{?} \frac{\left(\sqrt{x^2+y^2+t^2}\right)^3}{x^2+y^2+t^2} = \sqrt{x^2+y^2+t^2}$$
                      In particolare si ha $\abs{x} \leq \sqrt{x^2+y^2+t^2} \Rightarrow x^2 \leq x^2+y^2+t^2 \iff y^2+t^2 \geq 0$, lo stesso vale per 
                      $\abs{y} \leq \sqrt{x^2+y^2+t^2}$ e $\abs{t} \leq \sqrt{x^2+y^2+t^2}$, quindi otteniamo:
                      $$0 \leq \abs{\frac{xyt}{x^2+y^2+t^2}} \leq \sqrt{x^2+y^2+t^2}$$
                      Che tende a 0 per $(x,y,t)\to (0,0,0)$, quindi grazie al teorema del confronto il limite vale 0
              \end{itemize}
  \end{enumerate}
\end{eexercise}
\begin{eexercise}
  Data $\f$ definita da $$f(x,y) = \left\{\begin{array}{cl}
    g(x,y) & (x,y) \neq (0,0) \\
    0 & (x,y) = (0,0) \\
  \end{array}\right.$$
  \begin{enumerate}
    \item[a)] $$g(x,y) = \frac{x\sin(x^2y)}{x^2+y^2} \,\forall (x,y) \not = (0,0)$$
              La funzione f, che coincide con g $\forall (x,y) \not = (0,0)$, \ace \textbf{continua} $\forall (x,y) \not = (0,0)$
              perch\ace \ace \textbf{composizione} e \textbf{prodotto} di funzioni continue (\underline{Teorema}). \\
              Dobbiamo quindi vedere il comportamento della funzione in $(0,0)$, $$\lim_{(x,y)\to (0,0)} f(x,y) = f(0,0) = 0$$
              cio\ace
              $$= \lim_{(x,y)\to (0,0)}g(x,y) = \lim_{(x,y)\to (0,0)} \frac{x\sin(x^2y)}{x^2+y^2} \cdot \frac{x^2y}{x^2y}$$
              per $x \neq 0$ e $y \neq 0$. \\\\
              (i) $t = x^2y \to 0$ per $(x,y)\to (0,0)$, $\frac{\sin(t)}{t} \to 1$ 
              $$ = 1 \cdot \lim_{(x,y)\to (0,0)}\frac{x^3y}{x^2+y^2} = 1 \cdot 0 = 0$$
              Verifichiamolo tramite le coordinate polari.\\
              $\begin{array}{l}
                x = \rho\cos\vartheta \\
                y = \rho\sin\vartheta \\
              \end{array}$
              $$0 \leq \abs{\frac{x^3y}{x^2+y^2}} = \abs{\frac{\rho^4\cdot \cos^3\vartheta \sin\vartheta}
                {\rho^2\left(\cos^2\vartheta + \sin^2\vartheta\right)}} \leq \rho^2$$
                Quindi per $\rho \to 0$ anche il limite vale 0 grazie al teorema del confronto. \\
                Abbiamo verificato che il limite $\lim_{(x,y)\to (0,0)} f(x,y) = 0 = f(0,0)$, quindi la funzione 
                f \ace continua. \\ 
                Controlliamo ora: 
                \begin{itemize}
                  \item $y = 0$ e $x\not = 0$, $\lim_{x\to 0}\frac{x\cdot 0}{x^2} = 0$
                  \item $y \not = 0$ e $x = 0$, $\lim_{y\to 0}\frac{0}{y^2} = 0$
                \end{itemize}
    \item[b)] $$g(x,y) = \frac{\sin(2xy)}{e^{x^2+y^2}-1}$$
              Dobbiamo studiarne il comportamento in $(0,0)$
              $$ = \lim_{(x,y)\to (0,0)} 2\cdot \frac{\sin(2xy)}{2xy}\cdot \frac{xy}{x^2+y^2} \cdot \frac{x^2+y^2}{e^{x^2+^2}}$$
              \begin{itemize}
                \item[(i)] $t = 2xy$, $\frac{\sin(t)}{t} \xrightarrow[t\to 0]{} 1$ per $(x,y)\to (0,0)$
                \item[(ii)] $t = x^2+y^2 \to 0$ per $(x,y)\to (0,0)$, $\frac{t}{e^t-1} \xrightarrow[t\to 0]{} 1$
              \end{itemize}
              \begin{itemize}
                \item Proviamo con le coordinate polari: $\left\{\begin{array}{l}
                        x = \rho\cos\vartheta \\
                        y = \rho\sin\vartheta \\
                      \end{array}\right.$
                      $$\Rightarrow \frac{\rho^2\sin\vartheta\cos\vartheta}{\rho^2} \Rightarrow \sin\vartheta\cos\vartheta$$
                      Quindi non va bene, allora proviamo a prendere una restrizione del dominio.
                \item $y = mx$, 
                      $$\lim_{x \to 0} \frac{x^3m}{x^2(m^2+1)} \rightarrow \frac{m}{m^2+1}$$
                      Ottenimao due rislutati diversi, $\left((m=1,\lim=\frac{1}{2}), (m=2,\lim = \frac{2}{5})\right)$,
                      quindi ho trovare due restrizioni dove il limite \ace diverso, perci\aco $\nexists \lim$.
              \end{itemize}
  \end{enumerate}
\end{eexercise}
\begin{eexercise}
  Calcolare il gradiente delle seguenti funzioni:
  \begin{enumerate}
    \item $f(x,y) = \sin(x,y)$, $\nabla f(x,y) = \left(\ppartx, \pparty\right)$
          \begin{itemize}
            \item $\frac{\partial f}{\partial x} (x,y) = \cos(xy)\cdot \frac{\partial (xy)}{\partial x} = \cos(xy)\cdot y$
            \item $\frac{\partial f}{\partial y} (x,y) = \cos(xy)\cdot \frac{\partial (xy)}{\partial y} = \cos(xy)\cdot x$
          \end{itemize}
          $\nabla f(x,y) = \left(y\cos(xy),x\cos(xy)\right) = \cos(xy)\cdot (y,x)$.\\
          Calcolare la \underline{derivata direzionale} rispetto al vettore $v = \frac{1}{\sqrt{3}}
          \left(-\frac{1}{2}, \frac{3}{2}\right)$
          $$\frac{\partial f}{\partial v} (x,y) = \tuple{\nabla f(x,y), v} = \frac{\cos(xy)}{\sqrt{3}} \tuple{(y,x),(
            -\frac{1}{2}, \frac{3}{2})} = $$ 
          $$= \frac{\cos(xy)}{\sqrt{3}} \cdot \left(-\frac{y}{2}+\frac{3x}{2}\right) = \frac{\cos(xy)}{2\sqrt{3}} (3x-y)$$
          Calcoliamo il piano tangente nei punti $(0,0,f(0,0))$ e $(1,2,f(1,2))$, ricrodiamo la formula del piano:
          $$z = f(x_0,y_0) + \tuple{\nabla f(x_0,y_0), (x,y)-(x_0,y_0)}$$
          Cerchiamo ora i valori:
          \begin{itemize}
            \item $f(x,y) = \sin(xy)$, $f(0,0) = 0$
            \item $\nabla f(x,y) = \cos(xy)(y,x)$, $\nabla f(0,0) = 1\cdot (0,0) = 0$
          \end{itemize}
          Quindi $z = 0 + 0 \Rightarrow$ il piano tangente \ace $z = 0$. \\
          Chi \ace il normale? \\
          $n = (0,0,1)$, $(x_0,y_0) = (1,2)$ 
          \begin{itemize}
            \item $f(x,y) = \sin(xy)$, $f(1,2) = \sin(2)$
            \item $\nabla f(x,y) = \cos(xy)(y,x)$, $\nabla f(1,2) = \cos(2) \cdot (2,1)$
          \end{itemize}
          $z = \sin(2) + \tuple{\cos(2)\cdot (2,1), (x-1, y-2)} = \sin(2)+\cos(2)\cdot (2x+y-4)$
  \end{enumerate}
\end{eexercise}









  %\section{Lezione 3 - 06/04/2022}
\begin{eexercise}[Es 2, Provetta]
  Siano $f : \R^3 \to \R^2, g : \R^2 \to \R^3$, 
  \begin{itemize}
    \item $f(t,u,v) = \left(k(t+v), u^2+v\right)$
    \subitem $f_1 = k(t+v)$
    \subitem $f_2 = u^2+v$
    \item $g(x,y) = \left(\log(1+x^2+y^2), \sin(x-y), x-y\right)$
    \subitem $g_1 = \log(1+x^2+y^2)$
    \subitem $g_2 = \sin(x-y)$
    \subitem $g_3 = x-y$
  \end{itemize}
  \begin{itemize}
    \item[(1)] Calcolare $D f(t,u,v)$ $\forall (t,u,v)\in \R^3$ e $D g(x,y) \forall (x,y) \in \R^2$
    \item[(2)] Calcolare la matricve Jacobiana di h in (0,0), $D h(0,0)$, dove $h = f\circ g$ 
  \end{itemize}
  \begin{itemize}
    \item[(1)]
              Iniziamo osservando che l funzioni $f_i : \R^3 \to \R$ (i = 1,2) sono $C^{\infty}$ perch\ace sono polinomi e 
              $g_i : \R^2 \to \R$ (i = 1,2,3) sono $C^{\infty}$ perch\ace composizione di funzioni $C^{\infty}$,
              $\Rightarrow$ f e g sono differenziabili, per definizione di jacobiana si ha 
              $$Df(t,u,v) = \begin{bmatrix}
                \frac{\p f_1}{\p t }(t,u,v) & \frac{\p f_1}{\p u }(t,u,v) & \frac{\p f_1}{\p v }(t,u,v) \\
                \\
                \frac{\p f_2}{\p t }(t,u,v) & \frac{\p f_2}{\p u }(t,u,v) & \frac{\p f_2}{\p v }(t,u,v) \\
                \\
                \frac{\p f_3}{\p t }(t,u,v) & \frac{\p f_3}{\p u }(t,u,v) & \frac{\p f_3}{\p v }(t,u,v) \\
              \end{bmatrix} = $$ 
              $$= \begin{bmatrix}
                \nabla f_1 \\ \nabla f_2 \\ 
              \end{bmatrix} = \begin{bmatrix}
                k & 0 & k \\
                0 & 2u & 1 \\
              \end{bmatrix}$$
              $$D g(x,y) = \begin{bmatrix}
                \frac{\p g_1}{\p x }(x,y) & \frac{\p g_1}{\p y }(x,y) \\
                \\
                \frac{\p g_2}{\p x }(x,y) & \frac{\p g_2}{\p y }(x,y) \\
                \\
                \frac{\p g_3}{\p x }(x,y) & \frac{\p g_3}{\p y }(x,y) \\
              \end{bmatrix} = \begin{bmatrix}
                \frac{2x}{1+x^2+y^2} & \frac{2y}{1+x^2+y^2} \\
                \\
                \cos(x-y) & -\cos(x-y) \\
                \\
                1 & -1 \\
              \end{bmatrix}$$
    \item[(2)] $h = f \circ g = f(g (x,y)) = h(x,y)$, $h : \R^2 \to \R^2 \Rightarrow Dh $ \ace $2\times 2$,
              Essendo f e g differenziabili, segue che la funzione composta $h = f\circ g$ \ace differenziabile e 
              vale RDC, cio\ace $D h(0,0) = D f(g(0,0)) \cdot D g(0,0)$, poich\ace 
              $g(0,0)=(0,0,0)$ e $$D g(0,0) = \begin{bmatrix}
                0 & 0 \\ 
                1 & -1 \\
                1 & -1 \\
              \end{bmatrix}$$
              $$D f(0,0,0) = \begin{bmatrix}
                k & 0 & k \\
                0 & 0 & 1 \\
              \end{bmatrix} \Rightarrow det\begin{bmatrix}
                k & k \\
                0 & 1 \\ 
              \end{bmatrix} = k \neq 0 \text{, se } k \neq 0 \Rightarrow$$
              $$\Rightarrow D h(0,0) = \begin{bmatrix}
                k & 0 & k \\
                0 & 0 & 1 \\
              \end{bmatrix} \cdot \begin{bmatrix}
                0 & 0 \\ 
                1 & -1 \\
                1 & -1 \\
              \end{bmatrix} = \begin{bmatrix}
                k & -k \\
                1 & -1 \\
              \end{bmatrix}$$
  \end{itemize}
\end{eexercise}
\begin{eexercise}[Es. 3, Provetta]
  Consideriamo la funzione $\f$, $$f(x,y) = \arctan(1+x^3+\sqrt(2)kxy+y^2-x^2) \, \forall(x,y)\in\R^2$$
  Determinare se esistono punti di massimo e/o minimo relativo o di sella. \\\\
  $arctan(t) \Rightarrow (arctan(t))' = \frac{1}{1+t^2} > 0 \Rightarrow \text{ arctan strettamente crescente}$
  Siccome arctan \ace strettamente crescente i punti di min e max rel. e sella coincidono con i punti
  di max/min/sella della funzione:
  $$g(x,y) = 1+x^3+\sqrt(2)kxy+y^2-x^2$$
  I punti critici di g sono quelli dove si annulla il gradiente $\nabla g(x,y) = (0,0) \Rightarrow$
  $$\left\{
    \begin{array}{c}
      \frac{\p g}{\p x}(x,y) = 0 \\
      \\
      \frac{\p g}{\p y}(x,y) = 0 \\
    \end{array}
  \right. \iff 
  \left\{
    \begin{array}{c}
      3x^2 + \sqrt(2)ky - 2x = 0 \\
      \\
      \sqrt(2)kx + 2y = 0 \\
    \end{array}
  \right. \iff $$
  $$\iff \left\{
    \begin{array}{c}
      3x^2 + \sqrt(2) k\left(\frac{-kx}{\sqrt(2)}\right) - 2x = 0 \\
      \\
      y = -\frac{kx}{\sqrt(2)} \\
    \end{array}
  \right. \iff \left\{
    \begin{array}{c}
      3x^2 - k^2x -2x = 0 \\
      \\
      y = -\frac{kx}{\sqrt(2)} \\
    \end{array}
  \right. \iff $$
  $$\iff \left\{
    \begin{array}{c}
      x(3x-(k^2+2)) = 0 \\
      \\
      y = -\frac{kx}{\sqrt(2)} \\
    \end{array}
  \right. \iff$$ $$ \iff \left\{ \begin{array}{c}
    x = 0 \\
    y = 0 \\
  \end{array}\right. \vee \left\{\begin{array}{c}
    x = \frac{2+k^2}{3} \\
    \\
    y = \frac{-k(2+k^2)}{3\sqrt(2)} \\
  \end{array}\right.$$
  $$\Rightarrow \begin{array}{c}
    p_1 = (0,0) \\ 
    \\
    p_2 = \left(\frac{2+k^2}{3}, \frac{-k(2+k^2)}{3\sqrt(2)} \right)
  \end{array} \text{ sono punti stazionari}$$
  \begin{itemize}
    \item $ \frac{\p g}{\p x} (x,y) = 3x^2 + \sqrt(2)ky - 2x$
      \subitem $h_{11} = \frac{\p^2 g}{\p x^2} (x,y) = 6x-2$
    \item $\frac{\p g}{\p y} (x,y) = \sqrt(2)kx + 2y$
      \subitem $h_{22} = \frac{\p^2 g}{\p y^2} (x,y) = 2$
    \item $h_{12} = h_{21} = \frac{\p^2 g}{\p x \p y} (x,y) = \sqrt(2)k = \frac{\p^2 g}{\p y \p x}$ dal teorema di Schwartz
  \end{itemize}
  $$H g(x,y) =  \begin{bmatrix}
    \frac{\p^2 g}{\p x^2} (x,y) & \frac{\p^2 g}{\p x \p y} (x,y) \\
    \\
    \frac{\p^2 g}{\p y \p x} (x,y) & \frac{\p^2 g}{\p y^2} (x,y) \\
  \end{bmatrix} = \begin{bmatrix}
    h_{11} & h_{21} \\
    \\
    h_{12} & h_{22} \\    
  \end{bmatrix} = \begin{bmatrix}
    6x-2 & \sqrt(2)k \\
    \\
    \sqrt(2)k & 2 \\
  \end{bmatrix}$$
  \begin{itemize}
    \item Calcoliamo $Hg(p_1)$
      $$Hg(p_1) = H g(0,0) = \begin{bmatrix}
        -2 & \sqrt(2)k \\
        \sqrt(2)k & 2 \\
      \end{bmatrix}$$
      Autovalori di $H g(p_1)$, $det\begin{bmatrix}
        -2-\lambda & \sqrt(2)k \\
        \sqrt(2)k & 2 -\lambda \\
      \end{bmatrix} = -4 + \lambda^2 - 2k^2 = 0 \Rightarrow$ \\
      $\Rightarrow \lambda^2 = 2k^2 + 4 \iff \lambda \pm \sqrt{4+2k^2}$
      \begin{itemize}
        \item $\lambda_1 = \sqrt{4+2k^2} > 0$ 
        \item $\lambda_2 = -\sqrt{4+2k^2} < 0$
      \end{itemize}
      $\Rightarrow$ la matrice $H g(0,0)$ non \ace definita dal colorralio viso a lezione, \\
      $detH = -4 - 2k^2 < 0 \Leftarrow detH < 0 \Rightarrow$ non definita \\
      $\Rightarrow$ per i teremi visti a lezione $(0,0)$ \ace un punto di sella. 
    \item Calcoliamo $H g(p_2)$
    $$Hg(p_2) = H g\left(\frac{2+k^2}{3}, \frac{-k(2+k^2)}{3\sqrt(2)} \right) = \begin{bmatrix}
      2+2k^2 & \sqrt(2)k \\
      \sqrt(2)k & 2 \\
    \end{bmatrix}$$
    $$det\begin{bmatrix}
      2+2k^2 & \sqrt(2)k \\
      \sqrt(2)k & 2 \\
    \end{bmatrix} = 4 +4k^2 - 2k^2 = 4+2k^2 > 0$$
    Siccome $h_{11} > 0$ e $detH g(p_2) > 0$ si ha dal corollario visto a lezione che 
    $H g(p_2)$ \ace definita positiva. \\
    Quindi per il teorema visto a lezione $p_2$ \ace un punto di minimo relativo.
  \end{itemize}
\end{eexercise}
\begin{eexercise}[Es 1, Provetta]
  Data $\f$, $$f(x,y) = \left\{\begin{array}{ll}
    \frac{(1-\cos{x})(\sin(ky))}{kx^2 + y^4} & (x,y)\neq (0,0) \\
    \\
    0 & (x,y) = (0,0) \\
  \end{array}\right.$$
  \begin{enumerate}
    \item Dire se \ace continua in (0,0)
       Per definizione di continuit\aca, f \ace continua in (0,0) \\ $\iff \exists 
        \lim_{(x,y)\to (0,0)} f(x,y) = f(0,0) = 0$.\\
        Ricordiamo i limiti notevoli:
        \begin{itemize}
          \item $\lim_{t \to 0}\frac{1-\cos(t)}{t^2} = \frac{1}{2}$
          \item $\lim_{t \to 0} \frac{\sin{t}}{t} = 1$
        \end{itemize}
        Osserviamo che $f(x,0) = 0 \, \forall x \not = 0$ e $f(0,y) = 0 \, \forall y \not = 0$ e
        $$(*) f(x,y) = \frac{1-cos(x)}{x^2} \cdot x^2 \cdot \frac{\sin(ky)}{ky} \cdot \frac{ky}{kx^2 + y^4} = 
          \frac{1}{2} \cdot 1 \cdot \frac{ky}{kx^2 + y^4}$$ 
        Notiamo che:
        $$0 \leq \abs{\frac{ky}{kx^2 + y^4}} \leq \abs{\frac{ky}{kx^2}}\leq \abs{y}$$
        Quindi per $y \to 0$ e grazie al TDC $\frac{ky}{kx^2 + y^4} \to 0$, 
        siccome tutti e tre i limiti in $(*)$ esistono e sono finiti si ha:
        $$\lim_{(x,y) \to (0,0)} \frac{1}{2} \cdot 1 \cdot 0 = 0 \Rightarrow \text{ f \ace continua in (0,0)}$$
    \item Dire se $\exists \nabla f(0,0)$\\
          $\nabla f(0,0) = \left(\frac{\p f}{\p x}(0,0), \frac{\p f}{\p y}(0,0)\right)$
          \begin{itemize}
            \item $\frac{\p f}{\p x}(0,0) = \lim_{t \to 0} \frac{f(t,0) - f(0,0)}{t} = \lim_{t \to 0} \frac{0-0}{t} = 0$
            \item $\frac{\p f}{\p y}(0,0) = \lim_{t \to 0} \frac{f(0,t) - f(0,0)}{t} = \lim_{t \to 0} \frac{0-0}{t} = 0$
          \end{itemize}
          Quindi $\exists \nabla f(0,0) = (0,0)$
    \item Dire se f \ace differenziabile in (0,0)
          $$\lim_{(x,y)\to (0,0)} \frac{f(x,y)-\tuple{\nabla f(0,0), (x,y)}}{\sqrt{x^2+y^2}} ?= 0$$
          $\tuple{\nabla f(0,0), (x,y)} = \tuple{(0,0),(x,y)} = (0,0)$, \\
          $$\Rightarrow \lim_{(x,y)\to (0,0)}\frac{f(x,y)-(0,0)}{\sqrt(x^2+y^2)} ?= 0$$
          $\lim_{(x,y)\to (0,0) } f(x,y) = 0$ da svolgimento del primo punto (1) 
          $$\Rightarrow \lim_{(x,y)\to (0,0)} \frac{0 - (0,0)}{\sqrt{x^2+y^2}} = 0$$
  \end{enumerate}
\end{eexercise}


















  %\chapter{Teoremi Orale}
\section{Continuit\aca, derivabilit\aca, differenziabilit\aca, polinomio di Taylor.}
\subsection{Teorema del confronto}
\begin{theorem}[Teorema del confronto]
  Sia $h,g,\f$, supponiamo che:
  \begin{itemize}
    \item[5.1] $f(p) \leq g(p) \leq h(p)$, $\forall p \in A \setminus \{p_0\}$
    \item[5.2] $\exists\lim_{p \to p_0} f(p) = \lim_{p \to p\to p_0} h(p) = L \in \R \cup \{\pm \infty\}$
  \end{itemize}
  allora $\exists \lim_{p \to p_0} g(p) = L$
  \begin{proof}
    Supponiamo che $L \in \R$, dobbiamo provare che $\exists \lim_{p\to p_0} g(p) = L$, cio\ace per definizione:
    \begin{itemize}
      \item[1*] $\forall \varepsilon > 0 \, \exists \delta \left(=\delta(p_0, \varepsilon)\right) > 0$
                t.c. 
                $$\lvert g(p)-L\rvert < \varepsilon \, \,\forall p \in B(p_0,\delta) \cap (A \setminus \{p_0\})$$
                Per ipotesi sappiamo che 
                  $$\lim_{p\to p_0} f(p) = L, \lim_{p\to p_0} h(p) = L $$
                cio\ace: 
      \item[2*] $\forall \varepsilon > 0 $, $$\exists \delta_1 \left(=\delta_1(p_0, \varepsilon)\right) > 0$$ t.c. 
                $\lvert f(p)-L\rvert < \varepsilon$ o eq.
                $$L - \varepsilon < f(p) < L + \varepsilon \,\, \forall p \in B(p_0,\delta_1) \cap (A \setminus \{p_0\})$$
                E
      \item[3*] $\forall \varepsilon > 0 $, $$\exists \delta_2 \left(=\delta_2(p_0, \varepsilon)\right) > 0$$
                t.c. 
                $\lvert h(p)-L\rvert < \varepsilon$ o eq.
                $$L - \varepsilon < h(p) < L + \varepsilon \, \, \forall p \in B(p_0,\delta_2) \cap (A \setminus \{p_0\})$$
    \end{itemize} 
    Da (5.1),(2*),(3*) segue che $\forall \varepsilon > 0$, scegliendo $\delta = \min\{\delta_1,\delta_2\}$ vale che 
    $$L - \varepsilon < f(p) \leq g(p) \leq h(p) < L+\varepsilon$$ $\forall p \in B(p_0,\delta) \cap (A \setminus \{p_0\})$ 
    e dunque vale la (1*).
  \end{proof}
\end{theorem}
  %\subsection{Definizione di limite per una funzione $\f$}
\begin{definition}[Limite di funzioni di due variabili]
  Sia $\f$ e sia $p_0 \in \R^2$ punto di accomulazione per A. Si dice che:
  $$\exists lim_{(x,y)\to (x_0,y_0)} f(x,y) = L \in \R$$
  oppure $\exists \lim_{p \to p_0} f(p) = L$ se 
  $$\forall \varepsilon > 0, \exists \delta = d(p_0,\varepsilon) > 0 \mid 
  \lvert f(x,y)-L\rvert < \varepsilon, \forall (x,y) \in B(p,\delta) \cap (A \setminus \{p_0\})$$
\end{definition}
  %\subsection{Definizione di continuit\aca per una funzione $\f$}
\begin{definition}
  Sia $\f$
  \begin{enumerate}
    \item f si dice continua in $p_0 \in A$ se 
    \begin{enumerate}
      \item $p_0$ \ace un punto \underline{isolato} di A, oppure
      \item $p_0$ \ace un punto di accomulazione ed $\exists \lim_{p \to p_0} f(p) = f(p_0)$
    \end{enumerate}
    \item f si dice \underline{continua} su A se f \ace continua in ogni punto $p_0 \in A$
  \end{enumerate}
\end{definition}
  %\subsection{Definizione di derivate parziali e di vettore gradiente per una
funzione $\f$ A aperto}
\begin{definition}
  \begin{enumerate}
    \item Si dice che $f$ \ace \underline{derivabile}(parzialmente) rispetto alla variabile x nel punto $p_0 = (x_0,y_0)$ se 
          $$\exists \lim_{x \to x_0} \frac{f(x,y_0) - f(x_0,y_0)}{x-x_0} := \frac{\partial f}{\partial x}(x_0,y_0) = D_1 f(x_0,y_0) \in \R$$
    \item Si dice che $f$ \ace \underline{derivabile}(parzialmente) rispetto alla variabile y nel punto $p_0 = (x_0,y_0)$ se 
          $$\exists \lim_{y \to y_0} \frac{f(x_0,y) - f(x_0,y_0)}{y-y_0} := \frac{\partial f}{\partial y}(x_0,y_0) = D_2 f(x_0,y_0) \in \R$$
    \item Se $f$ \ace derivabile (parzialmente) sia rispetto ad x ed y nel punto $p_0 = (x_0,y_0)$, si chiama (vettore)\underline{gradiente} di $f$ in $p_0$
          il vettore:
          $$\nabla f(p_0) = \left(\frac{\partial f}{\partial x}(p_0), \frac{\partial f}{\partial y}(p_0)\right) \in \R^2$$
  \end{enumerate}
  Sia $\f$, A insieme aperto. Supponiamo che:
  $$\exists \frac{\partial f}{\partial x},\frac{\partial f}{\partial y} : A \to \R$$
  allora \ace ben definito il \underline{campo} dei vettori gradiente:
  $$\nabla f : \R^2 \supseteq A \ni p \to \nabla f(p) = \left(\frac{\partial f}{\partial x}(p), \frac{\partial f}{\partial y}(p)\right) \in \R^2$$
\end{definition}
  %\subsection{Definizione di differenziabilit\aca in un punto per una funzione $\f$ 
e relazione con l'esistenza del gradiente in quel punto}
\begin{definition}
  Dato $A \subseteq \R^2$ aperto e dato $p_0=(x_0,y_0)\in A$, la funzione $\f$ si dice \underline{differenziabile}
  nel punto $p_0$ se vale 
  $$\text{(D) } \exists \lim_{(x,y)\to (0,0)}\frac{f(x)-\left[a(x-x_0)+b(y-y_0)+f(x_0)\right]}{d(p,p_0)}$$
  dove $d(p,p_0) = \sqrt{(x-x_0)^2+(y-y_0)^2}$ e per $a,b \in \R$ opportuni. \\\\
  Se f \ace differenziabile nel punto $p_0 =(x_0,y_0)$, allora $$\exists\nabla f(p_0) = \left(\ppartx, \pparty\right)$$
  e $$a = \ppartx , b =\pparty $$
  \begin{proof}
    Supponiamo che f sia differenziabile in $p_0$, cio\ace che valga (D). \\
    Ponendo nella (D), $y = y_0$ otteniamo che:
    $$\exists \lim_{x\to x_0} \frac{f(x,y_0)- \left[a(x-x_0)+f(x_0,y_0)\right]}{\lvert x -x_0\rvert} = 0$$
    $$\Rightarrow \exists\ppartx = a$$
    Procediamo allo stesso modo, ponendo $x=x_0$ nella (D) e otteniamo $\pparty = b$
  \end{proof}
\end{definition}

  %\subsection{Regola della catena nel caso generale di due funzioni, 
$f:\R^n \to \R^m$ e $g:\R^m\to\R^k$}
\begin{theorem}[Regola della catena, RDC]
  Siano $g: A \subseteq \R^n\to \R^m$ e $f : B \subseteq \R^m \to \R^k$, A e B aperti
  \begin{enumerate}
    \item[(i)] $g(A) \subseteq B$
    \item[(ii)] Se $g = (g_1, \dots, g_m)$, $f = (f_1, \dots, f_k)$ \\
              Supponiamo che  $\begin{array}{l}
                g_i : A \subseteq \R^n \to \R \, (i = 1,\dots,m) \text{ sia diff. in un dato } x_0 \in A \\
                f_i : B \subseteq \R^m \to \R \, (i = 1,\dots,k) \text{ sia diff. in un dato } y_0 = g(x_0) \\ 
              \end{array}$ \\
              Consideriamo ora la funzione $h:= f \circ g : A \subseteq \R^n \to \R^k$, $h = (h_1, \dots, h_k)$
              con $h_i : A \subseteq \R^n \to \R$, \\ allora le funzioni 
              $h_i : A \to \R (i = 1,\dots,k) \text{ sono diff. in } x_0$ e 
              $$D h(x_0) = D f(g(x_0)) \cdot D g(x_0)$$
  \end{enumerate}
\end{theorem}
  %\subsection{Formula di Taylor del II ordine per una funzione di due variabili}
\begin{definition}
  Dato $m \in \Ins{N}$, $p_0 = (x_0,y_0) \in \R^2$ fissato, si chiama \underline{polinomio di ordine m} di $n=2$ variabili,
  centrato in $p_0$, una funzione $T: \R^2 \to \R$ del tipo
  $$T(x,y) = \sum_{h=0}^{m} \sum_{i = 0}^n c_{i,h-i} (x-x_0)^i(y-y_0)^{h-i}$$
  $(x,y)\in \R^2$, dove $c_{i,h-i}$ (i = 0,...,h e h = 0,..., m) sono $\frac{(m+1)(m+2)}{2}$ coeff. ass. \\\\
  Sia $f \in C^2\left(B(p_0,r)\right)$, $p_0 = (x_0,y_0) \in \R^2$ e $r > 0$ fissato. Allora vale:
  $$\left(FT_2\right) f(p) = T_2(p) + o\left(\norma{p-p_0}^2\right)$$
  $\forall p = (x,y) \in B(p_0,r)$, dove 
  $$T_2(p) := f(p_0) + \tuple{\nabla f(p_0), p-p_0} + \frac{1}{2} \tuple{D^2f(p_0) \cdot (p-p_0), p-p_0}$$
  se $p \in \R^2$. \\
  (polinomio di taylor del II ordine di f, centrato in $p_0$) 
\end{definition}
  %\subsection{ Definizione di matrice Hessiana per un funzione $\f$
e sua applicazione nella formula di Taylor del II ordine}
\begin{definition}
  Data $f \in C^2(A)$, $A \in \R^2$ aperto, si chiama, \underline{matrice hessiana}
  di f in un punto $p\in A$, la matrice $2\times 2$
  $$D^2f(p) = H(f)(p) = \begin{bmatrix}
    \frac{\p^2 f}{\p x^2}(p) & \frac{\p^2 f}{\p y \p x}(p) \\
    \frac{\p^2 f}{\p x \p y}(p) & \frac{\p^2 f}{\p y^2}(p) \\
  \end{bmatrix}_{2\times 2}$$
\end{definition}
\hfill \break
L'applicazione della matrice Hessiana nel PT2o si pu\aco trovare nello 
sviluppo della dimostrazione, infatti per una funzione
$F(t) = f(p_0+tv), t \in (-r,r) \text{ e } B(p_0,r)$ andando a calcolare il polinomio di Taylor 
per $t = 0$, e supponendo di avere $v = \frac{p-p_0}{\norma{p-p_0}}$,
otteniamo che $F''(t)$:
$$F''(t) = v_1 \cdot \tuple{\nabla \left(\frac{\p f}{\p x}\right)(p_0+tv), v} + 
            v_2 \cdot \tuple{\nabla \left(\frac{\p f}{\p y}\right)(p_0+tv), v} = $$
$$= v_1 \left(\frac{\p^2 f}{\p x^2}(p_0+tv)v_1 + \frac{\p^2 f}{\p y \p x}(p_0+tv)v_2\right) + 
    v_2 \left( \frac{\p^2 f}{\p x \p y}(p_0+tv)v_1 + \frac{\p^2 f}{\p y^2}(p_0+tv)v_2\right) = $$
$$ = \frac{\p^2 f}{\p x^2}(p_0+tv)v_1^2 + 2 \frac{\p^2 f}{\p y \p x}(p_0+tv)v_1 v_2 + 
      \frac{\p^2 f}{\p y^2}(p_0 + tv)v_2^2$$
Pertanto calcolando $F''(0)$ otteniamo:
$$F''(0) = \frac{\p^2 f}{\p x^2}(p_0+tv)v_1^2 + 2 \frac{\p^2 f}{\p y \p x}(p_0+tv)v_1 v_2 + 
  \frac{\p^2 f}{\p y^2}(p_0 + tv)v_2^2$$
Che pu\aco essere riscritto mediante matrice Hessiana del tipo:
$$F''(0) = \tuple{D^2f(p_0)v,v}$$
E sostituendola otteniamo
$$f(p_0+tv) = F(t) = f(p_0) + \tuple{\nabla f(p_0), v}t + \frac{1}{2} \tuple{D^2 f(p_0)v, v}t^2 + o(t^2) , \text{ per } t \to 0$$
Scegliendo $t = \norma{p-p_0}$ e otteniamo la forma del polinomio di Taylor di II ordine. 
Ed \ace questa l'applicazione della matrice Hessiana.
  %\section{Massimi e minimi}
\subsection{Definizione di punto di massimo/minimo relativo, massimo/minimo
assoluto e punto di sella per una funzione $\f$}
\begin{definition}
  Data $\f$: 
  \begin{enumerate}
    \item $p_0 \in A$ si dice, punto di \underline{massimo} (= max) \underline{relativo} di f su A se 
          $\exists r_0 > 0$ t.c. $f(p) \leq f(p_0) \, \forall p \in A \cap B(p_0,r_0)$ \\
          Rispettivamente $p_0 \in A$ si dice, punto di \underline{minimo} (= min) \underline{relativo} di f su A se 
          $\exists r_0 > 0$ t.c. $f(p) \geq f(p_0) \, \forall p \in A \cap B(p_0,r_0)$
    \item $p_0 \in A$ si dice punto di \underline{massimo} (= MAX) \underline{assoluto} se 
          $\forall p \in A$, $f(p) \leq f(p_0)$ \\
          Rispettivamente $p_0 \in A$ si dice punto di \underline{minimo} (= MIN) \underline{assoluto} se 
          $\forall p \in A$, $f(p) \geq f(p_0)$ 
  \end{enumerate}
\end{definition}
\begin{definition}
  Sia $\fn$, A aperto. Un punto $p_0 \in A$ si dice \underline{punto di sella} se $p_0$ \ace un punto 
  stazionario di f e $f(p)-f(p_0)$ amette sia valori positivi che negativi in ogni intorno di $p_0$
\end{definition}
  %\subsection{Teorema di Fermat sui punti stazionari di una funzione}
\begin{theorem}[Fermat]
  Sia $\fn$, A aperto. Supponiamo che esista $p_0 \in A$ t.c. 
  \begin{enumerate}
    \item[(i)] f differenziabile in $p_0$. In particolare $\exists \nabla f(p_0)$
    \item[(ii)] $p_0$ sia un estremo libero di f in A
  \end{enumerate}
  Allora $\nabla f(p_0) = \underline{O}_{\R^n} = (0,...,0) \text{ (n-volte)}$
\end{theorem}
\begin{definition}
  Data $\fn$, A aperto, un punto $p_0\in A$ si chiama \underline{punto stazionario}(o \underline{critico}) 
  di f se f \ace differenziabile in $p_0$ e $\nabla f(p_0) = \underline{O}_{\R^n}$
\end{definition}
  %\subsection{Teorema di Weierstrass sull’esistenza del massimo e minimo assoluto di una funzione}
\begin{theorem}[Weirestrass][BDPG,10.10]
  Sia $\fn$, Supponiamo che:
  \begin{itemize}
    \item[(i)] A sia limitato e chiuso, (in $n=1$, $A = [a,b], \p A = \{a,b\}, \overcirc{A} = (a,b)$)
    \item[(ii)] f sia continua su A 
  \end{itemize}
  Allora esiste $\min_{A}f$ e $\max_{A}f$
\end{theorem}
  %\subsection{Metodo dei moltiplicatori di Lagrange per la ricerca di massimi e
minimi vincolati per funzioni di due variabili}
\begin{theorem}[Teorema dei moltiplicatori di Lagrange, TML]
  Sia $f \in C^1(\R^2)$ e $\V = \{(x,y)\in \R^2 : g(x,y) = 0\}$ dove $g \in C^1(\R^2)$. Supponiamo che:
  \begin{enumerate}
    \item[(i)] $\exists \min_{\V} f = f(p_0) (\text{o } \exists \max_{\V} f = f(p_0))$ con $p_0 = (x_0,y_0) \in \V$
    \item[(ii)] $\exists \nabla g(p_0) \not = (0,0)$
  \end{enumerate}
  Allora esiste $\lambda_0 \in \R$ (detto \underline{moltiplicatore}) t.c. $(x_0,y_0,\lambda_0)\in \R^3$ \ace un 
  punto stazionario della funzione. \\
  Equivalentemente: $$\exists\lambda_0 \in \R \text{ t.c. } \left\{\begin{array}{l}
    g(p_0) = 0 \\
    \\
    \nabla f(p_0) + \lambda_0 \nabla g(p_0) = (0,0) \\
  \end{array}\right. (*)  \label{tml_v2}$$
\end{theorem}
  %\section{Integrali per funzioni in pi\acu variabili}
\subsection{Definizione di insieme insieme semplice (o normale) in $\R^2$
rispetto agli assi cartesiani}
\begin{definition}
  Un sottoinsieme $A \subset \R^2$ si dice 
  \begin{itemize}
    \item \underline{Dominio semplice} (o normale) rispetto all'asse y se esistono 
          $g_1, g_2 \in C^0([a,b])$ t.c. $g_1 \leq g_2$ su $[a,b]$ e 
          $$A = \{(x,y)\in\R^2 : x \in [a,b], g_1(x) \leq y \leq g_2(x)\}$$
    \item \underline{Dominio semplice} (o normale) rispetto all'asse x se esistono 
          $h_1, h_2 \in C^0([c,d])$ t.c. $h_1 \leq h_2$ su $[c,d]$ e 
          $$A = \{(x,y)\in\R^2 : y \in [c,d], h_1(y) \leq x \leq h_2(y)\}$$
  \end{itemize}
\end{definition}
  %\subsection{Formula di riduzione di integrali doppi su insiemi semplici}
\begin{theorem}[Forumla di riduzione su domini semplici][BDPG,14.17]
  Sia $A\subseteq\R^2$ un dominio semplice rispetto ad uno degli assi. Supponiamo che $f\in C^{0}(A)$, allora 
  $f\in\Rcal(A)$ e valgono le seguenti formule:
  \begin{enumerate}
    \item Se $A = \{(x,y)\in\R^2 : x \in [a,b], g_1(x) \leq y \leq g_2(x)\}$ con $g_1, g_2 \in C^0([a,b])$, allora 
          $$(1) \iint_{A} f = \int_{a}^{b} \left(\int_{g_1(x)}^{g_2(x)} f(x,y) \, dy\right) \, dx$$
          In particoalre A \ace misurabile e $\abs{A}_2 = \iint_{A} 1 = \int_{a}^{b} \left(g_2(x)-g_1(x)\right) \, dx$
    \item Se $A = \{(x,y)\in\R^2 : y \in [c,d], h_1(y) \leq x \leq h_2(y)\}$ con $h_1, h_2 \in C^0([c,d])$, allora 
          $$(2) \iint_{A} f = \int_{c}^{d} \left(\int_{h_1(y)}^{h_2(y)} f(x,y) \, dx\right) \, dy$$
          In particoalre A \ace misurabile e $\abs{A}_2 = \iint_{A} 1 = \int_{c}^{d} \left(h_2(y)-h_1(y)\right) \, dy$
  \end{enumerate}
\end{theorem}
  %\subsection{Formula di cambiamento di variabili per integrali doppi e tripli}
\begin{theorem}[Cambiamento di variabili negli integrali doppi][BDPG,14.19]
  Siano $D, D^* \subseteq \R^2$ aperti limitati e misurabili, sia $\psi : D^* \to D$
  un cambiamento di variabili e sia $f:D \to \R$ continua e limitata. \\
  Allora vale la formula
  $$(FCV)_2 \, \iint_{D} f(x,y) \,dx\,dy = \iint_{D^*} f(\psi(u,v))\abs{\det D\psi(u,v)} \,du\,dv$$ 
\end{theorem}
  %\subsection{Cambiamento di coordinate cilindriche e sferiche}
\subsubsection{Coordinate ellittiche}
\begin{definition}
  Error 404, non le trovo!! Suppongo: \\\\
  Coordinate ellittiche:
  $$\Psi \equiv \left\{\begin{array}{l}
    x = \rho\cos\vartheta \\
    y = \rho\sin\vartheta \\
  \end{array}\right.$$
  $0 \leq \vartheta\leq 2\pi, \rho\geq 0$,
  $\abs{\det D\Psi(\rho,\vartheta)} = \rho^2$
\end{definition}
\subsubsection{Coordinate sferiche}
\begin{definition}
  Coordinate sferiche:
  $$\Psi \equiv \left\{\begin{array}{l}
    x = r \sin\varphi\cos\vartheta \\
    y = r \sin\varphi\sin\vartheta \\
    z = r \cos\varphi \\
  \end{array}\right.$$
  $0 \leq \vartheta\leq 2\pi, r\geq 0, 0 \leq \varphi \leq \pi$,
  $\abs{\det D\Psi(r,\vartheta,\varphi)} = r^2 \sin\varphi$
\end{definition}
  %\subsection{Formule di riduzione per integrali tripli su un parallelepipedo}
\begin{theorem}[Formule di riduzione per integrali tripli rispetto a domini semplici][BDPG,14.28]
  Sia $A\subseteq \R^3$ un insieme semplice rispetto all'asse z di tipo 
  $$A=\{(x,y,z)\in\R^3: (x,y)\in E, g_1(x,y)\leq z \leq g_2(x,y)\}$$
  e sia $f\in C^0(A)$. Allora 
  $$\iiint_{A}f = \iint_{E} \left(\int_{g_1(x,y)}^{g_2(x,y)} f(x,y,z) \,dz\right) \,dx\,dy$$
\end{theorem}
  %\subsection{Definizione di insieme definito per fili e per strati}
\begin{definition}
  Dato un insieme $Q = [a_1,b_1] \times [a_2,b_2]\times[a_3,b_3]$ si definisce:
  \begin{itemize}
    \item L'insieme $\{(x,y,z)\in\R^3:(y,z)\in[a_2,b_2]\times[a_3,b_3]\}$ \ace uno strato.
    \item L'insieme $\{(x,y,z)\in\R^3: z\in[a_3,b_3]\}$ \ace un filo.
  \end{itemize}
\end{definition}
  %\subsection{Formula di integrazione per fili e per strati}
\begin{theorem}[Formule di riduzione su parallelepipedi][BDPG,14.26]
  Siano $Q = [a_1,b_1]\times [a_2,b_2]\times [a_3,b_3]$, $f \in C^0(Q)$
  \begin{enumerate}
    \item[(i)] La funzione $$[a_1,b_1]\times [a_2,b_2] \ni (x,y) \to \int_{a_3}^{b_3} f(x,y,z) \,dz$$
              \ace integrabile su $[a_1,b_1]\times [a_2,b_2]$ e 
              $$(1) \iiint_{Q} f = \iint_{[a_1,b_1]\times [a_2,b_2]} \left(\int_{a_3}^{b_3} f(x,y,z) \,dz\right)$$
    \item[(ii)] La funzione $$[a_1,b_1] \ni x \to \iint_{[a_2,b_2]\times [a_3,b_3]} f(x,y,z) \,dy\,dz$$
                \ace integrabile su $[a_1,b_1]$ e 
                $$(2) \iiint_{Q} f = \int_{a_1}^{b_1} \left(\iint_{[a_2,b_2]\times[a_3,b_3]} f(x,y,z) \,dz\right)$$
  \end{enumerate}
  La \begin{itemize}
    \item[(1)] si chiama formula di riduzione per \underline{fili}
    \item[(2)] si chiama formula di riduzione per \underline{strati}
  \end{itemize}
\end{theorem}
  %include{ListaTeoremiOrale/teo-20}
  %\subsection{Definizione di curva chiusa, semplice, regolare, orientazione (o
verso di percorrenza) di una curva semplice}
\begin{definition}
  \begin{itemize}
    \item[(i)] La curva $\gamma$ si dice \underline{chiusa} se $I = [a,b]$ e $\gamma(a) = \gamma(b)$
    \item[(ii)] La curva $\gamma:I\to\R^n$ si dice \underline{semplice} se $\gamma$ \ace iniettiva, o 
                se $\gamma$ \ace chiusa e $I = [a,b]$, allora $\gamma :[a,b) \to \R^n$ \ace iniettiva.
  \end{itemize}
\end{definition}
\begin{definition}
  Una curva $\gamma:I\to\R^n$ si dice \underline{regolare} se $\gamma$ 
  \ace di classe $C^1$ e $\gamma'(t) \neq \origine_{\R^n}$ $\forall t \in I$ 
\end{definition}
\begin{definition}
  Sia data una curva semplice $\gamma:I\to\R^n$. Allora 
  essa induce \underline{un'orientazione} sul suo sostegno $\gamma(I) \subseteq \R^n$.
  Pi\acu precisamente \\\\
  Data $\gamma:I\to\R^n$ curva semplice, si dice che il punto $x_1 = \gamma(t_1)$ \underline{precede}
  il punto $x_2 = \gamma(t_2)$ se $t_1 < t_2$. L'orientazione della curva viene detta
  anche \underline{verso} della curva.
\end{definition}
  %%include{ListaTeoremiOrale/teo-23}
  %\subsection{ Formula per il calcolo della lunghezza di una curva rettificabile}
Vogliamo ora definire la nozione di lunghezza di una curva.\\
Sia $\g:[a,b]\to\R^n$ una curva e sia $\D := {t_0 = a < t_1 < ... < t_N = b}$ 
una suddivisione di $[a,b]$: essa induce una suddivisione del sostegno di 
$\g$ in $N+1$ parti definite da 
$\g(t_0),\g(t_1)\dots \g(t_N)$. \\
Consideriamo i segmenti
$$[\g(t_{i-1}),\g(t_i)] := \{s\g(t_i)+(1-t)\g(t_{i-1}) : 0 \leq s \leq 1\}$$
$i = 1,...,N$. La lunghezza della spezzata definita dall'unione 
$\bigcup_{i=1}^{N} [\g(t_{i-1}),\g(t_i)]$ \ace data da 
$$L(\g,\D) := \sum_{i=1}^{N} \norma{\g(t_i)-\g(t_{i-1})} \in [0,+\infty)$$
Denotiamo 
$$L(\g) := sup_{\D} L(\g,\D) \in [0,+\infty] =_{def} [0,+\infty) \cup \{+\infty\}$$
  %\subsection{ Definizione di integrale curvilineo di prima specie per una funzione
continua f lungo una curva $\g$ di classe $C^1$}
\begin{definition}
  Sia $\g : [a,b]\to\R^n$ una curva di classe $C^1$ e sia $f:\Gamma \to \R$
  una funzione continua. Si definisce
  $$\int_{\gamma} f \,ds = \int_{a}^{b} f(\gamma(t))\norma{\g'(t)} \,dt$$
  e si chiama \underline{Integrale curvilineo} di I specie di f lungo $\g$.
\end{definition}
  %\subsection{ Definizione di integrale curvilineo di seconda specie di una forma
differenziale lungo una curva di classe $C^1$}
\begin{definition}
  Sia $\g:[a,b]\to E\subseteq\R^n$ una curva di classe $C^1$ e sia
  $\omega$ una forma differenziale di classe $C^0$ su E. \\
  Si definisce \underline{integrale curvilineo} di II specie di $\omega$
  (o del campo F) lungo $\g$ il valore 
  $$\int_{\g} \omega := \int_{a}^{b} \tuple{f(\g(t)), \g'(t)}\,dt = 
    \int_{a}^{b} \sum_{i=1}^{n} F_i(\g(t))\g_i'(t) \,dt$$
  Se $\g$ fosse chiusa il precedente integrale si scrive anche $\oint_{\g} \omega$
\end{definition}
  %\subsection{Definizione di forma differenziale esatta e di potenziale di una
forma differenziale}
Sia $E\subseteq\R^n$ un insieme aperto e sia $\U \in C^1(E)$. Possiamo associare ad $\U$
la forma diff. 
$$d\U = \tuple{\nabla\U,dx} = \frac{\p \U}{\p x_1}\,dx_1+\dots+\frac{\p \U}{\p x_n}\,dx_n$$
che viene anche chiamata \underline{differenziale di $\U$} poich\ace coincide con la notazione
con cui indichiamo il differenziale di $\U$
\begin{definition}
  Sia $E\subseteq\R^n$ un aperto e sia $\om = \tuple{F,dx}$ dove
  $F:E\to\R^n$ di classe $C^0$. La forma $\om$ si dice \underline{esatta} in E 
  se esiste $\U:E\to\R$ di classe $C^1$ t.c.
  $$\nabla\U(x) = F(x) \, \forall x \in E$$
  o, equivalentemente, $d\U = \om$.\\
  In tal caso $\U$ \ace detta \underline{funzione potenziale} (o primitiva) di $\om$ in E.
\end{definition}


  %\subsection{Formula per il calcolo dell’integrale curvilineo di una forma differenziale esatta}
\begin{theorem}[Integrale per forme esatte]
  Sia $E \subseteq \R^n$ aperto, $\om$ forma diff. continua ed esatta su E. Allora per ogni
  curva $\g:[a,b]\to E$ $C^1$ a tratti vale che 
  $$(*)\, \int_{\g}\om = \U(\g(b))-\U(\g(a))$$
  dove $\U:E\to\R$ \ace un qualunque potenziale di $\om$
\end{theorem}
  %\section{Superfici}
\subsection{Definizione di superficie elementare di $\R^3$: parametrizzazione di una superficie}
\begin{definition}
  Un sottoinsieme $S\subset\R^3$ si dice \underline{superficie} (elementare) se esiste una mappa
  $\sigma:\overline{D}\subseteq\R^2\to\R^3$, 
  $$\sigma(u,v) = (x(u,v),y(u,v),z(u,v))$$
  verificante
  \begin{enumerate}
    \item D \ace un aperto di $\R^2$, interno di una curva di Jordan
    \item $\sigma$ \ace continua e $\sigma:D\to\R^3$ \ace iniettiva
    \item $\sigma(\overline{D})=S$
  \end{enumerate}
  Una funzione verificante (1.-3.) \ace detta parametrizzazione di S. \\\\
  S si dice \underline{superficie cartesiana} se esiste una parametrizzazione
  $\sigma : \overline{D}\subseteq\R^2\to\R^3$ del tipo
  $$\sigma(u,v) = \begin{array}{lr}
    (u,v,f(u,v)) & z = f(x,y) \\
    \text{oppure} \\
    (f(u,v),u,v) & x = f(y,z) \\
    \text{oppure} \\
    (u,f(u,v),v) & y = f(x,z) \\
  \end{array} (u,v)\in\overline{D}$$
  dove $f:\overline{D}\to\R$ continua.
\end{definition}
  %\subsection{Definizione di area per una superficie regolare}
\begin{definition}
  Sia S una superficie regolare di parametrizzazione $\sigma : D\subseteq \R^2 \to \R^3$
  insieme misurabile e supponiamo che la funzione 
  $$(*) D \ni (u,v) \to \norma{\sigma_u\wedge\sigma_v}(u,v)$$
  sia limitata. \\
  Si chiama \underline{area di S} il valore 
  $$A(S) := \iint_D \norma{\sigma_u(u,v)\wedge\sigma_v(u,v)}\,du\,dv$$
  Una superficie S regolare per cui valga $(*)$ si dice \underline{di area ben definita}
\end{definition}
  %\subsection{Area di una superficie cartesiana regolare e di una superficie di rotazione}
\subsubsection{Superficie regolare cartesiana}
\begin{definition}
  Sia $D\subseteq\R^2$ interno di una curva di Jordan, e sia 
  $f\in C^0(\overline{D})\cap C^1(D)$ e supponiamo che 
  $\p_u f, \p_v f : D \to \R$ siano limitate. \\
  Allora se $S = G_f := \{(u,v,f(u,v)):(u,v)\in D\}$ (superficie cartesiana), 
  $$A(S) = \iint_{D} \sqrt{1+\abs{\nabla f(u,v)}^2} \,du \,dv$$
\end{definition}
\subsubsection{Superficie di rotazione}
\textbf{Non riesco a trovarla, suppongo:}
\begin{definition}
  Calcolare l'area della sfera di centro (0,0,0) e raggio $r>0$ \\
  \textbf{Soluzione:} \\\\
  Possiamo rappresentare 
  $$S = \{(x,y,z)\in\R^3:x^2+y^2+z^2 = r^2\}$$
  e consideriammo la sua parametrizzazione in coordinate sferiche, cio\ace 
  la mappa $\sigma:\overline{D}\to\R^3$, $\overline{D}=[0,2\pi]\times[0,\pi]$, 
  $$\sigma(u,v) = r\left(\cos{u}\sin{v}, \sin{u}\sin{v}, \\cos{v}\right)$$
  Sappiamo che $$\norma{\sigma_u\wedge\sigma_v} = r^2\abs{\sin{v}}$$
  Pertanto 
  $$A(S) = \iint_{D} \norma{\sigma_u\wedge\sigma_v} (u,v) \,du\,dv = \iint_{D} r^2\abs{\sin{v}} \,du\,dv = $$
  $$= r^2\left(\int_{0}^{2\pi} \,du\right)\cdot\left(\int_{0}^{\pi}\sin{v} \,dv\right) = 
      r^2\cdot 2\pi \left(\left.-cos{v}\right|_{0}^{\pi}\right) = 4\pi r^2$$
\end{definition}
  %\subsection{Definizione di integrale di superficie per una funzione continua
$f:\Sigma\subseteq\R^3\to\R$, dove $\Sigma$ è una superficie regolare}
\end{document}