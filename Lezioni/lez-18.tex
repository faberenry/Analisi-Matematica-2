\section{Lez - 18, Superfici orientabili}
Per introdurre il concetto di orientabilit\aca di una superficie
partiamo dal caso pi\acu semplice di una superficie cartesiana. \\
Per esempio, supponiamo che una superficie sia parametrizzata dalla mappa 
$$\sigma:\overline{D}\to\R^3 \, \sigma(u,v)=(u,v,f(u,v))$$
$D\subseteq \R^2$ interno di una curva di Jordan $f \in C^0(\overline{D})\cap C^1(D)$. \\
I due versori normali a S nel punto $p = \sigma(u,v)$, $(u,v)\in D$ sono date da 
$$\pm \frac{\sigma_u(u_0,v_0)\wedge\sigma_v(u_0,v_0)}
{\norma{\sigma_u(u_0,v_0)\wedge \sigma_v(u_0,v_0)}} = \pm \frac{\left(-\p_u f(u,v), -\p_v f(u,v), 1\right)}
    {\sqrt{1+\abs{\nabla f(u,v)}^2}}$$
Si noti che la terza componente del vettore 
$$\pm \frac{1}{\sqrt{1+\abs{\nabla f(u,v)}^2}} \neq 0$$
e $S' = \sigma(D)$. \\
Pertanto \ace sempre possibile selezionare, in ciascun punto, il versore normale che 
"punti verso l'alto" e lo denotiamo con 
$$N_S^{+} (x,y,z) = \frac{\sigma_u\wedge \sigma_v}{\norma{\sigma_u\wedge\sigma_v}}\left(\sigma^{-1}(x,y,z)\right)$$
$(x,y,z)\in S'$, dove $\sigma^{-1}:S'\to D$, $\sigma'(x,y,z) = (x,y)$, mentre 
denotiamo 
$$N_S^{-} (x,y,z) = - N_S^{+} (x,y,z)$$
Si noti che $N_S^{+}, N_S^{-}: S' \to \R^3$ sono continue. \\
Si dice in questo caso che sono possibili due \underline{orientazioni} della superficie S, indotte
dalla parametrizzazione. \\
Pi\acu in generale vale la seguente
\begin{definition}
  Una superficie $S\subseteq\R^3$ si dice \underline{orientabile} se esiste una mappa 
  $N_S^{+} : S'\to\R^3$ t.c. 
  \begin{enumerate}
    \item $N_S^{+}(p)$ coincide con uno dei due versori normali definiti tramite parametrizzazione;
    \item $N_S^{+}$ \ace continua.  
  \end{enumerate}
  $N_S^{+}$ \ace detto \underline{versore normale positivo} a S. \\
  Definiamo $N_S^{-} = - N_S^{+}$
\end{definition} 
\begin{example}
  \begin{itemize}
    \item (Porzione di paraboloide) \\
          Sia $S=\{(x,y,x^2+y^2): x^2+y^2\leq r^2\}$ \\
          Sappiamo che l'insieme dei punti interni \ace dato da 
          $$S'$$
  \end{itemize}
\end{example}
\subsection{Idea per definire l'area di una superficie}
