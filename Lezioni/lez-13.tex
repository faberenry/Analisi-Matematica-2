\section{Lez - 13, Integrali tripli, [BDPG,14.5]}
\subsection{Integrale triplo su un parallelepipedo}
Sia $Q:= [a_1,b_1]\times [a_2,b_2]\times [a_3,b_3]  \subset \R^3$ un parallelepipedo.
Siano
\begin{itemize}
  \item $\D_1 := \{a_1 = x_0 \leq \dots \leq x_i \leq \dots x_{n_1} = b_1\}$ sudd. di $[a_1,b_1]$
  \item $\D_2 := \{a_2 = y_0 \leq \dots \leq y_j \leq \dots y_{n_2} = b_1\}$ sudd. di $[a_2,b_2]$
  \item $\D_3 := \{a_3 = z_0 \leq \dots \leq z_k \leq \dots z_{n_3} = b_3\}$ sudd. di $[a_3,b_3]$
\end{itemize}
\begin{definition}
  Si chiama \underline{suddivisione $\D$} del parallelepipedo $Q=[a_1,b_1]\times [a_2,b_2]\times [a_3,b_3]$
  un insieme del tipo 
  $$\D = \D_1 \times \D_2 \times \D_3 = \{(x_i,y_j,z_k): i = 0,...,n_1; j = 0,...,n_2; k = 0,..., n_3\}$$
  se $\D_1, \D_2, \D_3$ sono definiti come sopra. 
\end{definition}
Data una suddivisione $\D$ di Q, Q risulta suddiviso in $n_1\times n_2 \times n_3$ parallelepipedi
$$Q_{ijk} = [x_{i-1}, x_i]\times [y_{j-1},y_j] \times [z_{k-1},z_k]$$
con $i = 1,..., n_1$, $j = 1,...,n_2$, $k = 1,..., n_3$, il cui volume \ace 
$$vol(Q_{ijk}) = \abs{Q_{ijk}}_3 := (x_i-x_{i-1})(y_j-y_{j-1})(z_k-z_{k-1})$$
Sia $f : Q \to \R$ limitata e definiamo
$$m_{ijk} = inf_{Q_{ijk}} f \in \R \text{ e } M_{ijk} = sup_{Q_{ijk}} f \in \R$$
con $i = 1,..., n_1$, $j = 1,...,n_2$, $k = 1,..., n_3$
Definiamo 
\begin{itemize}
  \item $S(f,Q) = \sum_{ijk} M_{ijk} \cdot \abs{Q_{ijk}}_3$ (somma superiore di f su Q)
  \item $s(f,Q) = \sum_{ijk} m_{ijk} \cdot \abs{Q_{ijk}}_3$ (somma inferiore di f su Q)
\end{itemize}
\begin{definition}
  Si dice che f \ace integrabile su Q e si scrive $f \in \Rcal(Q)$ se 
  $$L = sup_{\D} s(f,\D) = inf_{\D} S(f,\D) \in \R$$
  Il valore L prende nome di \underline{integrale triplo} di f su Q e si denota con i simboli
  $$\iiint_{Q} f(x,y,z) \,dx\,dy\,dz, \iiint_{Q} f, \int_{Q} f, \int_{Q} f \,dx\,dy\,dz$$
\end{definition}
Continuano a valere i risultati degli interali doppi su rettangoli.
\subsubsection{Propriet\aca}
\begin{theorem}[Esistenza dell'integrale]
  Sia $f \in C^0\left(Q\right)$ allora $f \in \Rcal(Q)$
\end{theorem}
\begin{theorem}[Propriet\aca dell'integrale]
  Siano $f,g \in \Rcal (Q)$
  \begin{itemize}
    \item[(i)] \textbf{Linearit\aca}: $\alpha f + \beta g \in \Rcal(Q)$, $\forall \alpha,\beta\in\R$ e 
              $$\int_{Q}\left(\alpha f + \beta g\right) = \alpha \int_{Q}f + \beta \int_{Q} g$$
    \item[(ii)] \textbf{Monotonia}: Se $g\leq f$ su Q, allora $$\int_{Q} g \leq \int_{Q}f$$
    \item[(iii)] \textbf{Valore assoluto}: $\abs{f} \in \Rcal(Q)$ e $$\abs{\int_{Q}f} \leq \int_{Q}\abs{f}$$
    \item[(iv)] \textbf{Teorema della media interale} 
                $$inf_{Q} f \leq \frac{1}{\abs{Q}_3}\int_{Q} f \leq sup_{Q}f$$
                Se $f \in C^0(Q)$, allora esiste $p_0 = (x_0,y_0,z_0)\in Q$ t.c. 
                $$f(p_0) = \frac{1}{\abs{Q}_3}\int_{Q} f$$
  \end{itemize}
\end{theorem}
\subsection{Integrale triplo su insiemi generali}
\begin{definition}
  Sia $A\subseteq \R^3$ limitato, $f: A \to \R$ limitata. Allora f si dice \underline{integrabile}
  su A, e si scrive $f\in\Rcal(A)$ se la funzione $\ftilde:Q\to\R$ definita come
  $$ \ftilde(x,y,z) := \left\{\begin{array}{ccl}
    f(x,y,z) & se & (x,y,z)\in A \\
    0 & se & (x,y,z)\in Q\setminus A \\
  \end{array}\right.$$
  \ace integrabile su Q, dove Q \ace un (qualunque) parallelepipedo contenente A.\\
  Si pone: $$\int_A f := \int_{Q} \ftilde$$ 
\end{definition}
\begin{definition}
  Un insieme $A \subseteq \R^3$ limitato si dice \underline{misurabile} (secondo Peano-Jordan) se la funzione
  $\ftilde : Q \to \R$, definita come 
  $$ \ftilde(x,y,z) := \left\{\begin{array}{ccl}
    1 & se & (x,y,z)\in A \\
    0 & se & (x,y,z)\in Q\setminus A \\
  \end{array}\right.$$
  \ace integrabile su Q, per un opportuno parallelepipedo $Q \supseteq A$
  $$\abs{A}_3 := \iiint_{A} 1 \,dx\,dy\,dz$$
  (misura 3-dimensionale o volume di A)
\end{definition}
Continuano a valere 
\begin{theorem}[Caraterizzazione degli insiemi misurabili]
  Sia $A \subseteq \R^3$ limitato. Allora 
  $$\text{ A \ace misurabile} \iff \p A \text{\ace misurabile e } \abs{\p A}_3 = 0$$
\end{theorem}
\begin{theorem}[BDPG, 14.11]
  Sia $E\subseteq \R^3$ limitato e misurabile, sia $g \in \Rcal(E)$. Allora 
  $G_{g} := \{(x,y,g(x,y)) : (x,y)\in E\} \subseteq \R^3$ \ace misurabile e $\abs{G_{g}}_3 = 0$
\end{theorem}
\begin{theorem}[Esistenza integrale tripli su insiemi misurabili]
  Sia $f : A \subseteq \R^3 \to \R$. Supponiamo che:
  \begin{itemize}
    \item A sia limitato e misurabile
    \item f sia limitato e $f \in C^0 (A)$
  \end{itemize}
  Allora $f\in \Rcal(A)$
\end{theorem}
\subsection{Formule di riduzione per integrali tripli}
\begin{theorem}[Formule di riduzione su parallelepipedi][BDPG,14.26]
  Siano $Q = [a_1,b_1]\times [a_2,b_2]\times [a_3,b_3]$, $f \in C^0(Q)$
  \begin{enumerate}
    \item[(i)] La funzione $$[a_1,b_1]\times [a_2,b_2] \ni (x,y) \to \int_{a_3}^{b_3} f(x,y,z) \,dz$$
              \ace integrabile su $[a_1,b_1]\times [a_2,b_2]$ e 
              $$(1) \iiint_{Q} f = \iint_{[a_1,b_1]\times [a_2,b_2]} \left(\int_{a_3}^{b_3} f(x,y,z) \,dz\right)$$
    \item[(ii)] La funzione $$[a_1,b_1] \ni x \to \iint_{[a_2,b_2]\times [a_3,b_3]} f(x,y,z) \,dy\,dz$$
                \ace integrabile su $[a_1,b_1]$ e 
                $$(2) \iiint_{Q} f = \int_{a_1}^{b_1} \left(\iint_{[a_2,b_2]\times[a_3,b_3]} f(x,y,z) \,dz\right)$$
  \end{enumerate}
  La \begin{itemize}
    \item[(1)] si chiama formula di riduzione per \underline{fili}
    \item[(2)] si chiama formula di riduzione per \underline{strati}
  \end{itemize}
  \begin{itemize}
    \item L'insieme $\{(x,y,z)\in\R^3:(y,z)\in[a_2,b_2]\times[a_3,b_3]\}$ \ace uno strato.
    \item L'insieme $\{(x,y,z)\in\R^3: z\in[a_3,b_3]\}$ \ace un filo.
  \end{itemize}
\end{theorem}
Un'immediata conseguenza della formula di riduzione sui parallelepipedi e di quella per i rettangoli \ref{ridRettangoli}
segue il seguente risultato:
\begin{corollary}
  Sia $f \in C^0(Q)$ allora 
  $$\iiint_{Q} f = \int_{a_1}^{b_1} \left( \int_{a_2}^{b_2} \left(\int_{a_3}^{b_3} f(x,y,z) \,dz\right)\,dy\right)\,dx$$
\end{corollary}
\begin{definition}
  Un insieme $A \subseteq \R^3$ si chiama \underline{semplice} (o \underline{normale}) rispetto all'asse z se 
  $$(*) \, A=\{(x,y,z)\in\R^3: (x,y)\in E, g_1(x,y)\leq z \leq g_2(x,y)\}$$
  dove $g_1,g_2 : E \subseteq \R^2 \to \R$, $g_1,g_2 \in C^0(E)$
\end{definition}
\begin{osservazione}
  Definizioni analoghe si possono dare per insiemi semplici rispetto agli assi x ed y.
\end{osservazione}
\begin{theorem}[Formule di riduzione per integrali tripli rispetto a domini semplici][BDPG,14.28]
  Sia $A\subseteq \R^3$ un insieme semplice rispetto all'asse z di tipo $(*)$ e sia 
  $f\in C^0(A)$. Allora 
  $$(3)\, \iiint_{A}f = \iint_{E} \left(\int_{g_1(x,y)}^{g_2(x,y)} f(x,y,z) \,dz\right) \,dx\,dy$$
\end{theorem}
\begin{osservazione}
  La formula (3) \ace una formula di riduzione per fili che generalizza la formula (1) di quella sui parallelepipedi.\\
  La formula (2) di riduzione per strati su parallelepipedi pu\aco essere estesa ad insiemi pi\acu generali.
  Sia $A\subseteq \R^3$ limitato e misurabile e supponiamo che 
  $$(**) \, A = A \cap \left(\R\times[a,b]\times\R\right)$$
  e sia tale che, lo strato di A 
  $$(***) \, A_y = \{(x,z)\in\R^2:(x,y,z)\in A\} \text{ sia misurabile (come insieme del piano x,z)}$$
\end{osservazione}
\begin{theorem}[Formula di riduzione per strati][BDPG,14.28]
  Sia $A\subseteq \R^3$ verificante $(**)$ e $(***)$ e sia $f \in C^0(A)$ limitata. Allora
  $$(4) \, \iiint_{A}f = \int_{a}^{b} \left(\iint_{A_y} f(x,y,z) \,dx \,dz\right)\,dy$$
\end{theorem}
\subsubsection{Applicazione della formula di riduzione per strati: volume di un solido di rotazione}
Sia $A = \{(x,y,z)\in\R^3: z \in [c,d], x^2+y^2\leq f(z)^2\}$ dovve $f\in C^0([c,d])$, $f\geq 0$.\\
A pu\aco essere visto come il \underline{solido di rotazione} ottenuto ruotando l'insieme
$$F:=\{(y,z):z\in[c,d], 0\leq y \leq f(z)\}$$
attorno all'asse z.\\
Fissato $z \in [c,d]$ lo strato di A
$$A_z = \{(x,y)\in \R^2: x^2+y^2 \leq f(z)^2\}$$
rappresenta un cerchio (nel piano x,y) con centro (0,0) e raggio $f(z)$. \\
Quindi 
$$area(A_z) = \pi f(z)^2 \, \forall z \in [c,d]$$
Pertanto applicando la (4), otteniamo 
$$\abs{A}_3 = \iiint_A 1 \,dx\,dy\,dz = \int_{c}^{d} \left(\iint_{A_z} 1 \,dx\,dy\,dz\right) = $$
$$= \int_{c}^{d} area(A_z) \,dz = \pi \int_{c}^{d} f(z)^2 \,dz$$
(formula del volume di un solido di rotazione)
\subsection{Cambiamento di variabili negli integrali tripli}
Siano $D^*, D \subseteq \R^3$ aperti limitati e misurabili, e sia 
$\Psi : D^* \to \D$, $\Psi = (\Psi_1, \Psi_2, \Psi_3) = (x,y,z)$, 
$\Psi_i = \Psi_i(u,v,w):D^*\to\R$ (i=1,2,3).
\begin{definition}
  La mappa $\Psi$ si dice \underline{cambiamento di variabile} (in $\R^3$) Se
  \begin{itemize}
    \item[(i)] $\Psi$ \ace bigettiva
    \item[(ii)] $\Psi_i \in C^1(D^*)$, 
                $$\Psi_i, \frac{\p \Psi_i}{\p u}, \frac{\p \Psi_i}{\p v}, \frac{\p \Psi_i}{\p w} : D^* \to \R$$
                limitate (i = 1,2,3) 
    \item[(iii)] $\det D\Psi(u,v,w) \neq 0$, dove 
                $$D\Psi(u,v,w) := \begin{bmatrix}
                  \frac{\p \Psi_1}{\p u} & \frac{\p \Psi_1}{\p v} & \frac{\p \Psi_1}{\p w} \\
                  \\
                  \frac{\p \Psi_2}{\p u} & \frac{\p \Psi_2}{\p v} & \frac{\p \Psi_2}{\p w} \\
                  \\
                  \frac{\p \Psi_3}{\p u} & \frac{\p \Psi_3}{\p v} & \frac{\p \Psi_3}{\p w} \\
                \end{bmatrix}$$
                se $(u,v,w)\in\D^*$
  \end{itemize}
\end{definition}
\begin{theorem}[Cambiamento di variabili negli integrali tripli]
  Siano $D^*, D \subset \R^3$ aperti limitati e misurabili, sia 
  $\Psi : D^* \to D$ un cambiamento di variabili e sia $f \in C^0(D)$ e limitata.
  Allora 
  $$\iiint_{D} f(x,y,z) \,dx\,dy\,dz = \iiint_{D^*} f(\Psi(u,v,w)) \abs{\det D\Psi(u,v,w)} \,du\,dv\,dw$$
\end{theorem}
\begin{definition}
  Coordinate sferiche:
  $$\Psi \equiv \left\{\begin{array}{l}
    x = r \sin\varphi\cos\vartheta \\
    y = r \sin\varphi\sin\vartheta \\
    z = r \cos\varphi \\
  \end{array}\right.$$
  $0 \leq \vartheta\leq 2\pi, r\geq 0, 0 \leq \varphi \leq \pi$, 
  $\abs{\det D\Psi(r,\vartheta,\varphi)} = r^2 \sin\varphi$
\end{definition}