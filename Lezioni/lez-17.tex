\chapter{Superfici ed integrali di superfici, [BDPG,15]}
\section{Lez - 17, Superfici in $\R^3$}
Intuitivamente una superficie nello spazio \ace un oggetto bidimensionale, senza spessore. \\
Prima delle definizione intriduciamo due esemi di superfici note. 
\begin{example}
  \begin{enumerate}
    \item Sia $D=\{(x,y): x^2+y^2 < 1\}$ e sia $f : \overline{D} \to \R$ la funzione 
          definita da $f(x,y) = x^2+y^2$. \\
          Il grafico di f, 
          $$S_1 = G_f = \{(x,y,f(x,y)): (x,y)\in\overline{D}\}$$
          (porzione di paraboloide) \\\\
          $S_1$ pu\aco essere vista come l'immagine della mappa (detta \underline{parametrizzazione})
          $$\sigma : \overline{D} \subseteq \R^2\to\R^3 , \, \sigma(x,y) = (x,y,x^2+y^2)$$
    \item La sfera di raggio 1 e centro (0,0,0) in $\R^3$ 1ace il sottoinsieme definito da 
          $$S_2 = \{(x,y,z): x^2+y^2+z^2=1\}$$
          \ac{E} noto che $S_2$ non pu\aco essere visto come il grafico di una funzione di due 
          variabili, ma pu\aco essere visto come immagine di una parametrizzazione. \\
          Per esempio, una parametrizzazione di $S_2$ pu\aco essere ottenuta tramite le 
          coordinate sferiche:
          $$\left\{\begin{array}{l}
            x = \cos\vartheta\sin\varphi \\
            y = \sin\vartheta\sin\varphi \\
            z = \cos\varphi \\
          \end{array}\right.$$
          dove $\vartheta \in [0,2\pi], \varphi \in [0,\pi]$. \\
          Pi\acu precisamente $S_2$ \ace l'immagine della mappa 
          $\sigma : [0,2\pi]\times[0,\pi] \subseteq \R^2 \to \R^3$
          $$\sigma(\vartheta,\varphi) := (\cos\vartheta\sin\varphi, \sin\vartheta\sin\varphi, \cos\varphi)$$
  \end{enumerate} 
\end{example}
Prima della definizione di superficie, premettiamo la nozione di curva di Jordan nel piano.
\begin{definition}
  Una \underline{curva di Jordan} \ace una curva piana $\gamma : [a,b]\to\R^2$ semplice e chiusa.
\end{definition}
\begin{theorem}[BDPG,p. 359]
  Sia $\g$ una curva di Jordan. Allora valgono le seguenti prop.
  \begin{enumerate}
    \item Il \underline{sostegno} $\Gamma = \g([a,b])$ divide il piano in due insiemi aperti
          di cui uno \ace limitato, e si chiama \underline{interno della curva} ($D_{int}$), e l'altro
          \ace illimitato, e si chiama \underline{esterno della curva} ($D_{est}$)
    \item $\Dint,\Dest$ hanno la stessa frontiera e coincide con $\Gamma$
  \end{enumerate}
\end{theorem}
\begin{definition}
  Un sottoinsieme $S\subset\R^3$ si dice \underline{superficie} (elementare) se esiste una mappa
  $\sigma:\overline{D}\subseteq\R^2\to\R^3$, 
  $$\sigma(u,v) = (x(u,v),y(u,v),z(u,v))$$
  verificante
  \begin{enumerate}
    \item D \ace un aperto di $\R^2$, interno di una curva di Jordan
    \item $\sigma$ \ace continua e $\sigma:D\to\R^3$ \ace iniettiva
    \item $\sigma(\overline{D})=S$
  \end{enumerate}
  Una funzione verificante (1.-3.) \ace detta parametrizzazione di S. \\\\
  S si dice \underline{superficie cartesiana} se esiste una parametrizzazione
  $\sigma : \overline{D}\subseteq\R^2\to\R^3$ del tipo
  $$\sigma(u,v) = \begin{array}{lr}
    (u,v,f(u,v)) & z = f(x,y) \\
    \text{oppure} \\
    (f(u,v),u,v) & x = f(y,z) \\
    \text{oppure} \\
    (u,f(u,v),v) & y = f(x,z) \\
  \end{array} (u,v)\in\overline{D}$$
  dove $f:\overline{D}\to\R$ continua.
\end{definition}
\begin{osservazione}
  \begin{enumerate}
    \item Si noti che, a differenza della nozione di curva, nella nozione di superficie \ace all'immagine
          della parametrizzazione che si assegna il nome "superficie" e non alla parametrizzazione
    \item Le superifici considerate sono \underline{limitate}. \\
          Per includere superfici illimitate si necessiterebbe nella definizione di un cambiamento 
          che non \ace tratto nel corso.
    \item Data una superficie $S\subseteq\R^3$, la parametrizzazione di S non \ace unica.
  \end{enumerate}
\end{osservazione}
\begin{exercise}
  Si provvi che la mappa 
  $\sigma^*(u,v) = (u\cos{v},u\sin{v}, u^2)$, $(u,v)\in[0,1]\times[0,2\pi]$
  \ace un'altra parametrizzazione della superficie $S_1$, con $D:= (0,1)\times(2\pi)$
\end{exercise}
\subsection{Punti interni e bordo di una superficie}
Vogliamo ora precisare la nozione di bordo e punti interni per una superficie 
$S\subseteq\R^3$ da un punto di vista intrinseco. \\
Osserivamo che la precisazione "da un punto di vista intrinseco" evidenzia la differenza
con le nozioni di frontiera e parte interna di S, visto come sottoinsieme di $\R^3$. \\
Infatti, si pu\aco provare che, data $S\subseteq\R^3$ superficie, allora 
$$\p S = S \text{ e } \overcirc{S} = \varnothing$$
D'altra parte \ace abbastanza intuitivo ritenere che, per esempio, il bordo (intrinseco) della 
porzione di paraboloide ($S_1$), sia la circonferenza 
$$\{(x,y,1):x^2+y^2=1\}$$
mentre il bordo (intrinseco) della sfera sia $\varnothing$. \\\\
Vogliamo introdurre due nozioni che formalizzino questa intuizione.
\begin{definition}
  Sia $S$ una superficie
  \begin{enumerate}
    \item Un puno $p\in S$ si dice \underline{interno a S} se esiste un 
          $B(p,r)$ ed una parametrizzazione $\sigma^* : \overline{D^*}\subseteq\R^2\to\R^3$
          di $\overline{B(p,r)\cap S}$ tale che $p\in \sigma^*(D^*)$. \\
          L'insime dei punti interni \ace denotato da: S'
    \item Si chiama \underline{bordo di S} e si denota $bor(S)$ l'insieme dei punti 
          che non sono interni ad S, cio\ace $bor(S) = S \setminus S'$
  \end{enumerate}
\end{definition}
\begin{example}
  \begin{enumerate}
    \item Si potrebbe provare che $S_1$ ha come punti interni l'insieme
          $$S_1' = \{(x,y,x^2+y^2):x^2+y^2<1\}$$
          mentre 
          $$bor(S_1) = \{(x,y,x^2+y^2):x^2+y^2=1\}$$
    \item La superficie $S_2$ ha come insieme dei punti interni tutti i punti, $S_2' = S_2$ 
          mentre $bor(S_2) = \varnothing$
  \end{enumerate}
\end{example}
\section{Regolarit\aca della parametrizzazione e piano tangente ad una superficie}
Possiamo intuire, tenendo presente il caso delle curve, che la regolarit\aca delle parametrizzazioni
di una superficie potrebbe non bastare per l'esistenza del piano tangente ad una superficie. \\
In effett, come vedremo, si possono costruire superifci che ammettono 
parametrizzazione di classe $C^1$ e non ammettono piano tangente in qualche punto.\\
Quindi necessiter\aca individuare, come nel caso delle curve, una condizione aggiuntiva alla regolarit\aca 
$C^1$ della parametrizzazione di una superficie, per l'esistenza del piano tangente. \\\\
Per capire quale sia questa condizione aggiuntiva, ci aiuteremo con un argomento geometrico
che utilizza la nozione di tangente di una curva. \\\\
Sia $S$ superficie, sia $\sigma:\overline{D}\to\R^3$ una sua parametrizzazione di classe $C^1$\\
L'esistenza di un piano tangente $\pi$ a S in un punto interno 
$p_0=\sigma(u_0,v_0)$ con $(u_0,v_0)\in D$ dovrebbe implicare la seguente propriet\aca:
se $\g :[a,b]\to D$ curva di classe $C^1$ con $\g'(t_0) \neq (0,0) \Rightarrow
\gtilde = \sigma\circ\g:[a,b]\to S$ \ace ancora di classe $C^1$ con 
$\gtilde'(t_0)\neq (0,0,0)$ e la retta tangente alla curva $\gtilde$, passante 
per $\gtilde(t_0)$, deve appartenere a $\pi$. \\
Siano $\g:[a,b]\to D$ di classe $C^1$ con $\g'(t_0)\neq (0,0)$ 
e $\g(t) = (u(t),v(t))$, $t\in[a,b]$, $\g(t_0)=(u_0,v_0)$, 
$$\sigma(u,v):= (x(u,v),y(u,v),z(u,v))$$ con $(u,v)\in\overline{D}$, 
$$\gtilde(t) := \sigma(\g(t)) = (x(\g(t)), y(\g(t)), z(\g(t)))$$ per RDC.
\begin{exercise}
  $(1) \, \g'(t_0) = u'(t_0)\sigma_u(u_0,v_0) + v'(t_0)\sigma_v(u_0,v_0)$
  dove 
  $$\sigma_u(u,v) := \left(\frac{\p x}{\p u}, \frac{\p y}{\p u}, \frac{\p z}{\p u}\right)(u,v)$$
  $$\sigma_v(u,v) := \left(\frac{\p x}{\p v}, \frac{\p y}{\p v}, \frac{\p z}{\p v}\right)(u,v)$$
  Vogliamo imporre che $\gtilde'(t_0)\neq(0,0,0)$. \\
  Essendo $\g'(t_0) \neq (0,0)$, allora da (1) segue 
  $$(2) \, \sigma_u(u_0,v_0) \text{ e } \sigma_v(u_0,v_0) \text{ sono L.I.}$$
  Ricordiamo ora che:
  $$(2) \iff \sigma_u(u_0,v_0)\wedge\sigma_v(u_0,v_0)\neq(0,0,0)$$
  dove dati $w=(w_1,w_2,w_3), z = (z_1,z_2,z_3) \in \R^3$
  $$w\wedge z = \det\begin{bmatrix}
    e_1 & e_2 & e_3 \\
    w_1 & w_2 & w_3 \\
    z_1 & z_2 & z_3 \\
  \end{bmatrix} = \left(w_2z_3 - z_2w_3, w_1z_3+z_1w_3, w_1z_2-z_1w_2\right) \in \R^3$$ 
  (prodotto vettore di w e z) \\
  Ricordiamo inoltre che valgono le seguenti propriet\aca
  \begin{itemize}
    \item $w\wedge z$ \ace ortogonale sia a w che a z;
    \item $\norma{w\wedge z} = \norma{w}\norma{z}\sin{\alpha}$;
    \item $w\wedge z = (0,0,0) \iff$ w e z sono paralleli;
  \end{itemize}
  Consideriamo ora il piano $\pi \subset \R^3$ definito da 
  $$\pi := \{\sigma(u_0,v_0) + \lambda \sigma_u(u_0,v_0) + \mu \sigma_v(u_0,v_0) : \lambda, \mu \in \R\}$$
  (eq. parametrica di un piano) \\\\
  Osserivamo che $\pi$ \ace il piano di eq.:
  $$a(x-x_0)+b(y-y_0)+c(z-z_0) = 0$$
  dove, $(a,b,c) := \sigma_u(u_0,v_0) \wedge \sigma_v(u_0,v_0) \neq (0,0,0)$, 
  $(x_0,y_0,z_0 ):= \sigma(u_0,v_0)$ \\
  Infatti, basta osservare che, se $(x,y,z)\in \pi$, allora $(x,y,z)$
  verifica (*), allora, per propriet\aca del prodotto vettore, esistono
  $\lambda,\mu \in \R$ t.c. 
  $$(x,y,z) = \sigma(u_0,v_0) + \lambda \sigma_u(u_0,v_0) + \mu \sigma_v(u_0,v_0) : \lambda, \mu \in \R$$
  Infine si osservi che la retta tangente alla curva $\gtilde$, passante pr 
  $\gtilde(t_0)$, di eq. parametrica 
  $$(x,y,z) = \sigma(u_0,v_0) + \gtilde'(t_0)(t-t_0) = 
      \sigma(u_0,v_0) + \left(u'(t_0)\sigma_u(u_0,v_0) + v'(t_0)\sigma_v(u_0,v_0)\right)(t-t_0)$$
  $$= \sigma(u_0,v_0) + u'(t_0)\sigma_u(u_0,v_0)(t-t_0) + v'(t_0)\sigma_v(u_0,v_0)(t-t_0) \in \pi$$
  $\forall t \in \R$. Dunquq essa \ace contenuta in $\pi$
\end{exercise}
\begin{definition}
  Sia S superfice e sia $p_0\in S'$
  \begin{enumerate}
    \item Il punto $p_0$ si dice \underline{regolare} se esistono 
          $B(p_0,r_0)$ ed una parametrizzazione $\sigma:\overline{D}\to\R^3$ di 
          $\overline{B(p_0,r_0)\cap S}$ t.c. 
          \begin{enumerate}
            \item[1.1] $\sigma$ \ace di classe $C^1$
            \item[1.2] vale (2) per il punto $(u_0,v_0)\in D$ t.c. $\sigma(u_0,v_0)=p_0$.\\
                        In tal caso il piano $\pi$ si chiama \underline{piano tangente}
                        a S nel punto $p_0$.\\
                        Le due direzioni (o versori)
                        $$\pm \frac{\sigma_u(u_0,v_0)\wedge\sigma_v(u_0,v_0)}
                                {\norma{\sigma_u(u_0,v_0)\wedge \sigma_v(u_0,v_0)}}$$
                        si chiamano \underline{direzioni} (o versori) normali a S in $p_0$
          \end{enumerate}
    \item S si dice \underline{regolare}, se tutti i punti interni di S sono regolari.
  \end{enumerate}
\end{definition}
\begin{osservazione}
  Sia $D\subseteq\R^2$ intorno di una curva di Jordan e sia $f\in C^0(\overline{D})\cap C^1(D)$. 
  Consideriamo la superficie cartesiana
  $$S=G_f = \{(x,y,f(x,y)) : (x,y) \in \overline{D}\}$$
  Allora 
  \begin{exercise}
    \begin{enumerate}
      \item ogni punto $p_0 = (x_0,y_0,f(x_0,y_0))$ con $(x_0,y_0)\in D$ \ace interno di S regolare.
      \item L'eq. del piano tangente a S nel punto $p_0$ \ace data da 
            $$z = f(x_0,y_0) + \frac{\p f}{\p x}(x_0,y_0) (x-x_0) + \frac{\p f}{\p y}(x_0,y_0)(y-y_0)$$
    \end{enumerate}
    \textbf{Soluzione:} Si consideri la parametrizzazione di S data da 
    $$\sigma(u,v) = (u,v,f(u,v))$$
    $(u,v)\in\overline{D}$
    \begin{exercise}
      Provare che
      $$\sigma_u(u_0,v_0) \wedge \sigma_v(u_0,v_0) = \left(-\p_u f(u,v), -\p_v f(u,v), 1\right) \, (u,v)\in D$$
    \end{exercise}
    Da ci\aco segue subito il punto uno. \\ 
    Inoltre ricordando l'eq cartesiana del piano tangente a S nel punto $p_0$ (vedi (*)), segue 
    subito il punto due. 
  \end{exercise}
\end{osservazione}