\section{Lez - 07}
\subsection{Derivate parziali di ordine superiore}
Sia $\f$, A aperto. Supponiamo che $\exists \frac{\p f}{\p x}, \frac{\p f}{\p y}: A \to \R$
poniamo:
$$\left.\begin{array}{l}
  \frac{\p^2 f}{\p x^2} := \frac{\p f}{\p x} \left(\frac{\p f}{\p x}\right) \\
  \\
  \frac{\p^2 f}{\p y^2} := \frac{\p f}{\p y} \left(\frac{\p f}{\p y}\right) \\
\end{array}\right\} \text{Derivate parziali seconde pure}$$
$$\left.\begin{array}{l}
  \frac{\p^2 f}{\p x \p y} := \frac{\p f}{\p x} \left(\frac{\p f}{\p y}\right) \\
  \\
  \frac{\p^2 f}{\p y \p x} := \frac{\p f}{\p y} \left(\frac{\p f}{\p x}\right) \\
\end{array}\right\} \text{Derivate parziali seconde miste}$$
quando tutte le derivate parziali scritte esistono.
\begin{osservazione}
  In generale $\frac{\p^2 f}{\p x \p y} \neq \frac{\p^2 f}{\p y \p x}$
\end{osservazione}
\subsection{Teo: Inversione dell'ordine di derivazione}
\begin{theorem}[sull'inversione dell'ordine di derivazione][BDPG, 11.11]
  Sia $\f$, A aperto, $p_0 = (x_0,y_0)\in A$ fissato. Supponiamo $\exists \frac{\p^2 f}{\p x \p y}, \frac{\p^2 f}{\p y \p x} 
  : A \to \R$ e siano continue in $p_0$, allora $\frac{\p^2 f}{\p x \p y}(p_0) 
  = \frac{\p^2 f}{\p y \p x} (p_0)$
\end{theorem}
Il teorema precedente pu\aco estendersi al caso di funzioni $n\geq 2$ variabili. \\\\
Sia $\fn$, A aperto. Supponimao che esiset $\frac{\p f}{\p x_i} :A \to \R$ ($i = 1,\dots,n$). \\
Se $\exists \frac{\p }{\p x_j}\left(\frac{\p f}{\p x_i}\right)(x_0)$ in un punto $x_0 \in A$ per 
$j = 1,\dots,n$, diciamo che 
$$\exists \frac{\p^2 f}{\p x_j \p x_i} (x_0) := \frac{\p f}{\p x_j}\left(\frac{\p f}{\p x_i}\right)(x_0)$$
Nel caso in cui $j=i \Rightarrow \frac{\p^2 f}{\p x_i^2} = \frac{\p }{\p x_i}\left(
\frac{\p f}{\p x_i}\right) = \frac{\p^2 f}{\p x_i \p x_i}(x_0)$. \\
Con queste notazioni, vale la seguente generalizzazione del teorema sull'inversione dell'ordine di derivazione.
\begin{theorem}
  Sia $\fn$, A aperto. Supponimao che per fissati $i,j = 1,..., n$, con $i \neq j$, 
  $\exists \frac{\p^2 f}{\p x_i \p x_j}, \frac{\p^2 f}{\p x_j \p x_i} : A \to \R$ e siano 
  continue in $x_0$. Allora:
  $$\frac{\p^2 f}{\p x_i \p x_j} (x_0) = \frac{\p^2 f}{\p x_j \p x_i} (x_0)$$
\end{theorem}
\begin{definition}
  Sia $\fn$, A aperto
  \begin{itemize}
    \item[(a)] f si dice di \underline{classe} $C^2(A)$ e scriveremo $f\in C^2(A)$ se $f \in C^0(A)$
              ed $\exists \frac{\p f}{\p x_i} : A \to \R$ continua $\forall i = 1,..., n$, 
              $\exists \frac{\p^2 f}{\p x_i \p x_j} : A \to \R$ continua $\forall i,j = 1,..., n$
    \item[(b)]  f si dice di \underline{classe} $C^m(A)$ e scriveremo $f \in C^m(A), (m \geq 1)$ se $f \in C^0(A)$
              e $\exists \frac{\p^k f}{\p x_{i_1} \p x_{i_2} \dots \p x_{i_k}}:A \to \R$ continua $\forall i_1, ..., i_k = 1,..., n$
               e $\forall 1 \leq k \leq n$
  \end{itemize}
\end{definition}
\begin{osservazione}
  Se $f \in C^m(A)$, con $m\geq 2$ per il teo sull'inversione dell'ordine di derivazione 
  $$\frac{\p^2 f}{\p x_j \p x_i}(x) = \frac{\p^2 f}{\p x_i \p x_j}(x)$$
  $\forall x \in A$, $\forall i,j = 1,..., n$
\end{osservazione}
\subsection{Taylor per funzioni di pi\acu variabili}
\textbf{Problema:} Data $f: B(p_0,r) \subset \R^2 \to \R$ funzione di classe $C^m\left(B(p_0,r)\right)$, 
approssimare f con un polinomio di $n=2$ variabili di ordine m, nel modo "migliore possibile"
\begin{definition}
  Dato $m \in \Ins{N}$, $p_0 = (x_0,y_0) \in \R^2$ fissato, si chiama \underline{polinomio di ordine m} di $n=2$ variabili,
  centrato in $p_0$, una funzione $T: \R^2 \to \R$ del tipo
  $$T(x,y) = \sum_{h=0}^{m} \sum_{i = 0}^n c_{i,h-i} (x-x_0)^i(y-y_0)^{h-i}$$
  $(x,y)\in \R^2$, dove $c_{i,h-i}$ (i = 0,...,h e h = 0,..., m) sono $\frac{(m+1)(m+2)}{2}$ coeff. ass. 
\end{definition}
\begin{example}
  \begin{itemize}
    \item[(a)] Se m = 0, $T(x,y) = c_{0,0} \in \R \, \forall (x,y) \in \R^2$
    \item[(b)] Se m = 1, $T(x,y) = c_{0,0} + c_{1,0}(x-x_0)+ c_{0,1}(y-y_0)$, $\forall (x,y)\in \R^2$
    \item[(c)] Se m = 2, $T(x,y) = c_{0,0} + c_{1,0}(x-x_0)+ c_{0,1}(y-y_0) + c_{2,0}(x-x_0)^2 + c_{1,1}(x-x_0)(y-y_0)
                + c_{0,2}(y-y_0)^2$, $\forall (x,y)\in \R^2$  
  \end{itemize}
\end{example}
\textbf{Problema:} Sia $f \in C^2\left(B(p_0,r)\right)$, determinare se esiste un polinomio $T: \R^2 \to \R$ di 
ordine 2, centrato in $p_0$, t.c. $$f(p) = T(p) + o\left(\norma{p-p_0}^2\right)$$
$\forall p = (x,y) \in B(p_0,r)$
\begin{notazione}
  Se $v,w \in \R^n$, $v \cdot w = \tuple{v,w}$
\end{notazione}
\begin{definition}
  Data $f \in C^2(A)$, $A \in \R^2$ aperto, si chiama, \underline{matrice hessiana}
  di f in un punto $p\in A$, la matrice $2\times 2$
  $$D^2f(p) = H(f)(p) = \begin{bmatrix}
    \frac{\p^2 f}{\p x^2}(p) & \frac{\p^2 f}{\p y \p x}(p) \\
    \frac{\p^2 f}{\p x \p y}(p) & \frac{\p^2 f}{\p y^2}(p) \\
  \end{bmatrix}_{2\times 2}$$
\end{definition}
\begin{osservazione}
  per il teo. dell'inv. dell'ordine di derivazione $D^2 f(p)$ \ace \underline{simmetrica}
\end{osservazione}
\subsection{Taylor del II ordine + resto di Peano}
Sia $f \in C^2\left(B(p_0,r)\right)$, $p_0 = (x_0,y_0) \in \R^2$ e $r > 0$ fissato. Allora vale:
$$\left(FT_2\right) f(p) = T_2(p) + o\left(\norma{p-p_0}^2\right)$$
$\forall p = (x,y) \in B(p_0,r)$, dove 
$$T_2(p) := f(p_0) + \tuple{\nabla f(p_0), p-p_0} + \frac{1}{2} \tuple{D^2f(p_0) \cdot (p-p_0), p-p_0}$$
se $p \in \R^2$. \\
(polinomio di taylor del II ordine di f, centrato in $p_0$) \\
Ricordiamo che con $o\left(\norma{p-p_0}^2\right) \Rightarrow \exists\lim_{p\to p_0} \frac{o\left(\norma{p-p_0}^2\right)}{\norma{p-p_0}^2} = 0$
\begin{proof}
  Fissiamo $p \in B(p_0,r)\setminus \{p_0\}$ e denotiamo $v:= \frac{p-p_0}{\norma{p-p_0}} = (v_1,v_2)$, (direzione $p-p_0$)
  e definiamo: $F(t) := f(p_0+tv)$, con $t \in \left(-r,r\right)$. \\
  Poich\ace la funzione $g: (-r,r) \to B(p_0,r) \subseteq \R^2$, $g(t) = p_0+tv := (x_0+tv_1, y_0+tv_2)$ \ace una funzione di 
  classe $C^2\left((-r,r)\right)$, come pure f, per RDC la funzione composta:
  $F(t) = f(g(t))$, $t \in (-r,r)$, \ace di classe $C^2((-r,r))$. Pertanto possiamo applicare la formula di Taylor
  del II ordine per una funzione di una variabile per $t=0$ e otteniamo:
  $$(1) \, F(t) = F(0) + F'(0)t + \frac{1}{2}F''(0)t^2 + o(t^2) \, \text{ per } t \to 0$$
  Calcoliamo $F(0), F'(0), F''(0)$. Per RDC:
  \begin{itemize}
    \item $$F'(t) = \tuple{\nabla f(p_0+tv), v} = \frac{\p f}{\p x}(p_0+tv) v_1 + \frac{\p f}{\p y}(p_0+tv) v_2$$
    \item $$F''(t) = v_1 \cdot \tuple{\nabla \left(\frac{\p f}{\p x}\right)(p_0+tv), v} + 
              v_2 \cdot \tuple{\nabla \left(\frac{\p f}{\p y}\right)(p_0+tv), v} = $$
          $$= v_1 \left(\frac{\p^2 f}{\p x^2}(p_0+tv)v_1 + \frac{\p^2 f}{\p y \p x}(p_0+tv)v_2\right) + 
              v_2 \left( \frac{\p^2 f}{\p x \p y}(p_0+tv)v_1 + \frac{\p^2 f}{\p y^2}(p_0+tv)v_2\right) = $$
          $$ = \frac{\p^2 f}{\p x^2}(p_0+tv)v_1^2 + 2 \frac{\p^2 f}{\p y \p x}(p_0+tv)v_1 v_2 + 
              \frac{\p^2 f}{\p y^2}(p_0 + tv)v_2^2$$
  \end{itemize}
  Pertanto \begin{itemize}
    \item[(2)] $F(0) = f(p_0)$
    \item[(3)] $F'(0) = \tuple{\nabla f(p_0), v}$
    \item[(4)] $F''(0) = \frac{\p^2 f}{\p x^2}(p_0+tv)v_1^2 + 2 \frac{\p^2 f}{\p y \p x}(p_0+tv)v_1 v_2 + 
    \frac{\p^2 f}{\p y^2}(p_0 + tv)v_2^2$  
  \end{itemize}
  Osserviamo che $F''(0)$ pu\aco essere riscritto mediante hessiana $D^2f(p_0)$, come $(4bis) F''(0) = \tuple{D^2f(p_0)v,v}$, 
  pertanto da (1), (2), (3), (4bis) otteniamo:
  $$f(p_0+tv) = F(t) = f(p_0) + \tuple{\nabla f(p_0), v}t + \frac{1}{2} \tuple{D^2 f(p_0)v, v}t^2 + o(t^2) , \text{ per } t \to 0$$
  Scegliendo $t = \norma{p-p_0}$, otteniamo la tesi.
\end{proof} 
\begin{osservazione}[BDPG,11.14]
  Si pu\aco ottenere una formula di taylor con resto di Peano e Lagrange anche per funzioni $f : B(p_0,r) \subseteq \R^n \to \R$
  di classe $C^2(B(p_0,r))$ con $n \geq 3$.\\
  Essa \ace molto pi\acu complicata perci\aco la omettiamo
\end{osservazione}
\begin{exercise}
  $$f : \R \to \R, f(x) = \left\{\begin{array}{cl}
    0 & x = 0 \\
    x^2 \sin(\frac{1}{x}) & x \neq 0 \\
  \end{array}\right.$$
  $\exists f' : \R \to \R$, ma f' non \ace continua nel punto $x_0=0$
\end{exercise}