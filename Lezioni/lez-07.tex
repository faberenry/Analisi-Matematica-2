\section{Lez - 07}
\subsection{Derivate parziali di ordine superiore}
Sia $\f$, A aperto. Supponiamo che $\exists \frac{\p f}{\p x}, \frac{\p f}{\p y}: A \to \R$
poniamo:
$$\left.\begin{array}{l}
  \frac{\p^2 f}{\p x^2} := \frac{\p f}{\p x} \left(\frac{\p f}{\p x}\right) \\
  \\
  \frac{\p^2 f}{\p y^2} := \frac{\p f}{\p y} \left(\frac{\p f}{\p y}\right) \\
\end{array}\right\} \text{Derivate parziali seconde pure}$$
$$\left.\begin{array}{l}
  \frac{\p^2 f}{\p x \p y} := \frac{\p f}{\p x} \left(\frac{\p f}{\p y}\right) \\
  \\
  \frac{\p^2 f}{\p y \p x} := \frac{\p f}{\p y} \left(\frac{\p f}{\p x}\right) \\
\end{array}\right\} \text{Derivate parziali seconde miste}$$
quando tutte le derivate parziali scritte esistono.
\begin{osservazione}
  In generale $\frac{\p^2 f}{\p x \p y} \neq \frac{\p^2 f}{\p y \p x}$
\end{osservazione}
\subsection{Teo: Inversione dell'ordine di derivazione}
\begin{theorem}[sull'inversione dell'ordine di derivazione][BDPG, 11.11]
  Sia $\f$, A aperto, $p_0 = (x_0,y_0)\in A$ fissato. Supponiamo $\exists \frac{\p^2 f}{\p x \p y}, \frac{\p^2 f}{\p y \p x} 
  : A \to \R$ e siano continue in $p_0$, allora $\frac{\p^2 f}{\p x \p y}(p_0) 
  = \frac{\p^2 f}{\p y \p x} (p_0)$
\end{theorem}
Il teorema precedente pu\aco estendersi al caso di funzioni $n\geq 2$ variabili. \\\\
Sia $\fn$, A aperto. Supponimao che esiset $\frac{\p f}{\p x_i} :A \to \R$ ($i = 1,\dots,n$). \\
Se $\exists \frac{\p }{\p x_j}\left(\frac{\p f}{\p x_i}\right)(x_0)$ in un punto $x_0 \in A$ per 
$j = 1,\dots,n$, diciamo che 
$$\exists \frac{\p^2 f}{\p x_j \p x_i} (x_0) := \frac{\p f}{\p x_j}\left(\frac{\p f}{\p x_i}\right)(x_0)$$
Nel caso in cui $j=i \Rightarrow \frac{\p^2 f}{\p x_i^2} = \frac{\p }{\p x_i}\left(
\frac{\p f}{\p x_i}\right) = \frac{\p^2 f}{\p x_i \p x_i}(x_0)$. \\
Con queste notazioni, vale la seguente generalizzazione del teorema sull'inversione dell'ordine di derivazione.
\begin{theorem}
  
\end{theorem}
\subsection{Taylor per funzioni di pi\acu variabili}
\subsection{Taylor del II ordine + resto di Peano}