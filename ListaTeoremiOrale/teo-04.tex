\subsection{Definizione di derivate parziali e di vettore gradiente per una
funzione $\f$ A aperto}
\begin{definition}
  \begin{enumerate}
    \item Si dice che $f$ \ace \underline{derivabile}(parzialmente) rispetto alla variabile x nel punto $p_0 = (x_0,y_0)$ se 
          $$\exists \lim_{x \to x_0} \frac{f(x,y_0) - f(x_0,y_0)}{x-x_0} := \frac{\partial f}{\partial x}(x_0,y_0) = D_1 f(x_0,y_0) \in \R$$
    \item Si dice che $f$ \ace \underline{derivabile}(parzialmente) rispetto alla variabile y nel punto $p_0 = (x_0,y_0)$ se 
          $$\exists \lim_{y \to y_0} \frac{f(x_0,y) - f(x_0,y_0)}{y-y_0} := \frac{\partial f}{\partial y}(x_0,y_0) = D_2 f(x_0,y_0) \in \R$$
    \item Se $f$ \ace derivabile (parzialmente) sia rispetto ad x ed y nel punto $p_0 = (x_0,y_0)$, si chiama (vettore)\underline{gradiente} di $f$ in $p_0$
          il vettore:
          $$\nabla f(p_0) = \left(\frac{\partial f}{\partial x}(p_0), \frac{\partial f}{\partial y}(p_0)\right) \in \R^2$$
  \end{enumerate}
  Sia $\f$, A insieme aperto. Supponiamo che:
  $$\exists \frac{\partial f}{\partial x},\frac{\partial f}{\partial y} : A \to \R$$
  allora \ace ben definito il \underline{campo} dei vettori gradiente:
  $$\nabla f : \R^2 \supseteq A \ni p \to \nabla f(p) = \left(\frac{\partial f}{\partial x}(p), \frac{\partial f}{\partial y}(p)\right) \in \R^2$$
\end{definition}