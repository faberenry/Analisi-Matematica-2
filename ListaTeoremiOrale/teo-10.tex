\subsection{Teorema di Fermat sui punti stazionari di una funzione}
\begin{theorem}[Fermat]
  Sia $\fn$, A aperto. Supponiamo che esista $p_0 \in A$ t.c. 
  \begin{enumerate}
    \item[(i)] f differenziabile in $p_0$. In particolare $\exists \nabla f(p_0)$
    \item[(ii)] $p_0$ sia un estremo libero di f in A
  \end{enumerate}
  Allora $\nabla f(p_0) = \underline{O}_{\R^n} = (0,...,0) \text{ (n-volte)}$
\end{theorem}
\begin{definition}
  Data $\fn$, A aperto, un punto $p_0\in A$ si chiama \underline{punto stazionario}(o \underline{critico}) 
  di f se f \ace differenziabile in $p_0$ e $\nabla f(p_0) = \underline{O}_{\R^n}$
\end{definition}