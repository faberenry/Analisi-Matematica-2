\section{Superfici}
\subsection{Definizione di superficie elementare di $\R^3$: parametrizzazione di una superficie}
\begin{definition}
  Un sottoinsieme $S\subset\R^3$ si dice \underline{superficie} (elementare) se esiste una mappa
  $\sigma:\overline{D}\subseteq\R^2\to\R^3$, 
  $$\sigma(u,v) = (x(u,v),y(u,v),z(u,v))$$
  verificante
  \begin{enumerate}
    \item D \ace un aperto di $\R^2$, interno di una curva di Jordan
    \item $\sigma$ \ace continua e $\sigma:D\to\R^3$ \ace iniettiva
    \item $\sigma(\overline{D})=S$
  \end{enumerate}
  Una funzione verificante (1.-3.) \ace detta parametrizzazione di S. \\\\
  S si dice \underline{superficie cartesiana} se esiste una parametrizzazione
  $\sigma : \overline{D}\subseteq\R^2\to\R^3$ del tipo
  $$\sigma(u,v) = \begin{array}{lr}
    (u,v,f(u,v)) & z = f(x,y) \\
    \text{oppure} \\
    (f(u,v),u,v) & x = f(y,z) \\
    \text{oppure} \\
    (u,f(u,v),v) & y = f(x,z) \\
  \end{array} (u,v)\in\overline{D}$$
  dove $f:\overline{D}\to\R$ continua.
\end{definition}