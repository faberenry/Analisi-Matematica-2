\subsection{Definizione di curva chiusa, semplice, regolare, orientazione (o
verso di percorrenza) di una curva semplice}
\begin{definition}
  \begin{itemize}
    \item[(i)] La curva $\gamma$ si dice \underline{chiusa} se $I = [a,b]$ e $\gamma(a) = \gamma(b)$
    \item[(ii)] La curva $\gamma:I\to\R^n$ si dice \underline{semplice} se $\gamma$ \ace iniettiva, o 
                se $\gamma$ \ace chiusa e $I = [a,b]$, allora $\gamma :[a,b) \to \R^n$ \ace iniettiva.
  \end{itemize}
\end{definition}
\begin{definition}
  Una curva $\gamma:I\to\R^n$ si dice \underline{regolare} se $\gamma$ 
  \ace di classe $C^1$ e $\gamma'(t) \neq \origine_{\R^n}$ $\forall t \in I$ 
\end{definition}
\begin{definition}
  Sia data una curva semplice $\gamma:I\to\R^n$. Allora 
  essa induce \underline{un'orientazione} sul suo sostegno $\gamma(I) \subseteq \R^n$.
  Pi\acu precisamente \\\\
  Data $\gamma:I\to\R^n$ curva semplice, si dice che il punto $x_1 = \gamma(t_1)$ \underline{precede}
  il punto $x_2 = \gamma(t_2)$ se $t_1 < t_2$. L'orientazione della curva viene detta
  anche \underline{verso} della curva.
\end{definition}