\subsection{Definizione di forma differenziale esatta e di potenziale di una
forma differenziale}
Sia $E\subseteq\R^n$ un insieme aperto e sia $\U \in C^1(E)$. Possiamo associare ad $\U$
la forma diff. 
$$d\U = \tuple{\nabla\U,dx} = \frac{\p \U}{\p x_1}\,dx_1+\dots+\frac{\p \U}{\p x_n}\,dx_n$$
che viene anche chiamata \underline{differenziale di $\U$} poich\ace coincide con la notazione
con cui indichiamo il differenziale di $\U$
\begin{definition}
  Sia $E\subseteq\R^n$ un aperto e sia $\om = \tuple{F,dx}$ dove
  $F:E\to\R^n$ di classe $C^0$. La forma $\om$ si dice \underline{esatta} in E 
  se esiste $\U:E\to\R$ di classe $C^1$ t.c.
  $$\nabla\U(x) = F(x) \, \forall x \in E$$
  o, equivalentemente, $d\U = \om$.\\
  In tal caso $\U$ \ace detta \underline{funzione potenziale} (o primitiva) di $\om$ in E.
\end{definition}

