\section{Integrali per funzioni in pi\acu variabili}
\subsection{Definizione di insieme insieme semplice (o normale) in $\R^2$
rispetto agli assi cartesiani}
\begin{definition}
  Un sottoinsieme $A \subset \R^2$ si dice 
  \begin{itemize}
    \item \underline{Dominio semplice} (o normale) rispetto all'asse y se esistono 
          $g_1, g_2 \in C^0([a,b])$ t.c. $g_1 \leq g_2$ su $[a,b]$ e 
          $$A = \{(x,y)\in\R^2 : x \in [a,b], g_1(x) \leq y \leq g_2(x)\}$$
    \item \underline{Dominio semplice} (o normale) rispetto all'asse x se esistono 
          $h_1, h_2 \in C^0([c,d])$ t.c. $h_1 \leq h_2$ su $[c,d]$ e 
          $$A = \{(x,y)\in\R^2 : y \in [c,d], h_1(y) \leq x \leq h_2(y)\}$$
  \end{itemize}
\end{definition}