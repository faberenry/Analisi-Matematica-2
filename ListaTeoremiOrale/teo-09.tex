\section{Massimi e minimi}
\subsection{Definizione di punto di massimo/minimo relativo, massimo/minimo
assoluto e punto di sella per una funzione $\f$}
\begin{definition}
  Data $\f$: 
  \begin{enumerate}
    \item $p_0 \in A$ si dice, punto di \underline{massimo} (= max) \underline{relativo} di f su A se 
          $\exists r_0 > 0$ t.c. $f(p) \leq f(p_0) \, \forall p \in A \cap B(p_0,r_0)$ \\
          Rispettivamente $p_0 \in A$ si dice, punto di \underline{minimo} (= min) \underline{relativo} di f su A se 
          $\exists r_0 > 0$ t.c. $f(p) \geq f(p_0) \, \forall p \in A \cap B(p_0,r_0)$
    \item $p_0 \in A$ si dice punto di \underline{massimo} (= MAX) \underline{assoluto} se 
          $\forall p \in A$, $f(p) \leq f(p_0)$ \\
          Rispettivamente $p_0 \in A$ si dice punto di \underline{minimo} (= MIN) \underline{assoluto} se 
          $\forall p \in A$, $f(p) \geq f(p_0)$ 
  \end{enumerate}
\end{definition}
\begin{definition}
  Sia $\fn$, A aperto. Un punto $p_0 \in A$ si dice \underline{punto di sella} se $p_0$ \ace un punto 
  stazionario di f e $f(p)-f(p_0)$ amette sia valori positivi che negativi in ogni intorno di $p_0$
\end{definition}